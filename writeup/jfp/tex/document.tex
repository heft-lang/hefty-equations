\documentclass[draft]{jfp}

\let\Bbbk\relax

\journaltitle{JFP}
\cpr{Cambridge University Press}
\doival{10.1017/xxxxx}

\lefttitle{Submitted for publication.}
\righttitle{Journal of Functional Programming}

\totalpg{\pageref{lastpage01}}
\jnlDoiYr{2021}

\usepackage{multicol}
\usepackage{xparse}
%\usepackage[capitalise,noabbrev]{cleveref}
%\usepackage{xcolor}
\usepackage{mathtools}
\usepackage{wasysym}
\usepackage{agda}
\usepackage{tex/agdadimmed}
\usepackage{newunicodechar}
\usepackage{mdframed}
\usepackage{wrapfig}
\usepackage{boxedminipage}
\usepackage{dsfont}
\usepackage{etoolbox}
\usepackage{natbib}

\usepackage{amsfonts}
\usepackage{amssymb}
\usepackage{stmaryrd}
\usepackage{scalerel}

\usepackage{tikz}
\usetikzlibrary{positioning,fit,calc}
\tikzset{block/.style={draw, thick, text width=3cm, minimum height=1.5cm, align=center}, line/.style={-latex}}
\usepackage{pgfplots} 
\newcommand{\todo}[1]{\textcolor{red}{\textbf{#1}}}


\DeclareRobustCommand{\VAN}[3]{#3}


%% 
%% %include agda.fmt
%% %include fmts/polycode.fmt 
%% %include fmts/agdabase.fmt
%% %include fmts/agda.fmt
%% %include fmts/syntax.fmt
%% %include fmts/misc.fmt
%%

%%
%% Agda typesetting commands shorthands, for
%% manual typesetting of inline code
%%

\newcommand{\af}{\AgdaFunction}
\newcommand{\un}{\AgdaUnderscore}
\newcommand{\ad}{\AgdaDatatype}
\newcommand{\ab}{\AgdaBound}
\newcommand{\ac}{\AgdaInductiveConstructor}
\newcommand{\aF}{\AgdaField}
\newcommand{\as}{\AgdaSymbol}
\newcommand{\ak}{\AgdaKeyword}
\newcommand{\ap}{\AgdaPrimitiveType}
\newcommand{\an}{\AgdaNumber}
\newcommand{\aC}{\AgdaComment}
\newcommand{\am}{\AgdaModule}


%%
%% Unicode for typesetting Agda code
%%%%%%%%%%%%%%%%%%%
%% AGDA UNICODE %%%
%%%%%%%%%%%%%%%%%%%

% hacks

\newunicodechar{◂}{} % for repeating defs from stdlib for example

%%
%% Symbols
%%

\newcommand\superequiv{\mathrel{\rlap{\raisebox{\fontdimen22\textfont2}{$=$}}\raisebox{-0.5\fontdimen22\textfont2}{$ = $}}}

\newunicodechar{×}{\ensuremath{\mathnormal\times}}
\newunicodechar{→}{\ensuremath{\mathnormal\to}}
\newunicodechar{←}{\ensuremath{\mathnormal\leftarrow}}
\newunicodechar{⟦}{\ensuremath{\mathnormal\llbracket}}
\newunicodechar{⟧}{\ensuremath{\mathnormal\rrbracket}}
\newunicodechar{ℕ}{\ensuremath{\mathbb{N}}}
\newunicodechar{ℤ}{\ensuremath{\mathbb{Z}}}
\newunicodechar{⊕}{\ensuremath{\mathnormal\oplus}}
\newunicodechar{∔}{\ensuremath{\mathnormal\dotplus}}
\newunicodechar{⋎}{\ensuremath{\mathnormal\curlyvee}}
\newunicodechar{⊞}{\ensuremath{\mathnormal\boxplus}}
\newunicodechar{⇒}{\ensuremath{\mathnormal\Rightarrow}}
\newunicodechar{⇛}{\ensuremath{\mathnormal\Rrightarrow}}
\newunicodechar{⟨}{\ensuremath{\mathnormal\langle}}
\newunicodechar{⟩}{\ensuremath{\mathnormal\rangle}}
\newunicodechar{∪}{\ensuremath{\mathnormal\cup}}
\newunicodechar{⦃}{\ensuremath{\mathnormal\{\mskip-4.5mu\mid}}
\newunicodechar{⦄}{\ensuremath{\mathnormal\mid\mskip-4.5mu\}}}
\newunicodechar{⊎}{\ensuremath{\mathnormal\uplus}}
% \newunicodechar{∗}{\ensuremath{\mathnormal\ast}}
\newunicodechar{✴}{\ensuremath{\mathnormal\ast}}
\newunicodechar{↦}{\ensuremath{\mathnormal\mapsto}}
\newunicodechar{≡}{\ensuremath{\mathnormal\equiv}}
\newunicodechar{∀}{\ensuremath{\mathnormal\forall}}
\newunicodechar{∙}{\ensuremath{\mathnormal\bullet}}
\newunicodechar{≣}{\ensuremath{\mathnormal\superequiv}}
\newunicodechar{▿}{\ensuremath{\mathnormal\triangledown}}
\newunicodechar{▸}{\raisebox{0.25ex}{\scaleobj{0.65}{\ensuremath{\mathnormal\blacktriangleright}}}}
\newunicodechar{∼}{\ensuremath{\mathnormal\sim}}
\newunicodechar{≤}{\ensuremath{\mathnormal\leq}}
\newunicodechar{↔}{\ensuremath{\mathnormal\leftrightarrow}}
\newunicodechar{⊂}{\ensuremath{\mathnormal\subset}}
\newunicodechar{∘}{\ensuremath{\mathnormal\circ}}
\newunicodechar{∃}{\ensuremath{\mathnormal\exists}}
\newunicodechar{↓}{\ensuremath{\mathnormal\downarrow}}
\newunicodechar{↑}{\ensuremath{\mathnormal\uparrow}}
\newunicodechar{⊔}{\ensuremath{\mathnormal\sqcup}}
\newunicodechar{⊢}{\ensuremath{\mathnormal\vdash}}
\newunicodechar{◇}{\ensuremath{\mathnormal\diamond}}
\newunicodechar{⊙}{\ensuremath{\mathnormal\odot}}
\newunicodechar{⊤}{\ensuremath{\mathnormal\top}}
\newunicodechar{⊥}{\ensuremath{\mathnormal\bot}}
\newunicodechar{∣}{\ensuremath{\mathnormal\mid}}
\newunicodechar{‵}{\ensuremath{^\backprime}}
\newunicodechar{′}{\ensuremath{^\prime}}
\newunicodechar{″}{\ensuremath{^{\prime\prime}}}
\newunicodechar{‴}{\ensuremath{^{\prime\prime\prime}}}
\newunicodechar{⅋}{}
\newunicodechar{∅}{\ensuremath{\emptyset}}
\newunicodechar{≺}{\ensuremath{\mathnormal{\prec}}}
\newunicodechar{≼}{\ensuremath{\mathnormal{\preceq}}}
\newunicodechar{∩}{\ensuremath{\mathnormal{\cap}}}
\newunicodechar{⟪}{\ensuremath{\langle\kern-.2em\langle}}
\newunicodechar{⟫}{\ensuremath{\rangle\kern-.2em\rangle}}
\newunicodechar{⊆}{\ensuremath{\mathnormal{\subseteq}}}
\newunicodechar{⊇}{\ensuremath{\mathnormal{\supseteq}}}
\newunicodechar{▻}{\ensuremath{\mathnormal\vartriangleright}}
\newunicodechar{∷}{\ensuremath{::}}
\newunicodechar{►}{\ensuremath{\mathnormal{\blacktriangleright}}}
\newunicodechar{▹}{\ensuremath{\mathnormal{\triangleright}}}
\newunicodechar{□}{\ensuremath{\mathnormal{\square}}}
\newunicodechar{⋯}{\ensuremath{\mathnormal{\cdots}}}
\newunicodechar{▣}{\ensuremath{\mathnormal{\ldots}}}
\newunicodechar{⋮}{\ensuremath{\mathnormal{\quad\quad\vdots}}}
\newunicodechar{∈}{\ensuremath{\mathnormal{\in}}}
\newunicodechar{⊑}{\ensuremath{\mathbin{\sqsubseteq}}}
\newunicodechar{𝓑}{\ensuremath{\mathnormal{\gg\!\!=}}}
\newunicodechar{𝓒}{\ensuremath{\mathbf{\mathscr{C}}}}
\newunicodechar{𝓓}{\ensuremath{\mathbf{\mathscr{D}}}}
\newunicodechar{♯}{\ensuremath{\sharp}}
\newunicodechar{∎}{\ensuremath{\blacksquare}}
\newunicodechar{■}{\ensuremath{\square}}
\newunicodechar{↣}{\ensuremath{\rightarrowtail}}
\newunicodechar{⟶}{\ensuremath{\longrightarrow}}
\newunicodechar{≲}{\ensuremath{\lesssim}}
\newunicodechar{ℓ}{\ensuremath{\ell}}
\newunicodechar{≈}{\ensuremath{\approx}}
\newunicodechar{≪}{\ensuremath{\ll}}
\newunicodechar{≅}{\ensuremath{\cong}}
\newunicodechar{◄}{\ensuremath{\blacktriangleleft}}

%%
%% Greek
%%

\newunicodechar{φ}{\ensuremath{\phi}}
\newunicodechar{Φ}{\ensuremath{\Phi}}
\newunicodechar{ψ}{\ensuremath{\psi}}
\newunicodechar{μ}{\ensuremath{\mu}}
\newunicodechar{α}{\ensuremath{\alpha}}
\newunicodechar{β}{\ensuremath{\beta}}
\newunicodechar{σ}{\ensuremath{\sigma}}
\newunicodechar{ξ}{\ensuremath{\xi}}
\newunicodechar{Ξ}{\ensuremath{\Xi}}
\newunicodechar{λ}{\ensuremath{\lambda}}
\newunicodechar{ε}{\ensuremath{\epsilon}}
\newunicodechar{γ}{\ensuremath{\gamma}}
\newunicodechar{Σ}{\ensuremath{\Sigma}}
\newunicodechar{Δ}{\ensuremath{\Delta}}
\newunicodechar{Π}{\ensuremath{\Pi}}
\newunicodechar{Γ}{\ensuremath{\Gamma}}
\newunicodechar{η}{\ensuremath{\eta}}
\newunicodechar{ζ}{\ensuremath{\zeta}}
\newunicodechar{δ}{\ensuremath{\delta}}
\newunicodechar{θ}{\ensuremath{\theta}}


%%
%% subscript/superscript
%%

\newunicodechar{₀}{\ensuremath{_{0}}}
\newunicodechar{₁}{\ensuremath{_{1}}}
\newunicodechar{₂}{\ensuremath{_{2}}}
\newunicodechar{₃}{\ensuremath{_{3}}}
\newunicodechar{₄}{\ensuremath{_{4}}}
\newunicodechar{₅}{\ensuremath{_{5}}}
\newunicodechar{₆}{\ensuremath{_{6}}}
\newunicodechar{₇}{\ensuremath{_{7}}}
\newunicodechar{₈}{\ensuremath{_{8}}}
\newunicodechar{₉}{\ensuremath{_{9}}}
\newunicodechar{¹}{\ensuremath{^{1}}}
\newunicodechar{⁻}{$^{-}$}
\newunicodechar{ᴬ}{$^{A}$}
\newunicodechar{ᴮ}{$^{B}$}
\newunicodechar{ᴱ}{$^{E}$}
\newunicodechar{ᴴ}{$^{\textsf{H}}$}
\newunicodechar{ˣ}{$^{×}$}
\newunicodechar{ᵈ}{$^{d}$}
\newunicodechar{ᵘ}{$^{u}$}
\newunicodechar{ᶠ}{$^{F}$}
\newunicodechar{ⁱ}{$^{i}$}
\newunicodechar{ᵒ}{$^{o}$}
\newunicodechar{ˢ}{$^{s}$}
\newunicodechar{ˡ}{$^{l}$}
\newunicodechar{ʳ}{$^r$}
\newunicodechar{ᴰ}{$^D$}
\newunicodechar{ᵀ}{$^T$}
\newunicodechar{ᵇ}{$^{b}$}
\newunicodechar{ᵐ}{$_m$}
\newunicodechar{ⁿ}{$_n$}
\newunicodechar{ₚ}{$_p$}
\newunicodechar{ₒ}{$_o$}
\newunicodechar{ₙ}{$_n$}
\newunicodechar{ₘ}{$_m$}
\newunicodechar{ᵛ}{$_v$}
\newunicodechar{ᵍ}{$_f$}
\newunicodechar{ᵢ}{$_i$}
\newunicodechar{ₗ}{$_l$}
\newunicodechar{ᵣ}{$_r$}
\newunicodechar{ₛ}{$_s$}
\newunicodechar{ₖ}{$_{k}$}
\newunicodechar{ₜ}{$_t$}
\newunicodechar{ᶜ}{$^c$}
\newunicodechar{ₐ}{$_a$}
\newunicodechar{∶}{$:$}
\newunicodechar{̅}{$^{\textit{d}}$}

\newunicodechar{𝑡}{\textit{t}}
\newunicodechar{ℎ}{\textit{h}}
\newunicodechar{𝑟}{\textit{r}}
\newunicodechar{𝑜}{\textit{o}}
\newunicodechar{𝑤}{\textit{w}}
\newunicodechar{𝑐}{\textit{c}}
\newunicodechar{𝑎}{\textit{a}}
\newunicodechar{𝑡}{\textit{t}}
\newunicodechar{ℰ}{\ensuremath{\mathcal{E}}}

\newunicodechar{𝓑}{\ensuremath{\mathnormal{\gg\!\!=}}}



%%
% Multi-column code
\NewDocumentCommand{\vcodesep}{}{{\color{lightgray}\vrule}}

% displays put some stuff like code in a paragraph, without interrupting it.
\setlength{\parskip}{0em} 
\setlength\mathindent{0.2cm}
%\setlength\abovedisplayskip{0em}
%\setlength\belowdisplayskip{em}
\usepackage{float}
\newcommand{\displayskip}[1]{%
  {\parskip=0pt\parindent=0pt\par\vskip #1\noindent}}
\NewDocumentCommand{\display}{m}
  {{\ifhmode%
    \parskip=0pt\parindent=0pt\par\vskip 5pt\noindent
    %\displayskip{\abovedisplayskip}%
    %\displayskip{0pt}%
       #1%
    %\displayskip{\belowdisplayskip}%
    \parskip=0pt\parindent=0pt\par\noindent
    \displayskip{0pt}%
   \else
     #1
   \fi}}

\NewDocumentCommand{\twocodes}{O{0.49\textwidth} O{0.49\textwidth} m m}
  {\display{\vskip -10pt
    \begin{minipage}[t]{#1}
    #3
    \end{minipage}
    \hfill\vcodesep
    \begin{minipage}[t]{#2}
    #4
    \end{minipage}
  }}

\begin{document}
\raggedbottom

%% Title information
\title{Hefty Algebras: Modular Elaboration of Higher-Order Effects}
% 
% %% Author with single affiliation.
% \author{Cas van der Rest}				
% \orcid{0000-0002-0059-5353}            
% \affiliation{
%   \institution{Delft University of Technology}          
%   \city{Delft}
%   \country{The Netherlands}                  
% }
% \email{c.r.vanderrest@tudelft.nl}       
% 
% \author{Casper Bach Poulsen}			
% \orcid{0000-0003-0622-7639}            
% \affiliation{
%   \institution{Delft University of Technology}          
%   \city{Delft}
%   \country{The Netherlands}                  
% }
% \email{c.b.poulsen@tudelft.nl}    
%

\begin{authgrp}
\author{Casper Bach Poulsen}
\affiliation{Delft University of Technology\\
  (\email{c.b.poulsen@tudelft.nl})}
\author{Cas van der Rest}
\affiliation{Delft University of Technology\\
  (\email{c.r.vanderrest@tudelft.nl})}
\end{authgrp}

\begin{abstract}
  Algebraic effects and handlers is an increasingly popular approach to
  programming with effects.  An attraction of the approach is its modularity:
  effectful programs are written against an interface of declared operations,
  which allows the implementation of these operations to be defined and refined
  without changing or recompiling programs written against the interface.
  However, higher-order operations (i.e., operations that take computations as
  arguments) break this modularity.  While it is possible to encode higher-order
  operations by elaborating them into more primitive algebraic effects and
  handlers, such elaborations are typically not modular.  In particular,
  operations defined by elaboration are typically not a part of any effect
  interface, so we cannot define and refine their implementation without
  changing or recompiling programs.  To resolve this problem, a recent line of
  research focuses on developing new and improved effect handlers.  In this
  paper we present a (surprisingly) simple alternative solution to the
  modularity problem with higher-order operations: we modularize the previously
  non-modular elaborations commonly used to encode higher-order operations.  Our
  solution is as expressive as the state of the art in effects and handlers.
\end{abstract}

% 
% \keywords{
%   Type Safety,
%   Modularity,
%   Reuse,
%   Definitional Interpreters,
%   Dependently Typed Programming
% } %% \keywords are mandatory in final camera-ready submission
% 

\maketitle

\input{tex/sections/1-introduction.tex}
\input{tex/sections/2-algebraic-effects.tex}
\input{tex/sections/3-hefty-algebras.tex}
\begin{code}[hide]%
\>[0]\AgdaSymbol{\{-\#}\AgdaSpace{}%
\AgdaKeyword{OPTIONS}\AgdaSpace{}%
\AgdaPragma{--overlapping-instances}\AgdaSpace{}%
\AgdaPragma{--instance-search-depth=10}\AgdaSpace{}%
\AgdaSymbol{\#-\}}\<%
\\
\>[0]\AgdaKeyword{module}\AgdaSpace{}%
\AgdaModule{tex.sections.5-examples}\AgdaSpace{}%
\AgdaKeyword{where}\<%
\\
%
\\[\AgdaEmptyExtraSkip]%
\>[0]\AgdaKeyword{open}\AgdaSpace{}%
\AgdaKeyword{import}\AgdaSpace{}%
\AgdaModule{tex.sections.2-algebraic-effects}\<%
\\
\>[0]\AgdaKeyword{open}\AgdaSpace{}%
\AgdaKeyword{import}\AgdaSpace{}%
\AgdaModule{tex.sections.3-hefty-algebras}\<%
\\
%
\\[\AgdaEmptyExtraSkip]%
\>[0]\AgdaKeyword{open}\AgdaSpace{}%
\AgdaKeyword{import}\AgdaSpace{}%
\AgdaModule{Function}\AgdaSpace{}%
\AgdaKeyword{hiding}\AgdaSpace{}%
\AgdaSymbol{(}\AgdaPrimitive{force}\AgdaSymbol{;}\AgdaSpace{}%
\AgdaOperator{\AgdaFunction{\AgdaUnderscore{}↣\AgdaUnderscore{}}}\AgdaSymbol{;}\AgdaSpace{}%
\AgdaOperator{\AgdaFunction{\AgdaUnderscore{}⟶\AgdaUnderscore{}}}\AgdaSymbol{)}\<%
\\
\>[0]\AgdaKeyword{open}\AgdaSpace{}%
\AgdaKeyword{import}\AgdaSpace{}%
\AgdaModule{Data.Empty}\<%
\\
\>[0]\AgdaKeyword{open}\AgdaSpace{}%
\AgdaKeyword{import}\AgdaSpace{}%
\AgdaModule{Data.Unit}\<%
\\
\>[0]\AgdaKeyword{open}\AgdaSpace{}%
\AgdaKeyword{import}\AgdaSpace{}%
\AgdaModule{Data.Bool}\AgdaSpace{}%
\AgdaKeyword{hiding}\AgdaSpace{}%
\AgdaSymbol{(}\AgdaFunction{T}\AgdaSymbol{)}\<%
\\
\>[0]\AgdaKeyword{open}\AgdaSpace{}%
\AgdaKeyword{import}\AgdaSpace{}%
\AgdaModule{Data.Sum}\<%
\\
\>[0]\AgdaKeyword{open}\AgdaSpace{}%
\AgdaKeyword{import}\AgdaSpace{}%
\AgdaModule{Data.Product}\<%
\\
\>[0]\AgdaKeyword{open}\AgdaSpace{}%
\AgdaKeyword{import}\AgdaSpace{}%
\AgdaModule{Data.Maybe}\AgdaSpace{}%
\AgdaKeyword{using}\AgdaSpace{}%
\AgdaSymbol{(}\AgdaDatatype{Maybe}\AgdaSymbol{;}\AgdaSpace{}%
\AgdaInductiveConstructor{just}\AgdaSymbol{;}\AgdaSpace{}%
\AgdaInductiveConstructor{nothing}\AgdaSymbol{;}\AgdaSpace{}%
\AgdaFunction{maybe}\AgdaSymbol{)}\<%
\\
\>[0]\AgdaKeyword{open}\AgdaSpace{}%
\AgdaKeyword{import}\AgdaSpace{}%
\AgdaModule{Data.List}\AgdaSpace{}%
\AgdaKeyword{using}\AgdaSpace{}%
\AgdaSymbol{(}\AgdaDatatype{List}\AgdaSymbol{;}\AgdaSpace{}%
\AgdaInductiveConstructor{[]}\AgdaSymbol{;}\AgdaSpace{}%
\AgdaOperator{\AgdaInductiveConstructor{\AgdaUnderscore{}∷\AgdaUnderscore{}}}\AgdaSymbol{;}\AgdaSpace{}%
\AgdaOperator{\AgdaFunction{\AgdaUnderscore{}++\AgdaUnderscore{}}}\AgdaSymbol{;}\AgdaSpace{}%
\AgdaFunction{head}\AgdaSymbol{)}\<%
\\
\>[0]\AgdaKeyword{open}\AgdaSpace{}%
\AgdaKeyword{import}\AgdaSpace{}%
\AgdaModule{Data.List.Membership.Propositional}\<%
\\
\>[0]\AgdaKeyword{open}\AgdaSpace{}%
\AgdaKeyword{import}\AgdaSpace{}%
\AgdaModule{Data.String}\AgdaSpace{}%
\AgdaKeyword{using}\AgdaSpace{}%
\AgdaSymbol{(}\AgdaPostulate{String}\AgdaSymbol{)}\<%
\\
\>[0]\AgdaKeyword{open}\AgdaSpace{}%
\AgdaKeyword{import}\AgdaSpace{}%
\AgdaModule{Relation.Binary.PropositionalEquality}\AgdaSpace{}%
\AgdaKeyword{hiding}\AgdaSpace{}%
\AgdaSymbol{(}\AgdaOperator{\AgdaInductiveConstructor{[\AgdaUnderscore{}]}}\AgdaSymbol{)}\<%
\\
\>[0]\<%
\end{code}

\section{Examples}
\label{sec:examples}

As discussed in \cref{sec:higher-order-effects}, there is a wide range of examples of higher-order effects that cannot be defined as algebraic operations directly, and are typically defined as non-modular elaborations instead.
In this section we give examples of such effects and show to define them modularly using hefty algebras.
The artifact~\citep{artifact} contains the full examples.


\subsection{$\lambda$ as a Higher-Order Operation}
\label{sec:higher-order-lambda}

\begin{code}[hide]%
\>[0]\AgdaKeyword{module}\AgdaSpace{}%
\AgdaModule{AbstractionModule}\AgdaSpace{}%
\AgdaKeyword{where}\<%
\\
\>[0][@{}l@{\AgdaIndent{0}}]%
\>[2]\AgdaKeyword{open}\AgdaSpace{}%
\AgdaModule{FreeModule}\AgdaSpace{}%
\AgdaKeyword{hiding}\AgdaSpace{}%
\AgdaSymbol{(}\AgdaOperator{\AgdaFunction{\AgdaUnderscore{}𝓑\AgdaUnderscore{}}}\AgdaSymbol{;}\AgdaSpace{}%
\AgdaOperator{\AgdaFunction{\AgdaUnderscore{}>>\AgdaUnderscore{}}}\AgdaSymbol{)}\<%
\\
%
\>[2]\AgdaKeyword{open}\AgdaSpace{}%
\AgdaModule{HeftyModule}\AgdaSpace{}%
\AgdaKeyword{hiding}\AgdaSpace{}%
\AgdaSymbol{(}\AgdaOperator{\AgdaFunction{\AgdaUnderscore{}𝓑\AgdaUnderscore{}}}\AgdaSymbol{;}\AgdaSpace{}%
\AgdaOperator{\AgdaFunction{\AgdaUnderscore{}>>\AgdaUnderscore{}}}\AgdaSymbol{)}\<%
\\
%
\>[2]\AgdaKeyword{open}\AgdaSpace{}%
\AgdaModule{ElabModule}\<%
\\
%
\>[2]\AgdaKeyword{open}\AgdaSpace{}%
\AgdaModule{Algᴴ}\<%
\\
%
\>[2]\AgdaKeyword{open}\AgdaSpace{}%
\AgdaOperator{\AgdaModule{⟨\AgdaUnderscore{}!\AgdaUnderscore{}⇒\AgdaUnderscore{}⇒\AgdaUnderscore{}!\AgdaUnderscore{}⟩}}\<%
\\
%
\>[2]\AgdaKeyword{open}\AgdaSpace{}%
\AgdaModule{Effect}\<%
\\
%
\>[2]\AgdaKeyword{open}\AgdaSpace{}%
\AgdaModule{Effectᴴ}\<%
\\
%
\>[2]\AgdaKeyword{open}\AgdaSpace{}%
\AgdaModule{Univ}\AgdaSpace{}%
\AgdaSymbol{⦃}\AgdaSpace{}%
\AgdaSymbol{...}\AgdaSpace{}%
\AgdaSymbol{⦄}\<%
\end{code}

As recently observed by \citet{BergSPW21}, the (higher-order) operations for $\lambda$ abstraction and application are neither algebraic nor scoped effects.
We demonstrate how hefty algebras allow us to modularly define and elaborate an interface of higher-order operations for $\lambda$ abstraction and application, inspired by Levy's call-by-push-value~\citep{Levy06}.
The interface we will consider is parametric in a universe of types given by the following record:

\begin{code}%
%
\>[2]\AgdaKeyword{record}\AgdaSpace{}%
\AgdaRecord{LamUniv}\AgdaSpace{}%
\AgdaSymbol{:}\AgdaSpace{}%
\AgdaPrimitive{Set₁}\AgdaSpace{}%
\AgdaKeyword{where}\<%
\\
\>[2][@{}l@{\AgdaIndent{0}}]%
\>[4]\AgdaKeyword{field}%
\>[11]\AgdaSymbol{⦃}\AgdaSpace{}%
\AgdaField{u}\AgdaSpace{}%
\AgdaSymbol{⦄}%
\>[18]\AgdaSymbol{:}\AgdaSpace{}%
\AgdaRecord{Univ}\<%
\\
%
\>[11]\AgdaOperator{\AgdaField{\AgdaUnderscore{}↣\AgdaUnderscore{}}}%
\>[18]\AgdaSymbol{:}\AgdaSpace{}%
\AgdaField{Type}\AgdaSpace{}%
\AgdaSymbol{→}\AgdaSpace{}%
\AgdaField{Type}\AgdaSpace{}%
\AgdaSymbol{→}\AgdaSpace{}%
\AgdaField{Type}\<%
\\
%
\>[11]\AgdaField{c}%
\>[18]\AgdaSymbol{:}\AgdaSpace{}%
\AgdaField{Type}\AgdaSpace{}%
\AgdaSymbol{→}\AgdaSpace{}%
\AgdaField{Type}\<%
\end{code}
%
The \aF{\_↣\_} field represents a function type, whereas \aF{c} is the type of \emph{thunk values}.
Distinguishing thunks in this way allows us to assign either a call-by-value or call-by-name semantics to the interface for $\lambda$ abstraction, given by the higher-order effect signature in \cref{fig:ho-lam-sig}, and summarized by the following smart constructors:
%
\begin{code}[hide]%
%
\>[2]\AgdaKeyword{open}\AgdaSpace{}%
\AgdaModule{LamUniv}\AgdaSpace{}%
\AgdaSymbol{⦃}\AgdaSpace{}%
\AgdaSymbol{...}\AgdaSpace{}%
\AgdaSymbol{⦄}\<%
\\
%
\\[\AgdaEmptyExtraSkip]%
%
\>[2]\AgdaKeyword{module}\AgdaSpace{}%
\AgdaModule{LamModule}\AgdaSpace{}%
\AgdaKeyword{where}\<%
\\
\>[2][@{}l@{\AgdaIndent{0}}]%
\>[4]\AgdaKeyword{open}\AgdaSpace{}%
\AgdaKeyword{import}\AgdaSpace{}%
\AgdaModule{Data.List.Relation.Unary.All}\<%
\\
%
\>[4]\AgdaKeyword{open}\AgdaSpace{}%
\AgdaModule{Inverse}\AgdaSpace{}%
\AgdaSymbol{⦃}\AgdaSpace{}%
\AgdaSymbol{...}\AgdaSpace{}%
\AgdaSymbol{⦄}\<%
\\
%
\\[\AgdaEmptyExtraSkip]%
%
\>[4]\AgdaComment{--\ Read\ :\ Set\ →\ Effect}\<%
\\
%
\>[4]\AgdaComment{--\ Op\ \ (Read\ A)\ \ \ \ \ \ =\ ReadOp}\<%
\\
%
\>[4]\AgdaComment{--\ Ret\ (Read\ A)\ ask\ \ =\ A}\<%
\\
%
\\[\AgdaEmptyExtraSkip]%
%
\>[4]\AgdaComment{--\ ‵ask\ :\ ⦃\ Δ\ ∼\ Read\ A\ ▸\ Δ′\ ⦄\ →\ Free\ Δ\ A}\<%
\\
%
\>[4]\AgdaComment{--\ ‵ask\ ⦃\ w\ ⦄\ =\ impure\ (inj▸ₗ\ ask)\ (pure\ ∘\ proj-ret▸ₗ\ ⦃\ w\ ⦄)}\<%
\\
%
\\[\AgdaEmptyExtraSkip]%
%
\\[\AgdaEmptyExtraSkip]%
%
\>[4]\AgdaComment{--\ hRead\ :\ ParameterizedHandler\ (Read\ A)\ A\ id}\<%
\\
%
\>[4]\AgdaComment{--\ ret\ hRead\ x\ \AgdaUnderscore{}\ \ \ \ \ \ =\ x}\<%
\\
%
\>[4]\AgdaComment{--\ hdl\ hRead\ ask\ k\ r\ \ =\ k\ r\ r}\<%
\\
%
\\[\AgdaEmptyExtraSkip]%
%
\>[4]\AgdaKeyword{data}\AgdaSpace{}%
\AgdaDatatype{LamOp}\AgdaSpace{}%
\AgdaSymbol{⦃}\AgdaSpace{}%
\AgdaBound{l}\AgdaSpace{}%
\AgdaSymbol{:}\AgdaSpace{}%
\AgdaRecord{LamUniv}\AgdaSpace{}%
\AgdaSymbol{⦄}\AgdaSpace{}%
\AgdaSymbol{:}\AgdaSpace{}%
\AgdaPrimitive{Set}\AgdaSpace{}%
\AgdaKeyword{where}\<%
\\
\>[4][@{}l@{\AgdaIndent{0}}]%
\>[6]\AgdaInductiveConstructor{lam}\AgdaSpace{}%
\AgdaSymbol{:}\AgdaSpace{}%
\AgdaSymbol{\{}\AgdaBound{t₁}\AgdaSpace{}%
\AgdaBound{t₂}\AgdaSpace{}%
\AgdaSymbol{:}\AgdaSpace{}%
\AgdaField{Type}\AgdaSymbol{\}}%
\>[47]\AgdaSymbol{→}\AgdaSpace{}%
\AgdaDatatype{LamOp}\<%
\\
%
\>[6]\AgdaInductiveConstructor{var}\AgdaSpace{}%
\AgdaSymbol{:}\AgdaSpace{}%
\AgdaSymbol{\{}\AgdaBound{t}\AgdaSpace{}%
\AgdaSymbol{:}\AgdaSpace{}%
\AgdaField{Type}\AgdaSymbol{\}}%
\>[28]\AgdaSymbol{→}\AgdaSpace{}%
\AgdaOperator{\AgdaField{⟦}}\AgdaSpace{}%
\AgdaField{c}\AgdaSpace{}%
\AgdaBound{t}\AgdaSpace{}%
\AgdaOperator{\AgdaField{⟧ᵀ}}%
\>[48]\AgdaSymbol{→}\AgdaSpace{}%
\AgdaDatatype{LamOp}\<%
\\
%
\>[6]\AgdaInductiveConstructor{app}\AgdaSpace{}%
\AgdaSymbol{:}\AgdaSpace{}%
\AgdaSymbol{\{}\AgdaBound{t₁}\AgdaSpace{}%
\AgdaBound{t₂}\AgdaSpace{}%
\AgdaSymbol{:}\AgdaSpace{}%
\AgdaField{Type}\AgdaSymbol{\}}%
\>[28]\AgdaSymbol{→}\AgdaSpace{}%
\AgdaOperator{\AgdaField{⟦}}\AgdaSpace{}%
\AgdaSymbol{(}\AgdaField{c}\AgdaSpace{}%
\AgdaBound{t₁}\AgdaSymbol{)}\AgdaSpace{}%
\AgdaOperator{\AgdaField{↣}}\AgdaSpace{}%
\AgdaBound{t₂}\AgdaSpace{}%
\AgdaOperator{\AgdaField{⟧ᵀ}}%
\>[48]\AgdaSymbol{→}\AgdaSpace{}%
\AgdaDatatype{LamOp}\<%
\\
%
\\[\AgdaEmptyExtraSkip]%
%
\>[4]\AgdaFunction{Lam}\AgdaSpace{}%
\AgdaSymbol{:}\AgdaSpace{}%
\AgdaSymbol{⦃}\AgdaSpace{}%
\AgdaBound{l}\AgdaSpace{}%
\AgdaSymbol{:}\AgdaSpace{}%
\AgdaRecord{LamUniv}\AgdaSpace{}%
\AgdaSymbol{⦄}\AgdaSpace{}%
\AgdaSymbol{→}\AgdaSpace{}%
\AgdaRecord{Effectᴴ}\<%
\\
%
\>[4]\AgdaField{Opᴴ}%
\>[11]\AgdaFunction{Lam}%
\>[37]\AgdaSymbol{=}\AgdaSpace{}%
\AgdaDatatype{LamOp}\<%
\\
%
\>[4]\AgdaField{Retᴴ}%
\>[11]\AgdaFunction{Lam}%
\>[16]\AgdaSymbol{(}\AgdaInductiveConstructor{lam}\AgdaSpace{}%
\AgdaSymbol{\{}\AgdaBound{t₁}\AgdaSymbol{\}}\AgdaSpace{}%
\AgdaSymbol{\{}\AgdaBound{t₂}\AgdaSymbol{\})}%
\>[37]\AgdaSymbol{=}\AgdaSpace{}%
\AgdaOperator{\AgdaField{⟦}}\AgdaSpace{}%
\AgdaSymbol{(}\AgdaField{c}\AgdaSpace{}%
\AgdaBound{t₁}\AgdaSymbol{)}\AgdaSpace{}%
\AgdaOperator{\AgdaField{↣}}\AgdaSpace{}%
\AgdaBound{t₂}\AgdaSpace{}%
\AgdaOperator{\AgdaField{⟧ᵀ}}\<%
\\
%
\>[4]\AgdaField{Retᴴ}%
\>[11]\AgdaFunction{Lam}%
\>[16]\AgdaSymbol{(}\AgdaInductiveConstructor{var}\AgdaSpace{}%
\AgdaSymbol{\{}\AgdaBound{t}\AgdaSymbol{\}}\AgdaSpace{}%
\AgdaSymbol{\AgdaUnderscore{})}%
\>[37]\AgdaSymbol{=}\AgdaSpace{}%
\AgdaOperator{\AgdaField{⟦}}\AgdaSpace{}%
\AgdaBound{t}\AgdaSpace{}%
\AgdaOperator{\AgdaField{⟧ᵀ}}\<%
\\
%
\>[4]\AgdaField{Retᴴ}%
\>[11]\AgdaFunction{Lam}%
\>[16]\AgdaSymbol{(}\AgdaInductiveConstructor{app}\AgdaSpace{}%
\AgdaSymbol{\{}\AgdaBound{t₁}\AgdaSymbol{\}}\AgdaSpace{}%
\AgdaSymbol{\{}\AgdaBound{t₂}\AgdaSymbol{\}}\AgdaSpace{}%
\AgdaSymbol{\AgdaUnderscore{})}%
\>[37]\AgdaSymbol{=}\AgdaSpace{}%
\AgdaOperator{\AgdaField{⟦}}\AgdaSpace{}%
\AgdaBound{t₂}\AgdaSpace{}%
\AgdaOperator{\AgdaField{⟧ᵀ}}\<%
\\
%
\>[4]\AgdaField{Fork}%
\>[11]\AgdaFunction{Lam}%
\>[16]\AgdaSymbol{(}\AgdaInductiveConstructor{lam}\AgdaSpace{}%
\AgdaSymbol{\{}\AgdaBound{t₁}\AgdaSymbol{\}}\AgdaSpace{}%
\AgdaSymbol{\{}\AgdaBound{t₂}\AgdaSymbol{\})}%
\>[37]\AgdaSymbol{=}\AgdaSpace{}%
\AgdaOperator{\AgdaField{⟦}}\AgdaSpace{}%
\AgdaField{c}\AgdaSpace{}%
\AgdaBound{t₁}\AgdaSpace{}%
\AgdaOperator{\AgdaField{⟧ᵀ}}\<%
\\
%
\>[4]\AgdaField{Fork}%
\>[11]\AgdaFunction{Lam}%
\>[16]\AgdaSymbol{(}\AgdaInductiveConstructor{var}\AgdaSpace{}%
\AgdaSymbol{\AgdaUnderscore{})}%
\>[37]\AgdaSymbol{=}\AgdaSpace{}%
\AgdaFunction{⊥}\<%
\\
%
\>[4]\AgdaField{Fork}%
\>[11]\AgdaFunction{Lam}%
\>[16]\AgdaSymbol{(}\AgdaInductiveConstructor{app}\AgdaSpace{}%
\AgdaSymbol{\{}\AgdaBound{t₁}\AgdaSymbol{\}}\AgdaSpace{}%
\AgdaSymbol{\{}\AgdaBound{t₂}\AgdaSymbol{\}}\AgdaSpace{}%
\AgdaSymbol{\AgdaUnderscore{})}%
\>[37]\AgdaSymbol{=}\AgdaSpace{}%
\AgdaRecord{⊤}\<%
\\
%
\>[4]\AgdaField{Ty}%
\>[11]\AgdaFunction{Lam}%
\>[16]\AgdaSymbol{\{}\AgdaInductiveConstructor{lam}\AgdaSpace{}%
\AgdaSymbol{\{}\AgdaBound{t₁}\AgdaSymbol{\}}\AgdaSpace{}%
\AgdaSymbol{\{}\AgdaBound{t₂}\AgdaSymbol{\}\}}\AgdaSpace{}%
\AgdaSymbol{\AgdaUnderscore{}}%
\>[37]\AgdaSymbol{=}\AgdaSpace{}%
\AgdaOperator{\AgdaField{⟦}}\AgdaSpace{}%
\AgdaBound{t₂}\AgdaSpace{}%
\AgdaOperator{\AgdaField{⟧ᵀ}}\<%
\\
%
\>[4]\AgdaField{Ty}%
\>[11]\AgdaFunction{Lam}%
\>[16]\AgdaSymbol{\{}\AgdaInductiveConstructor{var}\AgdaSpace{}%
\AgdaSymbol{\AgdaUnderscore{}\}}\AgdaSpace{}%
\AgdaSymbol{()}\<%
\\
%
\>[4]\AgdaField{Ty}%
\>[11]\AgdaFunction{Lam}%
\>[16]\AgdaSymbol{\{}\AgdaInductiveConstructor{app}\AgdaSpace{}%
\AgdaSymbol{\{}\AgdaBound{t₁}\AgdaSymbol{\}}\AgdaSpace{}%
\AgdaSymbol{\{}\AgdaBound{t₂}\AgdaSymbol{\}}\AgdaSpace{}%
\AgdaSymbol{\AgdaUnderscore{}\}}\AgdaSpace{}%
\AgdaSymbol{\AgdaUnderscore{}}%
\>[37]\AgdaSymbol{=}\AgdaSpace{}%
\AgdaOperator{\AgdaField{⟦}}\AgdaSpace{}%
\AgdaBound{t₁}\AgdaSpace{}%
\AgdaOperator{\AgdaField{⟧ᵀ}}\<%
\\
%
\\[\AgdaEmptyExtraSkip]%
%
\>[4]\AgdaKeyword{module}\AgdaSpace{}%
\AgdaModule{\AgdaUnderscore{}}\AgdaSpace{}%
\AgdaSymbol{⦃}\AgdaSpace{}%
\AgdaBound{l}\AgdaSpace{}%
\AgdaSymbol{:}\AgdaSpace{}%
\AgdaRecord{LamUniv}\AgdaSpace{}%
\AgdaSymbol{⦄}\AgdaSpace{}%
\AgdaSymbol{⦃}\AgdaSpace{}%
\AgdaBound{w}\AgdaSpace{}%
\AgdaSymbol{:}\AgdaSpace{}%
\AgdaFunction{Lam}\AgdaSpace{}%
\AgdaOperator{\AgdaFunction{≲ᴴ}}\AgdaSpace{}%
\AgdaGeneralizable{H}\AgdaSpace{}%
\AgdaSymbol{⦄}\AgdaSpace{}%
\AgdaKeyword{where}\<%
\end{code}
%
\begin{code}%
\>[4][@{}l@{\AgdaIndent{1}}]%
\>[6]\AgdaFunction{‵lam}%
\>[12]\AgdaSymbol{:}%
\>[15]\AgdaSymbol{\{}\AgdaBound{t₁}\AgdaSpace{}%
\AgdaBound{t₂}\AgdaSpace{}%
\AgdaSymbol{:}\AgdaSpace{}%
\AgdaField{Type}\AgdaSymbol{\}}%
\>[31]\AgdaSymbol{→}\AgdaSpace{}%
\AgdaSymbol{(}\AgdaOperator{\AgdaField{⟦}}\AgdaSpace{}%
\AgdaField{c}\AgdaSpace{}%
\AgdaBound{t₁}\AgdaSpace{}%
\AgdaOperator{\AgdaField{⟧ᵀ}}\AgdaSpace{}%
\AgdaSymbol{→}\AgdaSpace{}%
\AgdaDatatype{Hefty}\AgdaSpace{}%
\AgdaBound{H}\AgdaSpace{}%
\AgdaOperator{\AgdaField{⟦}}\AgdaSpace{}%
\AgdaBound{t₂}\AgdaSpace{}%
\AgdaOperator{\AgdaField{⟧ᵀ}}\AgdaSymbol{)}%
\>[69]\AgdaSymbol{→}\AgdaSpace{}%
\AgdaDatatype{Hefty}\AgdaSpace{}%
\AgdaBound{H}\AgdaSpace{}%
\AgdaOperator{\AgdaField{⟦}}\AgdaSpace{}%
\AgdaSymbol{(}\AgdaField{c}\AgdaSpace{}%
\AgdaBound{t₁}\AgdaSymbol{)}\AgdaSpace{}%
\AgdaOperator{\AgdaField{↣}}\AgdaSpace{}%
\AgdaBound{t₂}\AgdaSpace{}%
\AgdaOperator{\AgdaField{⟧ᵀ}}\<%
\\
%
\>[6]\AgdaFunction{‵var}%
\>[12]\AgdaSymbol{:}%
\>[15]\AgdaSymbol{\{}\AgdaBound{t}\AgdaSpace{}%
\AgdaSymbol{:}\AgdaSpace{}%
\AgdaField{Type}\AgdaSymbol{\}}%
\>[31]\AgdaSymbol{→}\AgdaSpace{}%
\AgdaOperator{\AgdaField{⟦}}\AgdaSpace{}%
\AgdaField{c}\AgdaSpace{}%
\AgdaBound{t}\AgdaSpace{}%
\AgdaOperator{\AgdaField{⟧ᵀ}}%
\>[69]\AgdaSymbol{→}\AgdaSpace{}%
\AgdaDatatype{Hefty}\AgdaSpace{}%
\AgdaBound{H}\AgdaSpace{}%
\AgdaOperator{\AgdaField{⟦}}\AgdaSpace{}%
\AgdaBound{t}\AgdaSpace{}%
\AgdaOperator{\AgdaField{⟧ᵀ}}\<%
\\
%
\>[6]\AgdaFunction{‵app}%
\>[12]\AgdaSymbol{:}%
\>[15]\AgdaSymbol{\{}\AgdaBound{t₁}\AgdaSpace{}%
\AgdaBound{t₂}\AgdaSpace{}%
\AgdaSymbol{:}\AgdaSpace{}%
\AgdaField{Type}\AgdaSymbol{\}}%
\>[31]\AgdaSymbol{→}\AgdaSpace{}%
\AgdaOperator{\AgdaField{⟦}}\AgdaSpace{}%
\AgdaSymbol{(}\AgdaField{c}\AgdaSpace{}%
\AgdaBound{t₁}\AgdaSymbol{)}\AgdaSpace{}%
\AgdaOperator{\AgdaField{↣}}\AgdaSpace{}%
\AgdaBound{t₂}\AgdaSpace{}%
\AgdaOperator{\AgdaField{⟧ᵀ}}\AgdaSpace{}%
\AgdaSymbol{→}\AgdaSpace{}%
\AgdaDatatype{Hefty}\AgdaSpace{}%
\AgdaBound{H}\AgdaSpace{}%
\AgdaOperator{\AgdaField{⟦}}\AgdaSpace{}%
\AgdaBound{t₁}\AgdaSpace{}%
\AgdaOperator{\AgdaField{⟧ᵀ}}%
\>[69]\AgdaSymbol{→}\AgdaSpace{}%
\AgdaDatatype{Hefty}\AgdaSpace{}%
\AgdaBound{H}\AgdaSpace{}%
\AgdaOperator{\AgdaField{⟦}}\AgdaSpace{}%
\AgdaBound{t₂}\AgdaSpace{}%
\AgdaOperator{\AgdaField{⟧ᵀ}}\<%
\end{code}
\begin{code}[hide]%
%
\>[6]\AgdaFunction{‵lam}\AgdaSpace{}%
\AgdaSymbol{\{}\AgdaBound{t₁}\AgdaSymbol{\}}\AgdaSpace{}%
\AgdaSymbol{\{}\AgdaBound{t₂}\AgdaSymbol{\}}\AgdaSpace{}%
\AgdaBound{b}\AgdaSpace{}%
\AgdaSymbol{=}\AgdaSpace{}%
\AgdaInductiveConstructor{impure}\AgdaSpace{}%
\AgdaSymbol{(}\AgdaFunction{injᴴ}\AgdaSpace{}%
\AgdaSymbol{\{}\AgdaArgument{M}\AgdaSpace{}%
\AgdaSymbol{=}\AgdaSpace{}%
\AgdaDatatype{Hefty}\AgdaSpace{}%
\AgdaSymbol{\AgdaUnderscore{}\}}\AgdaSpace{}%
\AgdaSymbol{(}\AgdaInductiveConstructor{lam}\AgdaSpace{}%
\AgdaSymbol{\{}\AgdaBound{t₁}\AgdaSymbol{\}}\AgdaSpace{}%
\AgdaSymbol{\{}\AgdaBound{t₂}\AgdaSymbol{\}}\AgdaSpace{}%
\AgdaOperator{\AgdaInductiveConstructor{,}}\AgdaSpace{}%
\AgdaInductiveConstructor{pure}\AgdaSpace{}%
\AgdaOperator{\AgdaInductiveConstructor{,}}\AgdaSpace{}%
\AgdaBound{b}\AgdaSymbol{))}\<%
\\
%
\>[6]\AgdaFunction{‵var}\AgdaSpace{}%
\AgdaBound{x}\AgdaSpace{}%
\AgdaSymbol{=}\AgdaSpace{}%
\AgdaInductiveConstructor{impure}\AgdaSpace{}%
\AgdaSymbol{(}\AgdaFunction{injᴴ}\AgdaSpace{}%
\AgdaSymbol{\{}\AgdaArgument{M}\AgdaSpace{}%
\AgdaSymbol{=}\AgdaSpace{}%
\AgdaDatatype{Hefty}\AgdaSpace{}%
\AgdaSymbol{\AgdaUnderscore{}\}}\AgdaSpace{}%
\AgdaSymbol{(}\AgdaInductiveConstructor{var}\AgdaSpace{}%
\AgdaBound{x}\AgdaSpace{}%
\AgdaOperator{\AgdaInductiveConstructor{,}}\AgdaSpace{}%
\AgdaInductiveConstructor{pure}\AgdaSpace{}%
\AgdaOperator{\AgdaInductiveConstructor{,}}\AgdaSpace{}%
\AgdaSymbol{λ}\AgdaSpace{}%
\AgdaSymbol{()))}\<%
\\
%
\>[6]\AgdaFunction{‵app}\AgdaSpace{}%
\AgdaBound{f}\AgdaSpace{}%
\AgdaBound{m}\AgdaSpace{}%
\AgdaSymbol{=}\AgdaSpace{}%
\AgdaInductiveConstructor{impure}\AgdaSpace{}%
\AgdaSymbol{(}\AgdaFunction{injᴴ}\AgdaSpace{}%
\AgdaSymbol{\{}\AgdaArgument{M}\AgdaSpace{}%
\AgdaSymbol{=}\AgdaSpace{}%
\AgdaDatatype{Hefty}\AgdaSpace{}%
\AgdaSymbol{\AgdaUnderscore{}\}}\AgdaSpace{}%
\AgdaSymbol{(}\AgdaInductiveConstructor{app}\AgdaSpace{}%
\AgdaBound{f}\AgdaSpace{}%
\AgdaOperator{\AgdaInductiveConstructor{,}}\AgdaSpace{}%
\AgdaInductiveConstructor{pure}\AgdaSpace{}%
\AgdaOperator{\AgdaInductiveConstructor{,}}\AgdaSpace{}%
\AgdaSymbol{λ}\AgdaSpace{}%
\AgdaBound{\AgdaUnderscore{}}\AgdaSpace{}%
\AgdaSymbol{→}\AgdaSpace{}%
\AgdaBound{m}\AgdaSymbol{))}\<%
\end{code}
%
Here \af{‵lam} is a higher-order operation with a function typed computation parameter and whose return type is a function value (\aF{⟦~c}~\ab{t₁}~\aF{↣}~\ab{t₂}~\aF{⟧ᵀ}).
The \af{‵var} operation accepts a thunk value as argument and yields a value of a matching type.
The \af{‵app} operation is also a higher-order operation: its first parameter is a function value type, whereas its second parameter is a computation parameter whose return type matches that of the function value parameter type.

\begin{figure}[t]
\begin{code}%
%
\>[4]\AgdaKeyword{data}\AgdaSpace{}%
\AgdaDatatype{LamOp⅋}\AgdaSpace{}%
\AgdaSymbol{⦃}\AgdaSpace{}%
\AgdaBound{l}\AgdaSpace{}%
\AgdaSymbol{:}\AgdaSpace{}%
\AgdaRecord{LamUniv}\AgdaSpace{}%
\AgdaSymbol{⦄}\AgdaSpace{}%
\AgdaSymbol{:}\AgdaSpace{}%
\AgdaPrimitive{Set}\AgdaSpace{}%
\AgdaKeyword{where}\<%
\\
\>[4][@{}l@{\AgdaIndent{0}}]%
\>[6]\AgdaInductiveConstructor{lam}%
\>[11]\AgdaSymbol{:}\AgdaSpace{}%
\AgdaSymbol{\{}\AgdaBound{t₁}\AgdaSpace{}%
\AgdaBound{t₂}\AgdaSpace{}%
\AgdaSymbol{:}\AgdaSpace{}%
\AgdaField{Type}\AgdaSymbol{\}}%
\>[48]\AgdaSymbol{→}\AgdaSpace{}%
\AgdaDatatype{LamOp⅋}\<%
\\
%
\>[6]\AgdaInductiveConstructor{var}%
\>[11]\AgdaSymbol{:}\AgdaSpace{}%
\AgdaSymbol{\{}\AgdaBound{t}\AgdaSpace{}%
\AgdaSymbol{:}\AgdaSpace{}%
\AgdaField{Type}\AgdaSymbol{\}}%
\>[29]\AgdaSymbol{→}\AgdaSpace{}%
\AgdaOperator{\AgdaField{⟦}}\AgdaSpace{}%
\AgdaField{c}\AgdaSpace{}%
\AgdaBound{t}\AgdaSpace{}%
\AgdaOperator{\AgdaField{⟧ᵀ}}%
\>[49]\AgdaSymbol{→}\AgdaSpace{}%
\AgdaDatatype{LamOp⅋}\<%
\\
%
\>[6]\AgdaInductiveConstructor{app}%
\>[11]\AgdaSymbol{:}\AgdaSpace{}%
\AgdaSymbol{\{}\AgdaBound{t₁}\AgdaSpace{}%
\AgdaBound{t₂}\AgdaSpace{}%
\AgdaSymbol{:}\AgdaSpace{}%
\AgdaField{Type}\AgdaSymbol{\}}%
\>[29]\AgdaSymbol{→}\AgdaSpace{}%
\AgdaOperator{\AgdaField{⟦}}\AgdaSpace{}%
\AgdaSymbol{(}\AgdaField{c}\AgdaSpace{}%
\AgdaBound{t₁}\AgdaSymbol{)}\AgdaSpace{}%
\AgdaOperator{\AgdaField{↣}}\AgdaSpace{}%
\AgdaBound{t₂}\AgdaSpace{}%
\AgdaOperator{\AgdaField{⟧ᵀ}}%
\>[49]\AgdaSymbol{→}\AgdaSpace{}%
\AgdaDatatype{LamOp⅋}\<%
\\
%
\\[\AgdaEmptyExtraSkip]%
%
\>[4]\AgdaFunction{Lam⅋}\AgdaSpace{}%
\AgdaSymbol{:}\AgdaSpace{}%
\AgdaSymbol{⦃}\AgdaSpace{}%
\AgdaBound{l}\AgdaSpace{}%
\AgdaSymbol{:}\AgdaSpace{}%
\AgdaRecord{LamUniv}\AgdaSpace{}%
\AgdaSymbol{⦄}\AgdaSpace{}%
\AgdaSymbol{→}\AgdaSpace{}%
\AgdaRecord{Effectᴴ}\<%
\\
%
\>[4]\AgdaField{Opᴴ}%
\>[11]\AgdaFunction{Lam⅋}%
\>[38]\AgdaSymbol{=}\AgdaSpace{}%
\AgdaDatatype{LamOp⅋}\<%
\\
%
\>[4]\AgdaField{Retᴴ}%
\>[11]\AgdaFunction{Lam⅋}%
\>[17]\AgdaSymbol{(}\AgdaInductiveConstructor{lam}\AgdaSpace{}%
\AgdaSymbol{\{}\AgdaBound{t₁}\AgdaSymbol{\}}\AgdaSpace{}%
\AgdaSymbol{\{}\AgdaBound{t₂}\AgdaSymbol{\})}%
\>[38]\AgdaSymbol{=}\AgdaSpace{}%
\AgdaOperator{\AgdaField{⟦}}\AgdaSpace{}%
\AgdaSymbol{(}\AgdaField{c}\AgdaSpace{}%
\AgdaBound{t₁}\AgdaSymbol{)}\AgdaSpace{}%
\AgdaOperator{\AgdaField{↣}}\AgdaSpace{}%
\AgdaBound{t₂}\AgdaSpace{}%
\AgdaOperator{\AgdaField{⟧ᵀ}}\<%
\\
%
\>[4]\AgdaField{Retᴴ}%
\>[11]\AgdaFunction{Lam⅋}%
\>[17]\AgdaSymbol{(}\AgdaInductiveConstructor{var}\AgdaSpace{}%
\AgdaSymbol{\{}\AgdaBound{t}\AgdaSymbol{\}}\AgdaSpace{}%
\AgdaSymbol{\AgdaUnderscore{})}%
\>[38]\AgdaSymbol{=}\AgdaSpace{}%
\AgdaOperator{\AgdaField{⟦}}\AgdaSpace{}%
\AgdaBound{t}\AgdaSpace{}%
\AgdaOperator{\AgdaField{⟧ᵀ}}\<%
\\
%
\>[4]\AgdaField{Retᴴ}%
\>[11]\AgdaFunction{Lam⅋}%
\>[17]\AgdaSymbol{(}\AgdaInductiveConstructor{app}\AgdaSpace{}%
\AgdaSymbol{\{}\AgdaBound{t₁}\AgdaSymbol{\}}\AgdaSpace{}%
\AgdaSymbol{\{}\AgdaBound{t₂}\AgdaSymbol{\}}\AgdaSpace{}%
\AgdaSymbol{\AgdaUnderscore{})}%
\>[38]\AgdaSymbol{=}\AgdaSpace{}%
\AgdaOperator{\AgdaField{⟦}}\AgdaSpace{}%
\AgdaBound{t₂}\AgdaSpace{}%
\AgdaOperator{\AgdaField{⟧ᵀ}}\<%
\\
%
\>[4]\AgdaField{Fork}%
\>[11]\AgdaFunction{Lam⅋}%
\>[17]\AgdaSymbol{(}\AgdaInductiveConstructor{lam}\AgdaSpace{}%
\AgdaSymbol{\{}\AgdaBound{t₁}\AgdaSymbol{\}}\AgdaSpace{}%
\AgdaSymbol{\{}\AgdaBound{t₂}\AgdaSymbol{\})}%
\>[38]\AgdaSymbol{=}\AgdaSpace{}%
\AgdaOperator{\AgdaField{⟦}}\AgdaSpace{}%
\AgdaField{c}\AgdaSpace{}%
\AgdaBound{t₁}\AgdaSpace{}%
\AgdaOperator{\AgdaField{⟧ᵀ}}\<%
\\
%
\>[4]\AgdaField{Fork}%
\>[11]\AgdaFunction{Lam⅋}%
\>[17]\AgdaSymbol{(}\AgdaInductiveConstructor{var}\AgdaSpace{}%
\AgdaSymbol{\AgdaUnderscore{})}%
\>[38]\AgdaSymbol{=}\AgdaSpace{}%
\AgdaFunction{⊥}\<%
\\
%
\>[4]\AgdaField{Fork}%
\>[11]\AgdaFunction{Lam⅋}%
\>[17]\AgdaSymbol{(}\AgdaInductiveConstructor{app}\AgdaSpace{}%
\AgdaSymbol{\{}\AgdaBound{t₁}\AgdaSymbol{\}}\AgdaSpace{}%
\AgdaSymbol{\{}\AgdaBound{t₂}\AgdaSymbol{\}}\AgdaSpace{}%
\AgdaSymbol{\AgdaUnderscore{})}%
\>[38]\AgdaSymbol{=}\AgdaSpace{}%
\AgdaRecord{⊤}\<%
\\
%
\>[4]\AgdaField{Ty}%
\>[11]\AgdaFunction{Lam⅋}%
\>[17]\AgdaSymbol{\{}\AgdaInductiveConstructor{lam}\AgdaSpace{}%
\AgdaSymbol{\{}\AgdaBound{t₁}\AgdaSymbol{\}}\AgdaSpace{}%
\AgdaSymbol{\{}\AgdaBound{t₂}\AgdaSymbol{\}\}}\AgdaSpace{}%
\AgdaSymbol{\AgdaUnderscore{}}%
\>[38]\AgdaSymbol{=}\AgdaSpace{}%
\AgdaOperator{\AgdaField{⟦}}\AgdaSpace{}%
\AgdaBound{t₂}\AgdaSpace{}%
\AgdaOperator{\AgdaField{⟧ᵀ}}\<%
\\
%
\>[4]\AgdaField{Ty}%
\>[11]\AgdaFunction{Lam⅋}%
\>[17]\AgdaSymbol{\{}\AgdaInductiveConstructor{var}\AgdaSpace{}%
\AgdaSymbol{\AgdaUnderscore{}\}}\AgdaSpace{}%
\AgdaSymbol{()}\<%
\\
%
\>[4]\AgdaField{Ty}%
\>[11]\AgdaFunction{Lam⅋}%
\>[17]\AgdaSymbol{\{}\AgdaInductiveConstructor{app}\AgdaSpace{}%
\AgdaSymbol{\{}\AgdaBound{t₁}\AgdaSymbol{\}}\AgdaSpace{}%
\AgdaSymbol{\{}\AgdaBound{t₂}\AgdaSymbol{\}}\AgdaSpace{}%
\AgdaSymbol{\AgdaUnderscore{}\}}\AgdaSpace{}%
\AgdaSymbol{\AgdaUnderscore{}}%
\>[38]\AgdaSymbol{=}\AgdaSpace{}%
\AgdaOperator{\AgdaField{⟦}}\AgdaSpace{}%
\AgdaBound{t₁}\AgdaSpace{}%
\AgdaOperator{\AgdaField{⟧ᵀ}}\<%
\end{code}
\caption{Higher-order effect signature of $\lambda$ abstraction and application}
\label{fig:ho-lam-sig}
\end{figure}

The interface above defines a kind of \emph{higher-order abstract syntax}~\citep{PfenningE88} which piggy-backs on Agda functions for name binding.
However, unlike most Agda functions, the constructors above represent functions with side-effects.
The representation in principle supports functions with arbitrary side-effects since it is parametric in what  the higher-order effect signature \ab{H} is.
Furthermore, we can assign different operational interpretations to the operations in the interface without having to change the interface or programs written against the interface.
To illustrate we give two different implementations of the interface: one that implements a call-by-value evaluation strategy, and one that implements call-by-name.


\subsubsection{Call-by-Value}

We give a call-by-value interpretation of \af{‵lam} by generically elaborating to algebraic effect trees with any set of effects \ab{Δ}.
Our interpretation is parametric in proof witnesses that the following isomorphisms hold for value types (\ad{↔} is the type of isomorphisms from the Agda standard library):
\begin{code}[hide]%
%
\>[4]\AgdaKeyword{module}\AgdaSpace{}%
\AgdaModule{\AgdaUnderscore{}}%
\>[395I]\AgdaSymbol{⦃}\AgdaSpace{}%
\AgdaBound{l}\AgdaSpace{}%
\AgdaSymbol{:}\AgdaSpace{}%
\AgdaRecord{LamUniv}\AgdaSpace{}%
\AgdaSymbol{⦄}\<%
\\
\>[.][@{}l@{}]\<[395I]%
\>[13]\AgdaSymbol{⦃}\AgdaSpace{}%
\AgdaBound{iso₁}%
\>[401I]\AgdaSymbol{:}\AgdaSpace{}%
\AgdaSymbol{\{}\AgdaBound{t₁}\AgdaSpace{}%
\AgdaBound{t₂}\AgdaSpace{}%
\AgdaSymbol{:}\AgdaSpace{}%
\AgdaField{Type}\AgdaSymbol{\}}\<%
\\
\>[.][@{}l@{}]\<[401I]%
\>[20]\AgdaSymbol{→}\AgdaSpace{}%
\AgdaOperator{\AgdaField{⟦}}\AgdaSpace{}%
\AgdaBound{t₁}\AgdaSpace{}%
\AgdaOperator{\AgdaField{↣}}\AgdaSpace{}%
\AgdaBound{t₂}\AgdaSpace{}%
\AgdaOperator{\AgdaField{⟧ᵀ}}\AgdaSpace{}%
\AgdaOperator{\AgdaFunction{↔}}\AgdaSpace{}%
\AgdaSymbol{(}\AgdaOperator{\AgdaField{⟦}}\AgdaSpace{}%
\AgdaBound{t₁}\AgdaSpace{}%
\AgdaOperator{\AgdaField{⟧ᵀ}}\AgdaSpace{}%
\AgdaSymbol{→}\AgdaSpace{}%
\AgdaDatatype{Free}\AgdaSpace{}%
\AgdaGeneralizable{Δ}\AgdaSpace{}%
\AgdaOperator{\AgdaField{⟦}}\AgdaSpace{}%
\AgdaBound{t₂}\AgdaSpace{}%
\AgdaOperator{\AgdaField{⟧ᵀ}}\AgdaSymbol{)}\AgdaSpace{}%
\AgdaSymbol{⦄}\<%
\\
%
\>[13]\AgdaSymbol{⦃}\AgdaSpace{}%
\AgdaBound{iso₂}%
\>[423I]\AgdaSymbol{:}\AgdaSpace{}%
\AgdaSymbol{\{}\AgdaBound{t}\AgdaSpace{}%
\AgdaSymbol{:}\AgdaSpace{}%
\AgdaField{Type}\AgdaSymbol{\}}\<%
\\
\>[.][@{}l@{}]\<[423I]%
\>[20]\AgdaSymbol{→}\AgdaSpace{}%
\AgdaOperator{\AgdaField{⟦}}\AgdaSpace{}%
\AgdaField{c}\AgdaSpace{}%
\AgdaBound{t}\AgdaSpace{}%
\AgdaOperator{\AgdaField{⟧ᵀ}}\AgdaSpace{}%
\AgdaOperator{\AgdaFunction{↔}}\AgdaSpace{}%
\AgdaOperator{\AgdaField{⟦}}\AgdaSpace{}%
\AgdaBound{t}\AgdaSpace{}%
\AgdaOperator{\AgdaField{⟧ᵀ}}%
\>[41]\AgdaSymbol{⦄}\AgdaSpace{}%
\AgdaKeyword{where}\<%
\\
\>[4][@{}l@{\AgdaIndent{0}}]%
\>[6]\AgdaKeyword{open}\AgdaSpace{}%
\AgdaModule{FreeModule}\AgdaSpace{}%
\AgdaKeyword{using}\AgdaSpace{}%
\AgdaSymbol{(}\AgdaOperator{\AgdaFunction{\AgdaUnderscore{}𝓑\AgdaUnderscore{}}}\AgdaSymbol{;}\AgdaSpace{}%
\AgdaOperator{\AgdaFunction{\AgdaUnderscore{}>>\AgdaUnderscore{}}}\AgdaSymbol{)}\<%
\\
%
\>[6]\AgdaKeyword{open}\AgdaSpace{}%
\AgdaModule{ElabModule}\<%
\\
\>[0]\AgdaComment{--\ \ \ \ \ \ open\ Elab}\<%
\\
%
\\[\AgdaEmptyExtraSkip]%
\>[0]\<%
\\
\>[0][@{}l@{\AgdaIndent{0}}]%
\>[6]\AgdaKeyword{private}\AgdaSpace{}%
\AgdaOperator{\AgdaFunction{\AgdaUnderscore{}>>=\AgdaUnderscore{}}}\AgdaSpace{}%
\AgdaSymbol{=}\AgdaSpace{}%
\AgdaOperator{\AgdaFunction{\AgdaUnderscore{}𝓑\AgdaUnderscore{}}}\<%
\\
%
\>[6]\AgdaKeyword{private}\AgdaSpace{}%
\AgdaKeyword{postulate}\<%
\end{code}
\begin{code}%
\>[6][@{}l@{\AgdaIndent{1}}]%
\>[8]\AgdaPostulate{iso₁⅋}%
\>[15]\AgdaSymbol{:}\AgdaSpace{}%
\AgdaSymbol{\{}\AgdaBound{t₁}\AgdaSpace{}%
\AgdaBound{t₂}\AgdaSpace{}%
\AgdaSymbol{:}\AgdaSpace{}%
\AgdaField{Type}\AgdaSymbol{\}}%
\>[33]\AgdaSymbol{→}\AgdaSpace{}%
\AgdaOperator{\AgdaField{⟦}}\AgdaSpace{}%
\AgdaBound{t₁}\AgdaSpace{}%
\AgdaOperator{\AgdaField{↣}}\AgdaSpace{}%
\AgdaBound{t₂}%
\>[46]\AgdaOperator{\AgdaField{⟧ᵀ}}%
\>[51]\AgdaOperator{\AgdaFunction{↔}}%
\>[55]\AgdaSymbol{(}\AgdaOperator{\AgdaField{⟦}}\AgdaSpace{}%
\AgdaBound{t₁}\AgdaSpace{}%
\AgdaOperator{\AgdaField{⟧ᵀ}}\AgdaSpace{}%
\AgdaSymbol{→}\AgdaSpace{}%
\AgdaDatatype{Free}\AgdaSpace{}%
\AgdaBound{Δ}\AgdaSpace{}%
\AgdaOperator{\AgdaField{⟦}}\AgdaSpace{}%
\AgdaBound{t₂}\AgdaSpace{}%
\AgdaOperator{\AgdaField{⟧ᵀ}}\AgdaSymbol{)}\<%
\\
%
\>[8]\AgdaPostulate{iso₂⅋}%
\>[15]\AgdaSymbol{:}\AgdaSpace{}%
\AgdaSymbol{\{}\AgdaBound{t}\AgdaSpace{}%
\AgdaSymbol{:}\AgdaSpace{}%
\AgdaField{Type}\AgdaSymbol{\}}%
\>[33]\AgdaSymbol{→}\AgdaSpace{}%
\AgdaOperator{\AgdaField{⟦}}\AgdaSpace{}%
\AgdaField{c}\AgdaSpace{}%
\AgdaBound{t}%
\>[46]\AgdaOperator{\AgdaField{⟧ᵀ}}%
\>[51]\AgdaOperator{\AgdaFunction{↔}}%
\>[55]\AgdaOperator{\AgdaField{⟦}}\AgdaSpace{}%
\AgdaBound{t}\AgdaSpace{}%
\AgdaOperator{\AgdaField{⟧ᵀ}}\<%
\end{code}
%
The first isomorphism says that a function value type corresponds to a function which accepts a value of type \ab{t₁} and produces a computation whose return type matches that of the function type.
The second says that thunk types coincide with value types.
Using these isomorphisms, the following defines a call-by-value elaboration of functions:
%
\begin{code}%
%
\>[6]\AgdaFunction{eLamCBV}\AgdaSpace{}%
\AgdaSymbol{:}\AgdaSpace{}%
\AgdaFunction{Elaboration}\AgdaSpace{}%
\AgdaFunction{Lam}\AgdaSpace{}%
\AgdaBound{Δ}\<%
\\
%
\>[6]\AgdaField{alg}\AgdaSpace{}%
\AgdaFunction{eLamCBV}\AgdaSpace{}%
\AgdaSymbol{(}\AgdaInductiveConstructor{lam}\AgdaSpace{}%
\AgdaOperator{\AgdaInductiveConstructor{,}}\AgdaSpace{}%
\AgdaBound{k}\AgdaSpace{}%
\AgdaOperator{\AgdaInductiveConstructor{,}}\AgdaSpace{}%
\AgdaBound{ψ}\AgdaSymbol{)}\AgdaSpace{}%
\AgdaSymbol{=}\AgdaSpace{}%
\AgdaBound{k}\AgdaSpace{}%
\AgdaSymbol{(}\AgdaField{from}\AgdaSpace{}%
\AgdaBound{ψ}\AgdaSymbol{)}\<%
\\
%
\>[6]\AgdaField{alg}\AgdaSpace{}%
\AgdaFunction{eLamCBV}\AgdaSpace{}%
\AgdaSymbol{(}\AgdaInductiveConstructor{var}\AgdaSpace{}%
\AgdaBound{x}\AgdaSpace{}%
\AgdaOperator{\AgdaInductiveConstructor{,}}\AgdaSpace{}%
\AgdaBound{k}\AgdaSpace{}%
\AgdaOperator{\AgdaInductiveConstructor{,}}\AgdaSpace{}%
\AgdaSymbol{\AgdaUnderscore{})}\AgdaSpace{}%
\AgdaSymbol{=}\AgdaSpace{}%
\AgdaBound{k}\AgdaSpace{}%
\AgdaSymbol{(}\AgdaField{to}\AgdaSpace{}%
\AgdaBound{x}\AgdaSymbol{)}\<%
\\
%
\>[6]\AgdaField{alg}\AgdaSpace{}%
\AgdaFunction{eLamCBV}\AgdaSpace{}%
\AgdaSymbol{(}\AgdaInductiveConstructor{app}\AgdaSpace{}%
\AgdaBound{f}\AgdaSpace{}%
\AgdaOperator{\AgdaInductiveConstructor{,}}\AgdaSpace{}%
\AgdaBound{k}\AgdaSpace{}%
\AgdaOperator{\AgdaInductiveConstructor{,}}\AgdaSpace{}%
\AgdaBound{ψ}\AgdaSymbol{)}\AgdaSpace{}%
\AgdaSymbol{=}\AgdaSpace{}%
\AgdaKeyword{do}\<%
\\
\>[6][@{}l@{\AgdaIndent{0}}]%
\>[8]\AgdaBound{a}\AgdaSpace{}%
\AgdaOperator{\AgdaFunction{←}}\AgdaSpace{}%
\AgdaBound{ψ}\AgdaSpace{}%
\AgdaInductiveConstructor{tt}\<%
\\
%
\>[8]\AgdaBound{v}\AgdaSpace{}%
\AgdaOperator{\AgdaFunction{←}}\AgdaSpace{}%
\AgdaField{to}\AgdaSpace{}%
\AgdaBound{f}\AgdaSpace{}%
\AgdaSymbol{(}\AgdaField{from}\AgdaSpace{}%
\AgdaBound{a}\AgdaSymbol{)}\<%
\\
%
\>[8]\AgdaBound{k}\AgdaSpace{}%
\AgdaBound{v}\<%
\end{code}
\begin{code}[hide]%
%
\>[6]\AgdaComment{--\ instance}\<%
\\
%
\>[6]\AgdaComment{--\ \ \ eLamCBV′\ :\ Elaboration\ Lam\ Δ}\<%
\\
%
\>[6]\AgdaComment{--\ \ \ elaborate\ eLamCBV′\ =\ eLamCBV}\<%
\end{code}
%
The \ac{lam} case passes the function body given by the sub-tree \ab{ψ} as a value to the continuation, where the \aF{from} function mediates the sub-tree of type \aF{⟦~c}~\ab{t₁}~\aF{⟧ᵀ}~\as{→}~\ad{Free}~\ab{Δ}~\aF{⟦}~\ab{t₂}~\aF{⟧ᵀ} to a value type \aF{⟦}~\as{(}\aF{c}~\ab{t₁}\as{)}~\aF{↣}~\ab{t₂}~\aF{⟧ᵀ}, using the isomorphism \af{iso₁}.
The \ac{var} case uses the \aF{to} function to mediate a \aF{⟦~c}~\ab{t}~\aF{⟧ᵀ} value to a \aF{⟦}~\ab{t}~\aF{⟧ᵀ} value, using the isomorphism \af{iso₂}.
The \ac{app} case first eagerly evaluates the argument expression of the application (in the sub-tree \ab{ψ}) to an argument value, and then passes the resulting value to the function value of the application.
The resulting value is passed to the continuation.

Using the elaboration above, we can evaluate programs such as the following which uses both the higher-order lambda effect, the algebraic state effect, and assumes that our universe has a number type \aF{⟦}~\ab{num}~\aF{⟧ᵀ}~\ad{↔}~\ad{ℕ}:
\begin{code}[hide]%
\>[0][@{}l@{\AgdaIndent{4}}]%
\>[4]\AgdaKeyword{open}\AgdaSpace{}%
\AgdaKeyword{import}\AgdaSpace{}%
\AgdaModule{Data.Nat}\AgdaSpace{}%
\AgdaKeyword{using}\AgdaSpace{}%
\AgdaSymbol{(}\AgdaDatatype{ℕ}\AgdaSymbol{;}\AgdaSpace{}%
\AgdaOperator{\AgdaPrimitive{\AgdaUnderscore{}+\AgdaUnderscore{}}}\AgdaSymbol{)}\<%
\\
%
\>[4]\AgdaKeyword{module}\AgdaSpace{}%
\AgdaModule{\AgdaUnderscore{}}%
\>[518I]\AgdaSymbol{⦃}\AgdaSpace{}%
\AgdaBound{u}\AgdaSpace{}%
\AgdaSymbol{:}\AgdaSpace{}%
\AgdaRecord{LamUniv}\AgdaSpace{}%
\AgdaSymbol{⦄}\AgdaSpace{}%
\AgdaSymbol{\{}\AgdaBound{num}\AgdaSpace{}%
\AgdaSymbol{:}\AgdaSpace{}%
\AgdaField{Type}\AgdaSymbol{\}}\<%
\\
\>[.][@{}l@{}]\<[518I]%
\>[13]\AgdaSymbol{⦃}\AgdaSpace{}%
\AgdaBound{iso₁}\AgdaSpace{}%
\AgdaSymbol{:}\AgdaSpace{}%
\AgdaOperator{\AgdaField{⟦}}\AgdaSpace{}%
\AgdaBound{num}\AgdaSpace{}%
\AgdaOperator{\AgdaField{⟧ᵀ}}\AgdaSpace{}%
\AgdaOperator{\AgdaFunction{↔}}\AgdaSpace{}%
\AgdaDatatype{ℕ}\AgdaSpace{}%
\AgdaSymbol{⦄}\AgdaSpace{}%
\AgdaKeyword{where}\<%
\\
\>[4][@{}l@{\AgdaIndent{0}}]%
\>[6]\AgdaKeyword{open}\AgdaSpace{}%
\AgdaModule{HeftyModule}\AgdaSpace{}%
\AgdaKeyword{using}\AgdaSpace{}%
\AgdaSymbol{(}\AgdaOperator{\AgdaFunction{\AgdaUnderscore{}𝓑\AgdaUnderscore{}}}\AgdaSymbol{;}\AgdaSpace{}%
\AgdaOperator{\AgdaFunction{\AgdaUnderscore{}>>\AgdaUnderscore{}}}\AgdaSymbol{)}\<%
\\
%
\\[\AgdaEmptyExtraSkip]%
%
\>[6]\AgdaKeyword{private}\AgdaSpace{}%
\AgdaOperator{\AgdaFunction{\AgdaUnderscore{}>>=\AgdaUnderscore{}}}\AgdaSpace{}%
\AgdaSymbol{=}\AgdaSpace{}%
\AgdaOperator{\AgdaFunction{\AgdaUnderscore{}𝓑\AgdaUnderscore{}}}\<%
\\
%
\\[\AgdaEmptyExtraSkip]%
%
\>[6]\AgdaKeyword{private}\AgdaSpace{}%
\AgdaKeyword{instance}\<%
\\
\>[6][@{}l@{\AgdaIndent{0}}]%
\>[8]\AgdaFunction{x₀}\AgdaSpace{}%
\AgdaSymbol{:}\AgdaSpace{}%
\AgdaFunction{Lam}\AgdaSpace{}%
\AgdaOperator{\AgdaFunction{≲ᴴ}}\AgdaSpace{}%
\AgdaSymbol{(}\AgdaFunction{Lam}\AgdaSpace{}%
\AgdaOperator{\AgdaFunction{∔}}\AgdaSpace{}%
\AgdaFunction{Lift}\AgdaSpace{}%
\AgdaFunction{State}\AgdaSpace{}%
\AgdaOperator{\AgdaFunction{∔}}\AgdaSpace{}%
\AgdaFunction{Lift}\AgdaSpace{}%
\AgdaFunction{Nil}\AgdaSymbol{)}\<%
\\
%
\>[8]\AgdaFunction{x₀}\AgdaSpace{}%
\AgdaSymbol{=}\AgdaSpace{}%
\AgdaFunction{≲ᴴ-left}\<%
\\
%
\>[8]\AgdaFunction{x₁}\AgdaSpace{}%
\AgdaSymbol{:}\AgdaSpace{}%
\AgdaFunction{Lift}\AgdaSpace{}%
\AgdaFunction{State}\AgdaSpace{}%
\AgdaOperator{\AgdaFunction{≲ᴴ}}\AgdaSpace{}%
\AgdaSymbol{(}\AgdaFunction{Lam}\AgdaSpace{}%
\AgdaOperator{\AgdaFunction{∔}}\AgdaSpace{}%
\AgdaFunction{Lift}\AgdaSpace{}%
\AgdaFunction{State}\AgdaSpace{}%
\AgdaOperator{\AgdaFunction{∔}}\AgdaSpace{}%
\AgdaFunction{Lift}\AgdaSpace{}%
\AgdaFunction{Nil}\AgdaSymbol{)}\<%
\\
%
\>[8]\AgdaFunction{x₁}\AgdaSpace{}%
\AgdaSymbol{=}\AgdaSpace{}%
\AgdaFunction{≲ᴴ-right}\AgdaSpace{}%
\AgdaSymbol{⦃}\AgdaSpace{}%
\AgdaFunction{≲ᴴ-left}\AgdaSpace{}%
\AgdaSymbol{⦄}\<%
\end{code}
\begin{code}%
%
\>[6]\AgdaFunction{ex}\AgdaSpace{}%
\AgdaSymbol{:}\AgdaSpace{}%
\AgdaDatatype{Hefty}\AgdaSpace{}%
\AgdaSymbol{(}\AgdaFunction{Lam}\AgdaSpace{}%
\AgdaOperator{\AgdaFunction{∔}}\AgdaSpace{}%
\AgdaFunction{Lift}\AgdaSpace{}%
\AgdaFunction{State}\AgdaSpace{}%
\AgdaOperator{\AgdaFunction{∔}}\AgdaSpace{}%
\AgdaFunction{Lift}\AgdaSpace{}%
\AgdaFunction{Nil}\AgdaSymbol{)}\AgdaSpace{}%
\AgdaDatatype{ℕ}\<%
\\
%
\>[6]\AgdaFunction{ex}\AgdaSpace{}%
\AgdaSymbol{=}\AgdaSpace{}%
\AgdaKeyword{do}\<%
\\
\>[6][@{}l@{\AgdaIndent{0}}]%
\>[8]\AgdaOperator{\AgdaFunction{↑}}\AgdaSpace{}%
\AgdaInductiveConstructor{put}\AgdaSpace{}%
\AgdaNumber{1}\<%
\\
%
\>[8]\AgdaBound{f}\AgdaSpace{}%
\AgdaOperator{\AgdaFunction{←}}%
\>[586I]\AgdaFunction{‵lam}\AgdaSpace{}%
\AgdaSymbol{(λ}\AgdaSpace{}%
\AgdaBound{x}\AgdaSpace{}%
\AgdaSymbol{→}\AgdaSpace{}%
\AgdaKeyword{do}\<%
\\
\>[586I][@{}l@{\AgdaIndent{0}}]%
\>[14]\AgdaBound{n₁}\AgdaSpace{}%
\AgdaOperator{\AgdaFunction{←}}\AgdaSpace{}%
\AgdaFunction{‵var}\AgdaSpace{}%
\AgdaBound{x}\<%
\\
%
\>[14]\AgdaBound{n₂}\AgdaSpace{}%
\AgdaOperator{\AgdaFunction{←}}\AgdaSpace{}%
\AgdaFunction{‵var}\AgdaSpace{}%
\AgdaBound{x}\<%
\\
%
\>[14]\AgdaInductiveConstructor{pure}\AgdaSpace{}%
\AgdaSymbol{(}\AgdaField{from}\AgdaSpace{}%
\AgdaSymbol{((}\AgdaField{to}\AgdaSpace{}%
\AgdaBound{n₁}\AgdaSymbol{)}\AgdaSpace{}%
\AgdaOperator{\AgdaPrimitive{+}}\AgdaSpace{}%
\AgdaSymbol{(}\AgdaField{to}\AgdaSpace{}%
\AgdaBound{n₂}\AgdaSymbol{))))}\<%
\\
%
\>[8]\AgdaBound{v}\AgdaSpace{}%
\AgdaOperator{\AgdaFunction{←}}\AgdaSpace{}%
\AgdaFunction{‵app}\AgdaSpace{}%
\AgdaBound{f}\AgdaSpace{}%
\AgdaFunction{incr}\<%
\\
%
\>[8]\AgdaInductiveConstructor{pure}\AgdaSpace{}%
\AgdaSymbol{(}\AgdaField{to}\AgdaSpace{}%
\AgdaBound{v}\AgdaSymbol{)}\<%
\\
%
\>[8]\AgdaKeyword{where}\AgdaSpace{}%
\AgdaFunction{incr}\AgdaSpace{}%
\AgdaSymbol{=}\AgdaSpace{}%
\AgdaKeyword{do}\AgdaSpace{}%
\AgdaBound{s₀}\AgdaSpace{}%
\AgdaOperator{\AgdaFunction{←}}\AgdaSpace{}%
\AgdaOperator{\AgdaFunction{↑}}\AgdaSpace{}%
\AgdaInductiveConstructor{get}\AgdaSymbol{;}\AgdaSpace{}%
\AgdaOperator{\AgdaFunction{↑}}\AgdaSpace{}%
\AgdaInductiveConstructor{put}\AgdaSpace{}%
\AgdaSymbol{(}\AgdaBound{s₀}\AgdaSpace{}%
\AgdaOperator{\AgdaPrimitive{+}}\AgdaSpace{}%
\AgdaNumber{1}\AgdaSymbol{);}\AgdaSpace{}%
\AgdaBound{s₁}\AgdaSpace{}%
\AgdaOperator{\AgdaFunction{←}}\AgdaSpace{}%
\AgdaOperator{\AgdaFunction{↑}}\AgdaSpace{}%
\AgdaInductiveConstructor{get}\AgdaSymbol{;}\AgdaSpace{}%
\AgdaInductiveConstructor{pure}\AgdaSpace{}%
\AgdaSymbol{(}\AgdaField{from}\AgdaSpace{}%
\AgdaBound{s₁}\AgdaSymbol{)}\<%
\end{code}
The program first sets the state to \an{1}.
Then it constructs a function that binds a variable \ab{x}, dereferences the variable twice, and adds the two resulting values together.
Finally, the application in the second-to-last line applies the function with an argument expression which increments the state by \an{1} and returns the resulting value.
Running the program produces \an{4} since the state increment expression is eagerly evaluated before the function is applied.
%
\begin{code}[hide]%
%
\>[4]\AgdaKeyword{module}\AgdaSpace{}%
\AgdaModule{CBVExample}\AgdaSpace{}%
\AgdaKeyword{where}\AgdaSpace{}%
\AgdaKeyword{private}\<%
\\
\>[4][@{}l@{\AgdaIndent{0}}]%
\>[6]\AgdaKeyword{open}\AgdaSpace{}%
\AgdaKeyword{import}\AgdaSpace{}%
\AgdaModule{Data.Nat}\AgdaSpace{}%
\AgdaKeyword{using}\AgdaSpace{}%
\AgdaSymbol{(}\AgdaDatatype{ℕ}\AgdaSymbol{)}\<%
\\
%
\>[6]\AgdaKeyword{open}\AgdaSpace{}%
\AgdaModule{HeftyModule}\AgdaSpace{}%
\AgdaKeyword{using}\AgdaSpace{}%
\AgdaSymbol{(}\AgdaOperator{\AgdaFunction{\AgdaUnderscore{}𝓑\AgdaUnderscore{}}}\AgdaSymbol{;}\AgdaSpace{}%
\AgdaOperator{\AgdaFunction{\AgdaUnderscore{}>>\AgdaUnderscore{}}}\AgdaSymbol{)}\<%
\\
%
\>[6]\AgdaKeyword{open}\AgdaSpace{}%
\AgdaModule{ElabModule}\<%
\\
%
\>[6]\AgdaKeyword{open}\AgdaSpace{}%
\AgdaKeyword{import}\AgdaSpace{}%
\AgdaModule{Function.Construct.Identity}%
\>[49]\AgdaKeyword{using}\AgdaSpace{}%
\AgdaSymbol{(}\AgdaFunction{↔-id}\AgdaSymbol{)}\<%
\\
%
\>[6]\AgdaKeyword{open}\AgdaSpace{}%
\AgdaModule{Inverse}\<%
\\
%
\>[6]\AgdaComment{--\ open\ Elab}\<%
\\
%
\\[\AgdaEmptyExtraSkip]%
%
\\[\AgdaEmptyExtraSkip]%
%
\>[6]\AgdaKeyword{data}\AgdaSpace{}%
\AgdaDatatype{LamType}\AgdaSpace{}%
\AgdaSymbol{:}\AgdaSpace{}%
\AgdaPrimitive{Set}\AgdaSpace{}%
\AgdaKeyword{where}\<%
\\
\>[6][@{}l@{\AgdaIndent{0}}]%
\>[8]\AgdaOperator{\AgdaInductiveConstructor{\AgdaUnderscore{}⟶\AgdaUnderscore{}}}\AgdaSpace{}%
\AgdaSymbol{:}\AgdaSpace{}%
\AgdaSymbol{(}\AgdaBound{t₁}\AgdaSpace{}%
\AgdaBound{t₂}\AgdaSpace{}%
\AgdaSymbol{:}\AgdaSpace{}%
\AgdaDatatype{LamType}\AgdaSymbol{)}\AgdaSpace{}%
\AgdaSymbol{→}\AgdaSpace{}%
\AgdaDatatype{LamType}\<%
\\
%
\>[8]\AgdaInductiveConstructor{num}\AgdaSpace{}%
\AgdaSymbol{:}\AgdaSpace{}%
\AgdaDatatype{LamType}\<%
\\
%
\\[\AgdaEmptyExtraSkip]%
%
\>[6]\AgdaKeyword{instance}\<%
\\
\>[6][@{}l@{\AgdaIndent{0}}]%
\>[8]\AgdaFunction{CBVUniv}\AgdaSpace{}%
\AgdaSymbol{:}\AgdaSpace{}%
\AgdaRecord{Univ}\<%
\\
%
\>[8]\AgdaField{Type}\AgdaSpace{}%
\AgdaSymbol{⦃}\AgdaSpace{}%
\AgdaFunction{CBVUniv}\AgdaSpace{}%
\AgdaSymbol{⦄}\AgdaSpace{}%
\AgdaSymbol{=}\AgdaSpace{}%
\AgdaDatatype{LamType}\<%
\\
%
\>[8]\AgdaOperator{\AgdaField{⟦\AgdaUnderscore{}⟧ᵀ}}\AgdaSpace{}%
\AgdaSymbol{⦃}\AgdaSpace{}%
\AgdaFunction{CBVUniv}\AgdaSpace{}%
\AgdaSymbol{⦄}\AgdaSpace{}%
\AgdaSymbol{(}\AgdaBound{t}\AgdaSpace{}%
\AgdaOperator{\AgdaInductiveConstructor{⟶}}\AgdaSpace{}%
\AgdaBound{t₁}\AgdaSymbol{)}%
\>[35]\AgdaSymbol{=}\AgdaSpace{}%
\AgdaOperator{\AgdaField{⟦}}\AgdaSpace{}%
\AgdaBound{t}\AgdaSpace{}%
\AgdaOperator{\AgdaField{⟧ᵀ}}\AgdaSpace{}%
\AgdaSymbol{→}\AgdaSpace{}%
\AgdaDatatype{Free}\AgdaSpace{}%
\AgdaSymbol{(}\AgdaFunction{State}\AgdaSpace{}%
\AgdaOperator{\AgdaFunction{⊕}}\AgdaSpace{}%
\AgdaFunction{Nil}\AgdaSymbol{)}\AgdaSpace{}%
\AgdaOperator{\AgdaField{⟦}}\AgdaSpace{}%
\AgdaBound{t₁}\AgdaSpace{}%
\AgdaOperator{\AgdaField{⟧ᵀ}}\<%
\\
%
\>[8]\AgdaOperator{\AgdaField{⟦\AgdaUnderscore{}⟧ᵀ}}\AgdaSpace{}%
\AgdaSymbol{⦃}\AgdaSpace{}%
\AgdaFunction{CBVUniv}\AgdaSpace{}%
\AgdaSymbol{⦄}\AgdaSpace{}%
\AgdaInductiveConstructor{num}%
\>[35]\AgdaSymbol{=}\AgdaSpace{}%
\AgdaDatatype{ℕ}\<%
\\
%
\\[\AgdaEmptyExtraSkip]%
%
\>[8]\AgdaFunction{iso-num}\AgdaSpace{}%
\AgdaSymbol{:}\AgdaSpace{}%
\AgdaDatatype{ℕ}\AgdaSpace{}%
\AgdaOperator{\AgdaFunction{↔}}\AgdaSpace{}%
\AgdaOperator{\AgdaField{⟦}}\AgdaSpace{}%
\AgdaInductiveConstructor{num}\AgdaSpace{}%
\AgdaOperator{\AgdaField{⟧ᵀ}}\<%
\\
%
\>[8]\AgdaFunction{iso-num}\AgdaSpace{}%
\AgdaSymbol{=}\AgdaSpace{}%
\AgdaFunction{↔-id}\AgdaSpace{}%
\AgdaSymbol{\AgdaUnderscore{}}\<%
\\
%
\\[\AgdaEmptyExtraSkip]%
%
\>[8]\AgdaFunction{iso-fun}%
\>[695I]\AgdaSymbol{:}\AgdaSpace{}%
\AgdaSymbol{\{}\AgdaBound{t₁}\AgdaSpace{}%
\AgdaBound{t₂}\AgdaSpace{}%
\AgdaSymbol{:}\AgdaSpace{}%
\AgdaDatatype{LamType}\AgdaSymbol{\}}\<%
\\
\>[.][@{}l@{}]\<[695I]%
\>[16]\AgdaSymbol{→}\AgdaSpace{}%
\AgdaSymbol{(}\AgdaOperator{\AgdaField{⟦}}\AgdaSpace{}%
\AgdaBound{t₁}\AgdaSpace{}%
\AgdaOperator{\AgdaField{⟧ᵀ}}\AgdaSpace{}%
\AgdaSymbol{→}\AgdaSpace{}%
\AgdaDatatype{Free}\AgdaSpace{}%
\AgdaSymbol{(}\AgdaFunction{State}\AgdaSpace{}%
\AgdaOperator{\AgdaFunction{⊕}}\AgdaSpace{}%
\AgdaFunction{Nil}\AgdaSymbol{)}\AgdaSpace{}%
\AgdaOperator{\AgdaField{⟦}}\AgdaSpace{}%
\AgdaBound{t₂}\AgdaSpace{}%
\AgdaOperator{\AgdaField{⟧ᵀ}}\AgdaSymbol{)}\AgdaSpace{}%
\AgdaOperator{\AgdaFunction{↔}}\AgdaSpace{}%
\AgdaOperator{\AgdaField{⟦}}\AgdaSpace{}%
\AgdaBound{t₁}\AgdaSpace{}%
\AgdaOperator{\AgdaInductiveConstructor{⟶}}\AgdaSpace{}%
\AgdaBound{t₂}\AgdaSpace{}%
\AgdaOperator{\AgdaField{⟧ᵀ}}\<%
\\
%
\>[8]\AgdaFunction{iso-fun}\AgdaSpace{}%
\AgdaSymbol{=}\AgdaSpace{}%
\AgdaFunction{↔-id}\AgdaSpace{}%
\AgdaSymbol{\AgdaUnderscore{}}\<%
\\
%
\\[\AgdaEmptyExtraSkip]%
%
\>[8]\AgdaFunction{iso-c}\AgdaSpace{}%
\AgdaSymbol{:}\AgdaSpace{}%
\AgdaSymbol{\{}\AgdaBound{t}\AgdaSpace{}%
\AgdaSymbol{:}\AgdaSpace{}%
\AgdaDatatype{LamType}\AgdaSymbol{\}}\AgdaSpace{}%
\AgdaSymbol{→}\AgdaSpace{}%
\AgdaOperator{\AgdaField{⟦}}\AgdaSpace{}%
\AgdaBound{t}\AgdaSpace{}%
\AgdaOperator{\AgdaField{⟧ᵀ}}\AgdaSpace{}%
\AgdaOperator{\AgdaFunction{↔}}\AgdaSpace{}%
\AgdaOperator{\AgdaField{⟦}}\AgdaSpace{}%
\AgdaFunction{id}\AgdaSpace{}%
\AgdaBound{t}\AgdaSpace{}%
\AgdaOperator{\AgdaField{⟧ᵀ}}\<%
\\
%
\>[8]\AgdaFunction{iso-c}\AgdaSpace{}%
\AgdaSymbol{=}\AgdaSpace{}%
\AgdaFunction{↔-id}\AgdaSpace{}%
\AgdaSymbol{\AgdaUnderscore{}}\<%
\\
%
\\[\AgdaEmptyExtraSkip]%
%
\>[8]\AgdaFunction{LamCBVUniv}\AgdaSpace{}%
\AgdaSymbol{:}\AgdaSpace{}%
\AgdaRecord{LamUniv}\<%
\\
%
\>[8]\AgdaField{u}%
\>[13]\AgdaSymbol{⦃}\AgdaSpace{}%
\AgdaFunction{LamCBVUniv}\AgdaSpace{}%
\AgdaSymbol{⦄}\AgdaSpace{}%
\AgdaSymbol{=}\AgdaSpace{}%
\AgdaFunction{CBVUniv}\<%
\\
%
\>[8]\AgdaOperator{\AgdaField{\AgdaUnderscore{}↣\AgdaUnderscore{}}}%
\>[13]\AgdaSymbol{⦃}\AgdaSpace{}%
\AgdaFunction{LamCBVUniv}\AgdaSpace{}%
\AgdaSymbol{⦄}\AgdaSpace{}%
\AgdaSymbol{=}\AgdaSpace{}%
\AgdaOperator{\AgdaInductiveConstructor{\AgdaUnderscore{}⟶\AgdaUnderscore{}}}\<%
\\
%
\>[8]\AgdaField{c}%
\>[13]\AgdaSymbol{⦃}\AgdaSpace{}%
\AgdaFunction{LamCBVUniv}\AgdaSpace{}%
\AgdaSymbol{⦄}\AgdaSpace{}%
\AgdaSymbol{=}\AgdaSpace{}%
\AgdaFunction{id}\<%
\\
%
\\[\AgdaEmptyExtraSkip]%
%
\>[6]\AgdaKeyword{module}\AgdaSpace{}%
\AgdaModule{\AgdaUnderscore{}}\AgdaSpace{}%
\AgdaKeyword{where}\<%
\\
\>[6][@{}l@{\AgdaIndent{0}}]%
\>[8]\AgdaKeyword{private}\AgdaSpace{}%
\AgdaKeyword{instance}\<%
\\
\>[8][@{}l@{\AgdaIndent{0}}]%
\>[10]\AgdaFunction{x₀}\AgdaSpace{}%
\AgdaSymbol{:}\AgdaSpace{}%
\AgdaFunction{Lam}\AgdaSpace{}%
\AgdaOperator{\AgdaFunction{≲ᴴ}}\AgdaSpace{}%
\AgdaSymbol{(}\AgdaFunction{Lam}\AgdaSpace{}%
\AgdaOperator{\AgdaFunction{∔}}\AgdaSpace{}%
\AgdaFunction{Lift}\AgdaSpace{}%
\AgdaFunction{State}\AgdaSpace{}%
\AgdaOperator{\AgdaFunction{∔}}\AgdaSpace{}%
\AgdaFunction{Lift}\AgdaSpace{}%
\AgdaFunction{Nil}\AgdaSymbol{)}\<%
\\
%
\>[10]\AgdaFunction{x₀}\AgdaSpace{}%
\AgdaSymbol{=}\AgdaSpace{}%
\AgdaFunction{≲ᴴ-left}\<%
\\
%
\>[10]\AgdaFunction{x₁}\AgdaSpace{}%
\AgdaSymbol{:}\AgdaSpace{}%
\AgdaFunction{Lift}\AgdaSpace{}%
\AgdaFunction{State}\AgdaSpace{}%
\AgdaOperator{\AgdaFunction{≲ᴴ}}\AgdaSpace{}%
\AgdaSymbol{(}\AgdaFunction{Lam}\AgdaSpace{}%
\AgdaOperator{\AgdaFunction{∔}}\AgdaSpace{}%
\AgdaFunction{Lift}\AgdaSpace{}%
\AgdaFunction{State}\AgdaSpace{}%
\AgdaOperator{\AgdaFunction{∔}}\AgdaSpace{}%
\AgdaFunction{Lift}\AgdaSpace{}%
\AgdaFunction{Nil}\AgdaSymbol{)}\<%
\\
%
\>[10]\AgdaFunction{x₁}\AgdaSpace{}%
\AgdaSymbol{=}\AgdaSpace{}%
\AgdaFunction{≲ᴴ-right}\AgdaSpace{}%
\AgdaSymbol{⦃}\AgdaSpace{}%
\AgdaFunction{≲ᴴ-left}\AgdaSpace{}%
\AgdaSymbol{⦄}\<%
\\
%
\\[\AgdaEmptyExtraSkip]%
%
\>[10]\AgdaFunction{y₀}\AgdaSpace{}%
\AgdaSymbol{:}\AgdaSpace{}%
\AgdaFunction{State}\AgdaSpace{}%
\AgdaOperator{\AgdaFunction{≲}}\AgdaSpace{}%
\AgdaSymbol{(}\AgdaFunction{State}\AgdaSpace{}%
\AgdaOperator{\AgdaFunction{⊕}}\AgdaSpace{}%
\AgdaFunction{Nil}\AgdaSymbol{)}\<%
\\
%
\>[10]\AgdaFunction{y₀}\AgdaSpace{}%
\AgdaSymbol{=}\AgdaSpace{}%
\AgdaFunction{≲-left}\<%
\end{code}
\begin{code}%
%
\>[8]\AgdaFunction{elab-cbv}\AgdaSpace{}%
\AgdaSymbol{:}\AgdaSpace{}%
\AgdaFunction{Elaboration}\AgdaSpace{}%
\AgdaSymbol{(}\AgdaFunction{Lam}\AgdaSpace{}%
\AgdaOperator{\AgdaFunction{∔}}\AgdaSpace{}%
\AgdaFunction{Lift}\AgdaSpace{}%
\AgdaFunction{State}\AgdaSpace{}%
\AgdaOperator{\AgdaFunction{∔}}\AgdaSpace{}%
\AgdaFunction{Lift}\AgdaSpace{}%
\AgdaFunction{Nil}\AgdaSymbol{)}\AgdaSpace{}%
\AgdaSymbol{(}\AgdaFunction{State}\AgdaSpace{}%
\AgdaOperator{\AgdaFunction{⊕}}\AgdaSpace{}%
\AgdaFunction{Nil}\AgdaSymbol{)}\<%
\\
%
\>[8]\AgdaFunction{elab-cbv}\AgdaSpace{}%
\AgdaSymbol{=}\AgdaSpace{}%
\AgdaFunction{eLamCBV}\AgdaSpace{}%
\AgdaOperator{\AgdaFunction{⋎}}\AgdaSpace{}%
\AgdaFunction{eLift}\AgdaSpace{}%
\AgdaOperator{\AgdaFunction{⋎}}\AgdaSpace{}%
\AgdaFunction{eNil}\<%
\\
%
\\[\AgdaEmptyExtraSkip]%
%
\>[8]\AgdaFunction{test-ex-cbv}\AgdaSpace{}%
\AgdaSymbol{:}\AgdaSpace{}%
\AgdaFunction{un}\AgdaSpace{}%
\AgdaSymbol{((}\AgdaOperator{\AgdaFunction{given}}\AgdaSpace{}%
\AgdaFunction{hSt}\AgdaSpace{}%
\AgdaOperator{\AgdaFunction{handle}}\AgdaSpace{}%
\AgdaSymbol{(}\AgdaFunction{elaborate}\AgdaSpace{}%
\AgdaFunction{elab-cbv}\AgdaSpace{}%
\AgdaFunction{ex}\AgdaSymbol{))}\AgdaSpace{}%
\AgdaNumber{0}\AgdaSymbol{)}\AgdaSpace{}%
\AgdaOperator{\AgdaDatatype{≡}}\AgdaSpace{}%
\AgdaSymbol{(}\AgdaNumber{4}\AgdaSpace{}%
\AgdaOperator{\AgdaInductiveConstructor{,}}\AgdaSpace{}%
\AgdaNumber{2}\AgdaSymbol{)}\<%
\\
%
\>[8]\AgdaFunction{test-ex-cbv}\AgdaSpace{}%
\AgdaSymbol{=}\AgdaSpace{}%
\AgdaInductiveConstructor{refl}\<%
\end{code}

\subsubsection{Call-by-Name}

The key difference between the call-by-value and the call-by-name interpretation of our $\lambda$ operations is that we now assume that thunks are computations.
That is, we assume that the following isomorphisms hold for value types:
\begin{code}[hide]%
%
\>[4]\AgdaKeyword{module}\AgdaSpace{}%
\AgdaModule{\AgdaUnderscore{}}%
\>[823I]\AgdaSymbol{⦃}\AgdaSpace{}%
\AgdaBound{u}\AgdaSpace{}%
\AgdaSymbol{:}\AgdaSpace{}%
\AgdaRecord{LamUniv}\AgdaSpace{}%
\AgdaSymbol{⦄}\<%
\\
\>[.][@{}l@{}]\<[823I]%
\>[13]\AgdaSymbol{⦃}\AgdaSpace{}%
\AgdaBound{iso₁}%
\>[829I]\AgdaSymbol{:}\AgdaSpace{}%
\AgdaSymbol{\{}\AgdaBound{t₁}\AgdaSpace{}%
\AgdaBound{t₂}\AgdaSpace{}%
\AgdaSymbol{:}\AgdaSpace{}%
\AgdaField{Type}\AgdaSymbol{\}}\<%
\\
\>[.][@{}l@{}]\<[829I]%
\>[20]\AgdaSymbol{→}\AgdaSpace{}%
\AgdaOperator{\AgdaField{⟦}}\AgdaSpace{}%
\AgdaBound{t₁}\AgdaSpace{}%
\AgdaOperator{\AgdaField{↣}}\AgdaSpace{}%
\AgdaBound{t₂}\AgdaSpace{}%
\AgdaOperator{\AgdaField{⟧ᵀ}}\AgdaSpace{}%
\AgdaOperator{\AgdaFunction{↔}}\AgdaSpace{}%
\AgdaSymbol{(}\AgdaOperator{\AgdaField{⟦}}\AgdaSpace{}%
\AgdaBound{t₁}\AgdaSpace{}%
\AgdaOperator{\AgdaField{⟧ᵀ}}\AgdaSpace{}%
\AgdaSymbol{→}\AgdaSpace{}%
\AgdaDatatype{Free}\AgdaSpace{}%
\AgdaGeneralizable{Δ}\AgdaSpace{}%
\AgdaOperator{\AgdaField{⟦}}\AgdaSpace{}%
\AgdaBound{t₂}\AgdaSpace{}%
\AgdaOperator{\AgdaField{⟧ᵀ}}\AgdaSymbol{)}%
\>[65]\AgdaSymbol{⦄}\<%
\\
%
\>[13]\AgdaSymbol{⦃}\AgdaSpace{}%
\AgdaBound{iso₂}%
\>[850I]\AgdaSymbol{:}\AgdaSpace{}%
\AgdaSymbol{\{}\AgdaBound{t}\AgdaSpace{}%
\AgdaSymbol{:}\AgdaSpace{}%
\AgdaField{Type}\AgdaSymbol{\}}\<%
\\
\>[.][@{}l@{}]\<[850I]%
\>[20]\AgdaSymbol{→}\AgdaSpace{}%
\AgdaOperator{\AgdaField{⟦}}\AgdaSpace{}%
\AgdaField{c}\AgdaSpace{}%
\AgdaBound{t}\AgdaSpace{}%
\AgdaOperator{\AgdaField{⟧ᵀ}}\AgdaSpace{}%
\AgdaOperator{\AgdaFunction{↔}}\AgdaSpace{}%
\AgdaDatatype{Free}\AgdaSpace{}%
\AgdaGeneralizable{Δ}\AgdaSpace{}%
\AgdaOperator{\AgdaField{⟦}}\AgdaSpace{}%
\AgdaBound{t}\AgdaSpace{}%
\AgdaOperator{\AgdaField{⟧ᵀ}}\AgdaSpace{}%
\AgdaSymbol{⦄}\AgdaSpace{}%
\AgdaKeyword{where}\<%
\\
\>[4][@{}l@{\AgdaIndent{0}}]%
\>[6]\AgdaKeyword{open}\AgdaSpace{}%
\AgdaModule{FreeModule}\AgdaSpace{}%
\AgdaKeyword{using}\AgdaSpace{}%
\AgdaSymbol{(}\AgdaOperator{\AgdaFunction{\AgdaUnderscore{}𝓑\AgdaUnderscore{}}}\AgdaSymbol{;}\AgdaSpace{}%
\AgdaOperator{\AgdaFunction{\AgdaUnderscore{}>>\AgdaUnderscore{}}}\AgdaSymbol{)}\<%
\\
%
\>[6]\AgdaKeyword{open}\AgdaSpace{}%
\AgdaKeyword{import}\AgdaSpace{}%
\AgdaModule{Data.Nat}\AgdaSpace{}%
\AgdaKeyword{using}\AgdaSpace{}%
\AgdaSymbol{(}\AgdaDatatype{ℕ}\AgdaSymbol{)}\<%
\\
%
\>[6]\AgdaKeyword{open}\AgdaSpace{}%
\AgdaModule{ElabModule}\<%
\\
\>[0]\AgdaComment{--\ \ \ \ \ \ open\ Elab}\<%
\\
%
\\[\AgdaEmptyExtraSkip]%
\>[0][@{}l@{\AgdaIndent{0}}]%
\>[6]\AgdaKeyword{private}\AgdaSpace{}%
\AgdaKeyword{postulate}\<%
\end{code}
\begin{code}%
\>[6][@{}l@{\AgdaIndent{1}}]%
\>[8]\AgdaPostulate{iso₁⅋}%
\>[15]\AgdaSymbol{:}%
\>[18]\AgdaSymbol{\{}\AgdaBound{t₁}\AgdaSpace{}%
\AgdaBound{t₂}\AgdaSpace{}%
\AgdaSymbol{:}\AgdaSpace{}%
\AgdaField{Type}\AgdaSymbol{\}}%
\>[34]\AgdaSymbol{→}\AgdaSpace{}%
\AgdaOperator{\AgdaField{⟦}}\AgdaSpace{}%
\AgdaBound{t₁}\AgdaSpace{}%
\AgdaOperator{\AgdaField{↣}}\AgdaSpace{}%
\AgdaBound{t₂}\AgdaSpace{}%
\AgdaOperator{\AgdaField{⟧ᵀ}}%
\>[50]\AgdaOperator{\AgdaFunction{↔}}%
\>[53]\AgdaSymbol{(}\AgdaOperator{\AgdaField{⟦}}\AgdaSpace{}%
\AgdaBound{t₁}\AgdaSpace{}%
\AgdaOperator{\AgdaField{⟧ᵀ}}\AgdaSpace{}%
\AgdaSymbol{→}\AgdaSpace{}%
\AgdaDatatype{Free}\AgdaSpace{}%
\AgdaBound{Δ}\AgdaSpace{}%
\AgdaOperator{\AgdaField{⟦}}\AgdaSpace{}%
\AgdaBound{t₂}\AgdaSpace{}%
\AgdaOperator{\AgdaField{⟧ᵀ}}\AgdaSymbol{)}\<%
\\
%
\>[8]\AgdaPostulate{iso₂⅋}%
\>[15]\AgdaSymbol{:}%
\>[18]\AgdaSymbol{\{}\AgdaBound{t}\AgdaSpace{}%
\AgdaSymbol{:}\AgdaSpace{}%
\AgdaField{Type}\AgdaSymbol{\}}%
\>[34]\AgdaSymbol{→}\AgdaSpace{}%
\AgdaOperator{\AgdaField{⟦}}\AgdaSpace{}%
\AgdaField{c}\AgdaSpace{}%
\AgdaBound{t}\AgdaSpace{}%
\AgdaOperator{\AgdaField{⟧ᵀ}}%
\>[50]\AgdaOperator{\AgdaFunction{↔}}%
\>[53]\AgdaDatatype{Free}\AgdaSpace{}%
\AgdaBound{Δ}\AgdaSpace{}%
\AgdaOperator{\AgdaField{⟦}}\AgdaSpace{}%
\AgdaBound{t}\AgdaSpace{}%
\AgdaOperator{\AgdaField{⟧ᵀ}}\<%
\end{code}
Using these isomorphisms, the following defines a call-by-name elaboration of functions:
\begin{code}%
%
\>[6]\AgdaFunction{eLamCBN}\AgdaSpace{}%
\AgdaSymbol{:}\AgdaSpace{}%
\AgdaFunction{Elaboration}\AgdaSpace{}%
\AgdaFunction{Lam}\AgdaSpace{}%
\AgdaBound{Δ}\<%
\\
%
\>[6]\AgdaField{alg}\AgdaSpace{}%
\AgdaFunction{eLamCBN}\AgdaSpace{}%
\AgdaSymbol{(}\AgdaInductiveConstructor{lam}\AgdaSpace{}%
\AgdaOperator{\AgdaInductiveConstructor{,}}\AgdaSpace{}%
\AgdaBound{k}\AgdaSpace{}%
\AgdaOperator{\AgdaInductiveConstructor{,}}\AgdaSpace{}%
\AgdaBound{ψ}\AgdaSymbol{)}\AgdaSpace{}%
\AgdaSymbol{=}\AgdaSpace{}%
\AgdaBound{k}\AgdaSpace{}%
\AgdaSymbol{(}\AgdaField{from}\AgdaSpace{}%
\AgdaBound{ψ}\AgdaSymbol{)}\<%
\\
%
\>[6]\AgdaField{alg}\AgdaSpace{}%
\AgdaFunction{eLamCBN}\AgdaSpace{}%
\AgdaSymbol{(}\AgdaInductiveConstructor{var}\AgdaSpace{}%
\AgdaBound{x}\AgdaSpace{}%
\AgdaOperator{\AgdaInductiveConstructor{,}}\AgdaSpace{}%
\AgdaBound{k}\AgdaSpace{}%
\AgdaOperator{\AgdaInductiveConstructor{,}}\AgdaSpace{}%
\AgdaSymbol{\AgdaUnderscore{})}\AgdaSpace{}%
\AgdaSymbol{=}\AgdaSpace{}%
\AgdaField{to}\AgdaSpace{}%
\AgdaBound{x}\AgdaSpace{}%
\AgdaOperator{\AgdaFunction{𝓑}}\AgdaSpace{}%
\AgdaBound{k}\<%
\\
%
\>[6]\AgdaField{alg}\AgdaSpace{}%
\AgdaFunction{eLamCBN}\AgdaSpace{}%
\AgdaSymbol{(}\AgdaInductiveConstructor{app}\AgdaSpace{}%
\AgdaBound{f}\AgdaSpace{}%
\AgdaOperator{\AgdaInductiveConstructor{,}}\AgdaSpace{}%
\AgdaBound{k}\AgdaSpace{}%
\AgdaOperator{\AgdaInductiveConstructor{,}}%
\>[32]\AgdaBound{ψ}\AgdaSymbol{)}\AgdaSpace{}%
\AgdaSymbol{=}\AgdaSpace{}%
\AgdaField{to}\AgdaSpace{}%
\AgdaBound{f}\AgdaSpace{}%
\AgdaSymbol{(}\AgdaField{from}\AgdaSpace{}%
\AgdaSymbol{(}\AgdaBound{ψ}\AgdaSpace{}%
\AgdaInductiveConstructor{tt}\AgdaSymbol{))}\AgdaSpace{}%
\AgdaOperator{\AgdaFunction{𝓑}}\AgdaSpace{}%
\AgdaBound{k}\<%
\end{code}
\begin{code}[hide]%
%
\>[6]\AgdaComment{--\ instance}\<%
\\
%
\>[6]\AgdaComment{--\ \ \ eLamCBN′\ :\ Elaboration\ Lam\ Δ}\<%
\\
%
\>[6]\AgdaComment{--\ \ \ elaborate\ eLamCBN′\ =\ eLamCBN}\<%
\end{code}
%
The case for \ac{lam} is the same as the call-by-value elaboration.
The case for \ac{var} now needs to force the thunk by running the computation and passing its result to \ab{k}.
The case for \ac{app} passes the argument sub-tree (\ab{ψ}) as an argument to the function \ab{f}, runs the computation resulting from doing so, and then passes its result to \ab{k}.
%
\begin{code}[hide]%
\>[0][@{}l@{\AgdaIndent{4}}]%
\>[4]\AgdaKeyword{module}\AgdaSpace{}%
\AgdaModule{CBNExample}\AgdaSpace{}%
\AgdaKeyword{where}\AgdaSpace{}%
\AgdaKeyword{private}\<%
\\
\>[4][@{}l@{\AgdaIndent{0}}]%
\>[6]\AgdaKeyword{open}\AgdaSpace{}%
\AgdaKeyword{import}\AgdaSpace{}%
\AgdaModule{Data.Nat}\AgdaSpace{}%
\AgdaKeyword{using}\AgdaSpace{}%
\AgdaSymbol{(}\AgdaDatatype{ℕ}\AgdaSymbol{)}\<%
\\
%
\>[6]\AgdaKeyword{open}\AgdaSpace{}%
\AgdaModule{HeftyModule}\AgdaSpace{}%
\AgdaKeyword{using}\AgdaSpace{}%
\AgdaSymbol{(}\AgdaOperator{\AgdaFunction{\AgdaUnderscore{}𝓑\AgdaUnderscore{}}}\AgdaSymbol{;}\AgdaSpace{}%
\AgdaOperator{\AgdaFunction{\AgdaUnderscore{}>>\AgdaUnderscore{}}}\AgdaSymbol{)}\<%
\\
%
\>[6]\AgdaKeyword{open}\AgdaSpace{}%
\AgdaModule{ElabModule}\<%
\\
%
\>[6]\AgdaKeyword{open}\AgdaSpace{}%
\AgdaKeyword{import}\AgdaSpace{}%
\AgdaModule{Function.Construct.Identity}%
\>[49]\AgdaKeyword{using}\AgdaSpace{}%
\AgdaSymbol{(}\AgdaFunction{↔-id}\AgdaSymbol{)}\<%
\\
%
\>[6]\AgdaKeyword{open}\AgdaSpace{}%
\AgdaModule{Inverse}\AgdaSpace{}%
\AgdaSymbol{⦃}\AgdaSpace{}%
\AgdaSymbol{...}\AgdaSpace{}%
\AgdaSymbol{⦄}\<%
\\
%
\>[6]\AgdaComment{--\ open\ Elab}\<%
\\
%
\\[\AgdaEmptyExtraSkip]%
%
\\[\AgdaEmptyExtraSkip]%
%
\>[6]\AgdaKeyword{data}\AgdaSpace{}%
\AgdaDatatype{LamType}\AgdaSpace{}%
\AgdaSymbol{:}\AgdaSpace{}%
\AgdaPrimitive{Set}\AgdaSpace{}%
\AgdaKeyword{where}\<%
\\
\>[6][@{}l@{\AgdaIndent{0}}]%
\>[8]\AgdaOperator{\AgdaInductiveConstructor{\AgdaUnderscore{}⟶\AgdaUnderscore{}}}\AgdaSpace{}%
\AgdaSymbol{:}\AgdaSpace{}%
\AgdaSymbol{(}\AgdaBound{t₁}\AgdaSpace{}%
\AgdaBound{t₂}\AgdaSpace{}%
\AgdaSymbol{:}\AgdaSpace{}%
\AgdaDatatype{LamType}\AgdaSymbol{)}%
\>[34]\AgdaSymbol{→}\AgdaSpace{}%
\AgdaDatatype{LamType}\<%
\\
%
\>[8]\AgdaInductiveConstructor{num}%
\>[13]\AgdaSymbol{:}%
\>[35]\AgdaDatatype{LamType}\<%
\\
%
\>[8]\AgdaInductiveConstructor{susp}\AgdaSpace{}%
\AgdaSymbol{:}\AgdaSpace{}%
\AgdaDatatype{LamType}%
\>[36]\AgdaSymbol{→}\AgdaSpace{}%
\AgdaDatatype{LamType}\<%
\\
%
\\[\AgdaEmptyExtraSkip]%
%
\>[6]\AgdaKeyword{instance}\<%
\\
\>[6][@{}l@{\AgdaIndent{0}}]%
\>[8]\AgdaFunction{CBNUniv}\AgdaSpace{}%
\AgdaSymbol{:}\AgdaSpace{}%
\AgdaRecord{Univ}\<%
\\
%
\>[8]\AgdaField{Type}\AgdaSpace{}%
\AgdaSymbol{⦃}\AgdaSpace{}%
\AgdaFunction{CBNUniv}\AgdaSpace{}%
\AgdaSymbol{⦄}\AgdaSpace{}%
\AgdaSymbol{=}\AgdaSpace{}%
\AgdaDatatype{LamType}\<%
\\
%
\>[8]\AgdaOperator{\AgdaField{⟦\AgdaUnderscore{}⟧ᵀ}}\AgdaSpace{}%
\AgdaSymbol{⦃}\AgdaSpace{}%
\AgdaFunction{CBNUniv}\AgdaSpace{}%
\AgdaSymbol{⦄}\AgdaSpace{}%
\AgdaSymbol{(}\AgdaBound{t}\AgdaSpace{}%
\AgdaOperator{\AgdaInductiveConstructor{⟶}}\AgdaSpace{}%
\AgdaBound{t₁}\AgdaSymbol{)}%
\>[35]\AgdaSymbol{=}\AgdaSpace{}%
\AgdaOperator{\AgdaField{⟦}}\AgdaSpace{}%
\AgdaBound{t}\AgdaSpace{}%
\AgdaOperator{\AgdaField{⟧ᵀ}}\AgdaSpace{}%
\AgdaSymbol{→}\AgdaSpace{}%
\AgdaDatatype{Free}\AgdaSpace{}%
\AgdaSymbol{(}\AgdaFunction{State}\AgdaSpace{}%
\AgdaOperator{\AgdaFunction{⊕}}\AgdaSpace{}%
\AgdaFunction{Nil}\AgdaSymbol{)}\AgdaSpace{}%
\AgdaOperator{\AgdaField{⟦}}\AgdaSpace{}%
\AgdaBound{t₁}\AgdaSpace{}%
\AgdaOperator{\AgdaField{⟧ᵀ}}\<%
\\
%
\>[8]\AgdaOperator{\AgdaField{⟦\AgdaUnderscore{}⟧ᵀ}}\AgdaSpace{}%
\AgdaSymbol{⦃}\AgdaSpace{}%
\AgdaFunction{CBNUniv}\AgdaSpace{}%
\AgdaSymbol{⦄}\AgdaSpace{}%
\AgdaInductiveConstructor{num}%
\>[36]\AgdaSymbol{=}\AgdaSpace{}%
\AgdaDatatype{ℕ}\<%
\\
%
\>[8]\AgdaOperator{\AgdaField{⟦\AgdaUnderscore{}⟧ᵀ}}\AgdaSpace{}%
\AgdaSymbol{⦃}\AgdaSpace{}%
\AgdaFunction{CBNUniv}\AgdaSpace{}%
\AgdaSymbol{⦄}\AgdaSpace{}%
\AgdaSymbol{(}\AgdaInductiveConstructor{susp}\AgdaSpace{}%
\AgdaBound{t}\AgdaSymbol{)}%
\>[36]\AgdaSymbol{=}\AgdaSpace{}%
\AgdaDatatype{Free}\AgdaSpace{}%
\AgdaSymbol{(}\AgdaFunction{State}\AgdaSpace{}%
\AgdaOperator{\AgdaFunction{⊕}}\AgdaSpace{}%
\AgdaFunction{Nil}\AgdaSymbol{)}\AgdaSpace{}%
\AgdaOperator{\AgdaField{⟦}}\AgdaSpace{}%
\AgdaBound{t}\AgdaSpace{}%
\AgdaOperator{\AgdaField{⟧ᵀ}}\<%
\\
%
\\[\AgdaEmptyExtraSkip]%
%
\>[8]\AgdaFunction{iso-num}\AgdaSpace{}%
\AgdaSymbol{:}\AgdaSpace{}%
\AgdaDatatype{ℕ}\AgdaSpace{}%
\AgdaOperator{\AgdaFunction{↔}}\AgdaSpace{}%
\AgdaOperator{\AgdaField{⟦}}\AgdaSpace{}%
\AgdaInductiveConstructor{num}\AgdaSpace{}%
\AgdaOperator{\AgdaField{⟧ᵀ}}\<%
\\
%
\>[8]\AgdaFunction{iso-num}\AgdaSpace{}%
\AgdaSymbol{=}\AgdaSpace{}%
\AgdaFunction{↔-id}\AgdaSpace{}%
\AgdaSymbol{\AgdaUnderscore{}}\<%
\\
%
\\[\AgdaEmptyExtraSkip]%
%
\>[8]\AgdaFunction{iso-fun}%
\>[1024I]\AgdaSymbol{:}\AgdaSpace{}%
\AgdaSymbol{\{}\AgdaBound{t₁}\AgdaSpace{}%
\AgdaBound{t₂}\AgdaSpace{}%
\AgdaSymbol{:}\AgdaSpace{}%
\AgdaDatatype{LamType}\AgdaSymbol{\}}\<%
\\
\>[.][@{}l@{}]\<[1024I]%
\>[16]\AgdaSymbol{→}\AgdaSpace{}%
\AgdaSymbol{(}\AgdaOperator{\AgdaField{⟦}}\AgdaSpace{}%
\AgdaBound{t₁}\AgdaSpace{}%
\AgdaOperator{\AgdaField{⟧ᵀ}}\AgdaSpace{}%
\AgdaSymbol{→}\AgdaSpace{}%
\AgdaDatatype{Free}\AgdaSpace{}%
\AgdaSymbol{(}\AgdaFunction{State}\AgdaSpace{}%
\AgdaOperator{\AgdaFunction{⊕}}\AgdaSpace{}%
\AgdaFunction{Nil}\AgdaSymbol{)}\AgdaSpace{}%
\AgdaOperator{\AgdaField{⟦}}\AgdaSpace{}%
\AgdaBound{t₂}\AgdaSpace{}%
\AgdaOperator{\AgdaField{⟧ᵀ}}\AgdaSymbol{)}\AgdaSpace{}%
\AgdaOperator{\AgdaFunction{↔}}\AgdaSpace{}%
\AgdaOperator{\AgdaField{⟦}}\AgdaSpace{}%
\AgdaBound{t₁}\AgdaSpace{}%
\AgdaOperator{\AgdaInductiveConstructor{⟶}}\AgdaSpace{}%
\AgdaBound{t₂}\AgdaSpace{}%
\AgdaOperator{\AgdaField{⟧ᵀ}}\<%
\\
%
\>[8]\AgdaFunction{iso-fun}\AgdaSpace{}%
\AgdaSymbol{=}\AgdaSpace{}%
\AgdaFunction{↔-id}\AgdaSpace{}%
\AgdaSymbol{\AgdaUnderscore{}}\<%
\\
%
\\[\AgdaEmptyExtraSkip]%
%
\>[8]\AgdaFunction{iso-susp}%
\>[1049I]\AgdaSymbol{:}\AgdaSpace{}%
\AgdaSymbol{\{}\AgdaBound{t}\AgdaSpace{}%
\AgdaSymbol{:}\AgdaSpace{}%
\AgdaField{Type}\AgdaSymbol{\}}\<%
\\
\>[.][@{}l@{}]\<[1049I]%
\>[17]\AgdaSymbol{→}\AgdaSpace{}%
\AgdaDatatype{Free}\AgdaSpace{}%
\AgdaSymbol{(}\AgdaFunction{State}\AgdaSpace{}%
\AgdaOperator{\AgdaFunction{⊕}}\AgdaSpace{}%
\AgdaFunction{Nil}\AgdaSymbol{)}\AgdaSpace{}%
\AgdaOperator{\AgdaField{⟦}}\AgdaSpace{}%
\AgdaBound{t}\AgdaSpace{}%
\AgdaOperator{\AgdaField{⟧ᵀ}}\AgdaSpace{}%
\AgdaOperator{\AgdaFunction{↔}}\AgdaSpace{}%
\AgdaOperator{\AgdaField{⟦}}\AgdaSpace{}%
\AgdaInductiveConstructor{susp}\AgdaSpace{}%
\AgdaBound{t}\AgdaSpace{}%
\AgdaOperator{\AgdaField{⟧ᵀ}}\<%
\\
%
\>[8]\AgdaFunction{iso-susp}\AgdaSpace{}%
\AgdaSymbol{=}\AgdaSpace{}%
\AgdaFunction{↔-id}\AgdaSpace{}%
\AgdaSymbol{\AgdaUnderscore{}}\<%
\\
%
\\[\AgdaEmptyExtraSkip]%
%
\>[8]\AgdaFunction{LamCBNUniv}\AgdaSpace{}%
\AgdaSymbol{:}\AgdaSpace{}%
\AgdaRecord{LamUniv}\<%
\\
%
\>[8]\AgdaField{u}\AgdaSpace{}%
\AgdaSymbol{⦃}\AgdaSpace{}%
\AgdaFunction{LamCBNUniv}\AgdaSpace{}%
\AgdaSymbol{⦄}\AgdaSpace{}%
\AgdaSymbol{=}\AgdaSpace{}%
\AgdaFunction{CBNUniv}\<%
\\
%
\>[8]\AgdaOperator{\AgdaField{\AgdaUnderscore{}↣\AgdaUnderscore{}}}\AgdaSpace{}%
\AgdaSymbol{⦃}\AgdaSpace{}%
\AgdaFunction{LamCBNUniv}\AgdaSpace{}%
\AgdaSymbol{⦄}\AgdaSpace{}%
\AgdaSymbol{=}\AgdaSpace{}%
\AgdaOperator{\AgdaInductiveConstructor{\AgdaUnderscore{}⟶\AgdaUnderscore{}}}\<%
\\
%
\>[8]\AgdaField{c}\AgdaSpace{}%
\AgdaSymbol{⦃}\AgdaSpace{}%
\AgdaFunction{LamCBNUniv}\AgdaSpace{}%
\AgdaSymbol{⦄}\AgdaSpace{}%
\AgdaSymbol{=}\AgdaSpace{}%
\AgdaInductiveConstructor{susp}\<%
\\
%
\\[\AgdaEmptyExtraSkip]%
%
\>[6]\AgdaKeyword{module}\AgdaSpace{}%
\AgdaModule{\AgdaUnderscore{}}\AgdaSpace{}%
\AgdaKeyword{where}\<%
\\
\>[6][@{}l@{\AgdaIndent{0}}]%
\>[8]\AgdaKeyword{private}\AgdaSpace{}%
\AgdaKeyword{instance}\AgdaSpace{}%
\AgdaFunction{y₀}\AgdaSpace{}%
\AgdaSymbol{:}\AgdaSpace{}%
\AgdaFunction{State}\AgdaSpace{}%
\AgdaOperator{\AgdaFunction{≲}}\AgdaSpace{}%
\AgdaSymbol{(}\AgdaFunction{State}\AgdaSpace{}%
\AgdaOperator{\AgdaFunction{⊕}}\AgdaSpace{}%
\AgdaFunction{Nil}\AgdaSymbol{)}\<%
\\
%
\>[8]\AgdaFunction{y₀}\AgdaSpace{}%
\AgdaSymbol{=}\AgdaSpace{}%
\AgdaFunction{≲-left}\<%
\end{code}
%
Running the example program \af{ex} from above now produces \an{5} as result, since the state increment expression in the argument of \af{‵app} is thunked and run twice during the evaluation of the called function.
%
\begin{code}%
%
\>[8]\AgdaFunction{elab-cbn}\AgdaSpace{}%
\AgdaSymbol{:}\AgdaSpace{}%
\AgdaFunction{Elaboration}\AgdaSpace{}%
\AgdaSymbol{(}\AgdaFunction{Lam}\AgdaSpace{}%
\AgdaOperator{\AgdaFunction{∔}}\AgdaSpace{}%
\AgdaFunction{Lift}\AgdaSpace{}%
\AgdaFunction{State}\AgdaSpace{}%
\AgdaOperator{\AgdaFunction{∔}}\AgdaSpace{}%
\AgdaFunction{Lift}\AgdaSpace{}%
\AgdaFunction{Nil}\AgdaSymbol{)}\AgdaSpace{}%
\AgdaSymbol{(}\AgdaFunction{State}\AgdaSpace{}%
\AgdaOperator{\AgdaFunction{⊕}}\AgdaSpace{}%
\AgdaFunction{Nil}\AgdaSymbol{)}\<%
\\
%
\>[8]\AgdaFunction{elab-cbn}\AgdaSpace{}%
\AgdaSymbol{=}\AgdaSpace{}%
\AgdaFunction{eLamCBN}\AgdaSpace{}%
\AgdaOperator{\AgdaFunction{⋎}}\AgdaSpace{}%
\AgdaFunction{eLift}\AgdaSpace{}%
\AgdaOperator{\AgdaFunction{⋎}}\AgdaSpace{}%
\AgdaFunction{eNil}\<%
\\
%
\\[\AgdaEmptyExtraSkip]%
%
\>[8]\AgdaFunction{test-ex-cbn}\AgdaSpace{}%
\AgdaSymbol{:}\AgdaSpace{}%
\AgdaFunction{un}\AgdaSpace{}%
\AgdaSymbol{((}\AgdaOperator{\AgdaFunction{given}}\AgdaSpace{}%
\AgdaFunction{hSt}\AgdaSpace{}%
\AgdaOperator{\AgdaFunction{handle}}\AgdaSpace{}%
\AgdaSymbol{(}\AgdaFunction{elaborate}\AgdaSpace{}%
\AgdaFunction{elab-cbn}\AgdaSpace{}%
\AgdaFunction{ex}\AgdaSymbol{))}\AgdaSpace{}%
\AgdaNumber{0}\AgdaSymbol{)}\AgdaSpace{}%
\AgdaOperator{\AgdaDatatype{≡}}\AgdaSpace{}%
\AgdaSymbol{(}\AgdaNumber{5}\AgdaSpace{}%
\AgdaOperator{\AgdaInductiveConstructor{,}}\AgdaSpace{}%
\AgdaNumber{3}\AgdaSymbol{)}\<%
\\
%
\>[8]\AgdaFunction{test-ex-cbn}\AgdaSpace{}%
\AgdaSymbol{=}\AgdaSpace{}%
\AgdaInductiveConstructor{refl}\<%
\end{code}


\subsection{Optionally Transactional Exception Catching}
\label{sec:optionally-transactional}

A feature of scoped effect handlers~\citep{WuSH14,PirogSWJ18,YangPWBS22} is that changing the order of handlers makes it possible to obtain different semantics of \emph{effect interaction}.
A classical example of effect interaction is the interaction between state and exception catching that we briefly considered at the end of \cref{sec:hefty-algebras} in connection with this \ad{transact} program:
%
\begin{code}[hide]%
\>[0][@{}l@{\AgdaIndent{6}}]%
\>[2]\AgdaKeyword{module}\AgdaSpace{}%
\AgdaModule{CCModule}\AgdaSpace{}%
\AgdaKeyword{where}\<%
\\
\>[2][@{}l@{\AgdaIndent{0}}]%
\>[4]\AgdaKeyword{open}\AgdaSpace{}%
\AgdaKeyword{import}\AgdaSpace{}%
\AgdaModule{Data.Nat}\AgdaSpace{}%
\AgdaKeyword{using}\AgdaSpace{}%
\AgdaSymbol{(}\AgdaDatatype{ℕ}\AgdaSymbol{)}\<%
\\
%
\>[4]\AgdaKeyword{open}\AgdaSpace{}%
\AgdaModule{FreeModule}\AgdaSpace{}%
\AgdaKeyword{hiding}\AgdaSpace{}%
\AgdaSymbol{(}\AgdaOperator{\AgdaFunction{\AgdaUnderscore{}𝓑\AgdaUnderscore{}}}\AgdaSymbol{;}\AgdaSpace{}%
\AgdaOperator{\AgdaFunction{\AgdaUnderscore{}>>\AgdaUnderscore{}}}\AgdaSymbol{)}\<%
\\
%
\>[4]\AgdaKeyword{open}\AgdaSpace{}%
\AgdaModule{Abbreviation}\<%
\\
%
\>[4]\AgdaKeyword{open}\AgdaSpace{}%
\AgdaModule{ElabModule}\<%
\\
%
\>[4]\AgdaKeyword{open}\AgdaSpace{}%
\AgdaModule{Algᴴ}\<%
\\
%
\>[4]\AgdaKeyword{open}\AgdaSpace{}%
\AgdaModule{Inverse}\AgdaSpace{}%
\AgdaSymbol{⦃}\AgdaSpace{}%
\AgdaSymbol{...}\AgdaSpace{}%
\AgdaSymbol{⦄}\<%
\\
%
\\[\AgdaEmptyExtraSkip]%
%
\>[4]\AgdaFunction{‵throwᴴ}%
\>[1147I]\AgdaSymbol{:}\AgdaSpace{}%
\AgdaSymbol{⦃}\AgdaSpace{}%
\AgdaBound{w}\AgdaSpace{}%
\AgdaSymbol{:}\AgdaSpace{}%
\AgdaFunction{Lift}\AgdaSpace{}%
\AgdaFunction{Throw}\AgdaSpace{}%
\AgdaOperator{\AgdaFunction{≲ᴴ}}\AgdaSpace{}%
\AgdaGeneralizable{H}\AgdaSpace{}%
\AgdaSymbol{⦄}\<%
\\
\>[1147I][@{}l@{\AgdaIndent{0}}]%
\>[13]\AgdaSymbol{→}\AgdaSpace{}%
\AgdaDatatype{Hefty}\AgdaSpace{}%
\AgdaGeneralizable{H}\AgdaSpace{}%
\AgdaGeneralizable{A}\<%
\\
%
\>[4]\AgdaFunction{‵throwᴴ}\AgdaSpace{}%
\AgdaSymbol{⦃}\AgdaSpace{}%
\AgdaBound{w}\AgdaSpace{}%
\AgdaSymbol{⦄}\AgdaSpace{}%
\AgdaSymbol{=}\AgdaSpace{}%
\AgdaSymbol{(}\AgdaOperator{\AgdaFunction{↑}}\AgdaSpace{}%
\AgdaInductiveConstructor{throw}\AgdaSymbol{)}\AgdaSpace{}%
\AgdaOperator{\AgdaFunction{𝓑}}\AgdaSpace{}%
\AgdaFunction{⊥-elim}\<%
\\
\>[4][@{}l@{\AgdaIndent{0}}]%
\>[6]\AgdaKeyword{where}\AgdaSpace{}%
\AgdaKeyword{open}\AgdaSpace{}%
\AgdaModule{HeftyModule}\AgdaSpace{}%
\AgdaKeyword{using}\AgdaSpace{}%
\AgdaSymbol{(}\AgdaOperator{\AgdaFunction{\AgdaUnderscore{}𝓑\AgdaUnderscore{}}}\AgdaSymbol{)}\<%
\\
%
\\[\AgdaEmptyExtraSkip]%
%
\>[4]\AgdaKeyword{module}\AgdaSpace{}%
\AgdaModule{\AgdaUnderscore{}}\AgdaSpace{}%
\AgdaSymbol{⦃}\AgdaSpace{}%
\AgdaBound{u}\AgdaSpace{}%
\AgdaSymbol{:}\AgdaSpace{}%
\AgdaRecord{Univ}\AgdaSpace{}%
\AgdaSymbol{⦄}\AgdaSpace{}%
\AgdaSymbol{\{}\AgdaBound{unit}\AgdaSpace{}%
\AgdaSymbol{:}\AgdaSpace{}%
\AgdaField{Type}\AgdaSymbol{\}}\AgdaSpace{}%
\AgdaSymbol{⦃}\AgdaSpace{}%
\AgdaBound{iso}\AgdaSpace{}%
\AgdaSymbol{:}\AgdaSpace{}%
\AgdaOperator{\AgdaField{⟦}}\AgdaSpace{}%
\AgdaBound{unit}\AgdaSpace{}%
\AgdaOperator{\AgdaField{⟧ᵀ}}\AgdaSpace{}%
\AgdaOperator{\AgdaFunction{↔}}\AgdaSpace{}%
\AgdaRecord{⊤}\AgdaSpace{}%
\AgdaSymbol{⦄}\AgdaSpace{}%
\AgdaKeyword{where}\<%
\\
\>[4][@{}l@{\AgdaIndent{0}}]%
\>[6]\AgdaKeyword{open}\AgdaSpace{}%
\AgdaModule{HeftyModule}\AgdaSpace{}%
\AgdaKeyword{using}\AgdaSpace{}%
\AgdaSymbol{(}\AgdaOperator{\AgdaFunction{\AgdaUnderscore{}𝓑\AgdaUnderscore{}}}\AgdaSymbol{;}\AgdaSpace{}%
\AgdaOperator{\AgdaFunction{\AgdaUnderscore{}>>\AgdaUnderscore{}}}\AgdaSymbol{)}\<%
\end{code}    
\begin{code}%
%
\>[6]\AgdaFunction{transact⅋}%
\>[1194I]\AgdaSymbol{:}\AgdaSpace{}%
\AgdaSymbol{⦃}\AgdaSpace{}%
\AgdaBound{wₛ}\AgdaSpace{}%
\AgdaSymbol{:}\AgdaSpace{}%
\AgdaFunction{Lift}\AgdaSpace{}%
\AgdaFunction{State}\AgdaSpace{}%
\AgdaOperator{\AgdaFunction{≲ᴴ}}\AgdaSpace{}%
\AgdaGeneralizable{H}\AgdaSpace{}%
\AgdaSymbol{⦄}\AgdaSpace{}%
\AgdaSymbol{⦃}\AgdaSpace{}%
\AgdaBound{wₜ}\AgdaSpace{}%
\AgdaSymbol{:}\AgdaSpace{}%
\AgdaFunction{Lift}\AgdaSpace{}%
\AgdaFunction{Throw}\AgdaSpace{}%
\AgdaOperator{\AgdaFunction{≲ᴴ}}\AgdaSpace{}%
\AgdaGeneralizable{H}\AgdaSpace{}%
\AgdaSymbol{⦄}\AgdaSpace{}%
\AgdaSymbol{⦃}\AgdaSpace{}%
\AgdaBound{w}%
\>[73]\AgdaSymbol{:}\AgdaSpace{}%
\AgdaFunction{Catch}\AgdaSpace{}%
\AgdaOperator{\AgdaFunction{≲ᴴ}}\AgdaSpace{}%
\AgdaGeneralizable{H}\AgdaSpace{}%
\AgdaSymbol{⦄}\<%
\\
\>[.][@{}l@{}]\<[1194I]%
\>[16]\AgdaSymbol{→}\AgdaSpace{}%
\AgdaDatatype{Hefty}\AgdaSpace{}%
\AgdaGeneralizable{H}\AgdaSpace{}%
\AgdaDatatype{ℕ}\<%
\\
%
\>[6]\AgdaFunction{transact⅋}\AgdaSpace{}%
\AgdaSymbol{=}\AgdaSpace{}%
\AgdaKeyword{do}\<%
\\
\>[6][@{}l@{\AgdaIndent{0}}]%
\>[8]\AgdaOperator{\AgdaFunction{↑}}\AgdaSpace{}%
\AgdaInductiveConstructor{put}\AgdaSpace{}%
\AgdaNumber{1}\<%
\\
%
\>[8]\AgdaFunction{‵catch}\AgdaSpace{}%
\AgdaSymbol{(}\AgdaKeyword{do}\AgdaSpace{}%
\AgdaOperator{\AgdaFunction{↑}}\AgdaSpace{}%
\AgdaInductiveConstructor{put}\AgdaSpace{}%
\AgdaNumber{2}\AgdaSymbol{;}\AgdaSpace{}%
\AgdaSymbol{(}\AgdaOperator{\AgdaFunction{↑}}\AgdaSpace{}%
\AgdaInductiveConstructor{throw}\AgdaSymbol{)}\AgdaSpace{}%
\AgdaOperator{\AgdaFunction{𝓑}}\AgdaSpace{}%
\AgdaFunction{⊥-elim}\AgdaSymbol{)}\AgdaSpace{}%
\AgdaSymbol{(}\AgdaInductiveConstructor{pure}\AgdaSpace{}%
\AgdaFunction{tt⅋}\AgdaSymbol{)}\<%
\\
%
\>[8]\AgdaOperator{\AgdaFunction{↑}}\AgdaSpace{}%
\AgdaInductiveConstructor{get}\<%
\end{code}
\begin{code}[hide]%
\>[6][@{}l@{\AgdaIndent{1}}]%
\>[7]\AgdaKeyword{where}\AgdaSpace{}%
\AgdaFunction{tt⅋}\AgdaSpace{}%
\AgdaSymbol{=}\AgdaSpace{}%
\AgdaField{from}\AgdaSpace{}%
\AgdaInductiveConstructor{tt}\<%
\end{code}
%
% The program first sets the state to \an{1}; then runs the ``try'' block of the \af{‵catch} operation, which sets the state to \an{2} and subsequently throws an exception.
% This causes the ``catch'' block of the \af{‵catch} operation to run, which does nothing.
% The last line of the program inspects the final state of the program.
% %
The state and exception catching effect can interact to give either of these two semantics:
\begin{enumerate}
\item \emph{Global} interpretation of state, where the \af{transact} program returns \an{2} since the \ac{put} operation in the ``try'' block causes the state to be updated globally.
\item \emph{Transactional} interpretation of state, where the \af{transact} program returns \an{1} since the state changes of the \ac{put} operation are \emph{rolled back} when the ``try'' block throws an exception.
\end{enumerate}
%
With monad transformers~\citep{cenciarelli1993syntactic,Liang1995monad} we can recover either of these semantics by permuting the order of monad transformers.
With scoped effect handlers we can also recover either by permuting the order of handlers.
However, the \ad{eCatch} elaboration in \cref{sec:hefty-algebras} always gives us a global interpretation of state.
In this section we demonstrate how we can recover a transactional interpretation of state by using a different elaboration of the \ac{catch} operation into an algebraically effectful program with the \ac{throw} operation and the off-the-shelf \emph{sub/jump} control effects due to \citet{thielecke1997phd,DBLP:conf/csl/FioreS14}.
The different elaboration is modular in the sense that we do not have to change the interface of the catch operation nor any programs written against the interface.

\subsubsection{Sub/Jump}
We recall how to define two operations, sub and jump, due to~\citet{thielecke1997phd,DBLP:conf/csl/FioreS14}.
We define these operations as algebraic effects following~\citet{SchrijversPWJ19}.
The algebraic effect signature of \ad{CC}~\ab{Ref} is given in \cref{fig:alg-eff-ccop}, and is summarized by the following smart constructors:
%
\begin{code}[hide]%
%
\>[4]\AgdaKeyword{data}\AgdaSpace{}%
\AgdaDatatype{CCOp}\AgdaSpace{}%
\AgdaSymbol{⦃}\AgdaSpace{}%
\AgdaBound{u}\AgdaSpace{}%
\AgdaSymbol{:}\AgdaSpace{}%
\AgdaRecord{Univ}\AgdaSpace{}%
\AgdaSymbol{⦄}\AgdaSpace{}%
\AgdaSymbol{(}\AgdaBound{Ref}\AgdaSpace{}%
\AgdaSymbol{:}\AgdaSpace{}%
\AgdaField{Type}\AgdaSpace{}%
\AgdaSymbol{→}\AgdaSpace{}%
\AgdaPrimitive{Set}\AgdaSymbol{)}\AgdaSpace{}%
\AgdaSymbol{:}\AgdaSpace{}%
\AgdaPrimitive{Set}\AgdaSpace{}%
\AgdaKeyword{where}\<%
\\
\>[4][@{}l@{\AgdaIndent{0}}]%
\>[6]\AgdaInductiveConstructor{sub}%
\>[12]\AgdaSymbol{:}\AgdaSpace{}%
\AgdaSymbol{\{}\AgdaBound{t}\AgdaSpace{}%
\AgdaSymbol{:}\AgdaSpace{}%
\AgdaField{Type}\AgdaSymbol{\}}%
\>[51]\AgdaSymbol{→}%
\>[54]\AgdaDatatype{CCOp}\AgdaSpace{}%
\AgdaBound{Ref}\<%
\\
%
\>[6]\AgdaInductiveConstructor{jump}%
\>[12]\AgdaSymbol{:}\AgdaSpace{}%
\AgdaSymbol{\{}\AgdaBound{t}\AgdaSpace{}%
\AgdaSymbol{:}\AgdaSpace{}%
\AgdaField{Type}\AgdaSymbol{\}}\AgdaSpace{}%
\AgdaSymbol{(}\AgdaBound{ref}\AgdaSpace{}%
\AgdaSymbol{:}\AgdaSpace{}%
\AgdaBound{Ref}\AgdaSpace{}%
\AgdaBound{t}\AgdaSymbol{)}\AgdaSpace{}%
\AgdaSymbol{(}\AgdaBound{x}\AgdaSpace{}%
\AgdaSymbol{:}\AgdaSpace{}%
\AgdaOperator{\AgdaField{⟦}}\AgdaSpace{}%
\AgdaBound{t}\AgdaSpace{}%
\AgdaOperator{\AgdaField{⟧ᵀ}}\AgdaSymbol{)}\AgdaSpace{}%
\AgdaSymbol{→}%
\>[55]\AgdaDatatype{CCOp}\AgdaSpace{}%
\AgdaBound{Ref}\<%
\\
%
\\[\AgdaEmptyExtraSkip]%
%
\>[4]\AgdaFunction{CC}\AgdaSpace{}%
\AgdaSymbol{:}\AgdaSpace{}%
\AgdaSymbol{⦃}\AgdaSpace{}%
\AgdaBound{u}\AgdaSpace{}%
\AgdaSymbol{:}\AgdaSpace{}%
\AgdaRecord{Univ}\AgdaSpace{}%
\AgdaSymbol{⦄}\AgdaSpace{}%
\AgdaSymbol{(}\AgdaBound{Ref}\AgdaSpace{}%
\AgdaSymbol{:}\AgdaSpace{}%
\AgdaField{Type}\AgdaSpace{}%
\AgdaSymbol{→}\AgdaSpace{}%
\AgdaPrimitive{Set}\AgdaSymbol{)}\AgdaSpace{}%
\AgdaSymbol{→}\AgdaSpace{}%
\AgdaRecord{Effect}\<%
\\
%
\>[4]\AgdaField{Op}%
\>[8]\AgdaSymbol{(}\AgdaFunction{CC}\AgdaSpace{}%
\AgdaBound{Ref}\AgdaSymbol{)}\AgdaSpace{}%
\AgdaSymbol{=}\AgdaSpace{}%
\AgdaDatatype{CCOp}\AgdaSpace{}%
\AgdaBound{Ref}\<%
\\
%
\>[4]\AgdaField{Ret}\AgdaSpace{}%
\AgdaSymbol{(}\AgdaFunction{CC}\AgdaSpace{}%
\AgdaBound{Ref}\AgdaSymbol{)}\AgdaSpace{}%
\AgdaSymbol{(}\AgdaInductiveConstructor{sub}\AgdaSpace{}%
\AgdaSymbol{\{}\AgdaBound{t}\AgdaSymbol{\})}%
\>[31]\AgdaSymbol{=}\AgdaSpace{}%
\AgdaBound{Ref}\AgdaSpace{}%
\AgdaBound{t}\AgdaSpace{}%
\AgdaOperator{\AgdaDatatype{⊎}}\AgdaSpace{}%
\AgdaOperator{\AgdaField{⟦}}\AgdaSpace{}%
\AgdaBound{t}\AgdaSpace{}%
\AgdaOperator{\AgdaField{⟧ᵀ}}\<%
\\
%
\>[4]\AgdaField{Ret}\AgdaSpace{}%
\AgdaSymbol{(}\AgdaFunction{CC}\AgdaSpace{}%
\AgdaBound{Ref}\AgdaSymbol{)}\AgdaSpace{}%
\AgdaSymbol{(}\AgdaInductiveConstructor{jump}\AgdaSpace{}%
\AgdaBound{ref}\AgdaSpace{}%
\AgdaBound{x}\AgdaSymbol{)}%
\>[31]\AgdaSymbol{=}\AgdaSpace{}%
\AgdaFunction{⊥}\<%
\\
%
\\[\AgdaEmptyExtraSkip]%
%
\>[4]\AgdaKeyword{module}\AgdaSpace{}%
\AgdaModule{\AgdaUnderscore{}}\AgdaSpace{}%
\AgdaSymbol{⦃}\AgdaSpace{}%
\AgdaBound{u}\AgdaSpace{}%
\AgdaSymbol{:}\AgdaSpace{}%
\AgdaRecord{Univ}\AgdaSpace{}%
\AgdaSymbol{⦄}\AgdaSpace{}%
\AgdaSymbol{\{}\AgdaBound{Ref}\AgdaSpace{}%
\AgdaSymbol{:}\AgdaSpace{}%
\AgdaField{Type}\AgdaSpace{}%
\AgdaSymbol{→}\AgdaSpace{}%
\AgdaPrimitive{Set}\AgdaSymbol{\}}\AgdaSpace{}%
\AgdaSymbol{\{}\AgdaBound{t}\AgdaSpace{}%
\AgdaSymbol{:}\AgdaSpace{}%
\AgdaField{Type}\AgdaSymbol{\}}\AgdaSpace{}%
\AgdaKeyword{where}\<%
\end{code}
\begin{code}%
\>[4][@{}l@{\AgdaIndent{1}}]%
\>[6]\AgdaFunction{‵sub}%
\>[13]\AgdaSymbol{:}\AgdaSpace{}%
\AgdaSymbol{⦃}\AgdaSpace{}%
\AgdaBound{w}\AgdaSpace{}%
\AgdaSymbol{:}\AgdaSpace{}%
\AgdaFunction{CC}\AgdaSpace{}%
\AgdaBound{Ref}\AgdaSpace{}%
\AgdaOperator{\AgdaFunction{≲}}\AgdaSpace{}%
\AgdaGeneralizable{Δ}\AgdaSpace{}%
\AgdaSymbol{⦄}\AgdaSpace{}%
\AgdaSymbol{(}\AgdaBound{b}\AgdaSpace{}%
\AgdaSymbol{:}\AgdaSpace{}%
\AgdaBound{Ref}\AgdaSpace{}%
\AgdaBound{t}\AgdaSpace{}%
\AgdaSymbol{→}\AgdaSpace{}%
\AgdaDatatype{Free}\AgdaSpace{}%
\AgdaGeneralizable{Δ}\AgdaSpace{}%
\AgdaGeneralizable{A}\AgdaSymbol{)}\AgdaSpace{}%
\AgdaSymbol{(}\AgdaBound{k}\AgdaSpace{}%
\AgdaSymbol{:}\AgdaSpace{}%
\AgdaOperator{\AgdaField{⟦}}\AgdaSpace{}%
\AgdaBound{t}\AgdaSpace{}%
\AgdaOperator{\AgdaField{⟧ᵀ}}\AgdaSpace{}%
\AgdaSymbol{→}\AgdaSpace{}%
\AgdaDatatype{Free}\AgdaSpace{}%
\AgdaGeneralizable{Δ}\AgdaSpace{}%
\AgdaGeneralizable{A}\AgdaSymbol{)}%
\>[82]\AgdaSymbol{→}\AgdaSpace{}%
\AgdaDatatype{Free}\AgdaSpace{}%
\AgdaGeneralizable{Δ}\AgdaSpace{}%
\AgdaGeneralizable{A}\<%
\\
%
\>[6]\AgdaFunction{‵jump}%
\>[13]\AgdaSymbol{:}\AgdaSpace{}%
\AgdaSymbol{⦃}\AgdaSpace{}%
\AgdaBound{w}\AgdaSpace{}%
\AgdaSymbol{:}\AgdaSpace{}%
\AgdaFunction{CC}\AgdaSpace{}%
\AgdaBound{Ref}\AgdaSpace{}%
\AgdaOperator{\AgdaFunction{≲}}\AgdaSpace{}%
\AgdaGeneralizable{Δ}\AgdaSpace{}%
\AgdaSymbol{⦄}\AgdaSpace{}%
\AgdaSymbol{(}\AgdaBound{ref}\AgdaSpace{}%
\AgdaSymbol{:}\AgdaSpace{}%
\AgdaBound{Ref}\AgdaSpace{}%
\AgdaBound{t}\AgdaSymbol{)}\AgdaSpace{}%
\AgdaSymbol{(}\AgdaBound{x}\AgdaSpace{}%
\AgdaSymbol{:}\AgdaSpace{}%
\AgdaOperator{\AgdaField{⟦}}\AgdaSpace{}%
\AgdaBound{t}\AgdaSpace{}%
\AgdaOperator{\AgdaField{⟧ᵀ}}\AgdaSymbol{)}%
\>[84]\AgdaSymbol{→}\AgdaSpace{}%
\AgdaDatatype{Free}\AgdaSpace{}%
\AgdaGeneralizable{Δ}\AgdaSpace{}%
\AgdaGeneralizable{B}\<%
\end{code}
\begin{code}[hide]%
%
\>[6]\AgdaFunction{‵sub}\AgdaSpace{}%
\AgdaSymbol{⦃}\AgdaSpace{}%
\AgdaBound{w}\AgdaSpace{}%
\AgdaSymbol{⦄}\AgdaSpace{}%
\AgdaBound{b}\AgdaSpace{}%
\AgdaBound{k}\AgdaSpace{}%
\AgdaSymbol{=}\AgdaSpace{}%
\AgdaInductiveConstructor{impure}\<%
\\
\>[6][@{}l@{\AgdaIndent{0}}]%
\>[8]\AgdaSymbol{(}\AgdaFunction{inj}\AgdaSpace{}%
\AgdaSymbol{⦃}\AgdaSpace{}%
\AgdaBound{w}\AgdaSpace{}%
\AgdaSymbol{⦄}\AgdaSpace{}%
\AgdaSymbol{(}\AgdaInductiveConstructor{sub}\AgdaSpace{}%
\AgdaOperator{\AgdaInductiveConstructor{,}}\AgdaSpace{}%
\AgdaOperator{\AgdaFunction{[}}\AgdaSpace{}%
\AgdaBound{b}\AgdaSpace{}%
\AgdaOperator{\AgdaFunction{,}}\AgdaSpace{}%
\AgdaBound{k}\AgdaSpace{}%
\AgdaOperator{\AgdaFunction{]}}\AgdaSymbol{))}\<%
\\
%
\>[6]\AgdaFunction{‵jump}\AgdaSpace{}%
\AgdaSymbol{⦃}\AgdaSpace{}%
\AgdaBound{w}\AgdaSpace{}%
\AgdaSymbol{⦄}\AgdaSpace{}%
\AgdaBound{ref}\AgdaSpace{}%
\AgdaBound{x}\AgdaSpace{}%
\AgdaSymbol{=}\AgdaSpace{}%
\AgdaInductiveConstructor{impure}\<%
\\
\>[6][@{}l@{\AgdaIndent{0}}]%
\>[8]\AgdaSymbol{(}\AgdaFunction{inj}\AgdaSpace{}%
\AgdaSymbol{⦃}\AgdaSpace{}%
\AgdaBound{w}\AgdaSpace{}%
\AgdaSymbol{⦄}\AgdaSpace{}%
\AgdaSymbol{(}\AgdaInductiveConstructor{jump}\AgdaSpace{}%
\AgdaBound{ref}\AgdaSpace{}%
\AgdaBound{x}\AgdaSpace{}%
\AgdaOperator{\AgdaInductiveConstructor{,}}\AgdaSpace{}%
\AgdaSymbol{λ}\AgdaSpace{}%
\AgdaSymbol{()))}\<%
\end{code}
%
An operation \af{‵sub}~\ab{f}~\ab{g} gives a computation \ab{f} access to the continuation \ab{g} via a reference value \ab{Ref}~\ab{t} which represents a continuation expecting a value of type \aF{⟦}~\ab{t}~\aF{⟧ᵀ}.
The \af{‵jump} operation invokes such continuations.

\begin{figure}[t]
\begin{code}%
%
\>[4]\AgdaKeyword{data}\AgdaSpace{}%
\AgdaDatatype{CCOp⅋}\AgdaSpace{}%
\AgdaSymbol{⦃}\AgdaSpace{}%
\AgdaBound{u}\AgdaSpace{}%
\AgdaSymbol{:}\AgdaSpace{}%
\AgdaRecord{Univ}\AgdaSpace{}%
\AgdaSymbol{⦄}\AgdaSpace{}%
\AgdaSymbol{(}\AgdaBound{Ref}\AgdaSpace{}%
\AgdaSymbol{:}\AgdaSpace{}%
\AgdaField{Type}\AgdaSpace{}%
\AgdaSymbol{→}\AgdaSpace{}%
\AgdaPrimitive{Set}\AgdaSymbol{)}\AgdaSpace{}%
\AgdaSymbol{:}\AgdaSpace{}%
\AgdaPrimitive{Set}\AgdaSpace{}%
\AgdaKeyword{where}\<%
\\
\>[4][@{}l@{\AgdaIndent{0}}]%
\>[6]\AgdaInductiveConstructor{sub}%
\>[12]\AgdaSymbol{:}\AgdaSpace{}%
\AgdaSymbol{\{}\AgdaBound{t}\AgdaSpace{}%
\AgdaSymbol{:}\AgdaSpace{}%
\AgdaField{Type}\AgdaSymbol{\}}%
\>[51]\AgdaSymbol{→}%
\>[54]\AgdaDatatype{CCOp⅋}\AgdaSpace{}%
\AgdaBound{Ref}\<%
\\
%
\>[6]\AgdaInductiveConstructor{jump}%
\>[12]\AgdaSymbol{:}\AgdaSpace{}%
\AgdaSymbol{\{}\AgdaBound{t}\AgdaSpace{}%
\AgdaSymbol{:}\AgdaSpace{}%
\AgdaField{Type}\AgdaSymbol{\}}\AgdaSpace{}%
\AgdaSymbol{(}\AgdaBound{ref}\AgdaSpace{}%
\AgdaSymbol{:}\AgdaSpace{}%
\AgdaBound{Ref}\AgdaSpace{}%
\AgdaBound{t}\AgdaSymbol{)}\AgdaSpace{}%
\AgdaSymbol{(}\AgdaBound{x}\AgdaSpace{}%
\AgdaSymbol{:}\AgdaSpace{}%
\AgdaOperator{\AgdaField{⟦}}\AgdaSpace{}%
\AgdaBound{t}\AgdaSpace{}%
\AgdaOperator{\AgdaField{⟧ᵀ}}\AgdaSymbol{)}\AgdaSpace{}%
\AgdaSymbol{→}%
\>[55]\AgdaDatatype{CCOp⅋}\AgdaSpace{}%
\AgdaBound{Ref}\<%
\\
%
\\[\AgdaEmptyExtraSkip]%
%
\>[4]\AgdaFunction{CC⅋}\AgdaSpace{}%
\AgdaSymbol{:}\AgdaSpace{}%
\AgdaSymbol{⦃}\AgdaSpace{}%
\AgdaBound{u}\AgdaSpace{}%
\AgdaSymbol{:}\AgdaSpace{}%
\AgdaRecord{Univ}\AgdaSpace{}%
\AgdaSymbol{⦄}\AgdaSpace{}%
\AgdaSymbol{(}\AgdaBound{Ref}\AgdaSpace{}%
\AgdaSymbol{:}\AgdaSpace{}%
\AgdaField{Type}\AgdaSpace{}%
\AgdaSymbol{→}\AgdaSpace{}%
\AgdaPrimitive{Set}\AgdaSymbol{)}\AgdaSpace{}%
\AgdaSymbol{→}\AgdaSpace{}%
\AgdaRecord{Effect}\<%
\\
%
\>[4]\AgdaField{Op}%
\>[8]\AgdaSymbol{(}\AgdaFunction{CC⅋}\AgdaSpace{}%
\AgdaBound{Ref}\AgdaSymbol{)}\AgdaSpace{}%
\AgdaSymbol{=}\AgdaSpace{}%
\AgdaDatatype{CCOp⅋}\AgdaSpace{}%
\AgdaBound{Ref}\<%
\\
%
\>[4]\AgdaField{Ret}\AgdaSpace{}%
\AgdaSymbol{(}\AgdaFunction{CC⅋}\AgdaSpace{}%
\AgdaBound{Ref}\AgdaSymbol{)}\AgdaSpace{}%
\AgdaSymbol{(}\AgdaInductiveConstructor{sub}\AgdaSpace{}%
\AgdaSymbol{\{}\AgdaBound{t}\AgdaSymbol{\})}%
\>[32]\AgdaSymbol{=}\AgdaSpace{}%
\AgdaBound{Ref}\AgdaSpace{}%
\AgdaBound{t}\AgdaSpace{}%
\AgdaOperator{\AgdaDatatype{⊎}}\AgdaSpace{}%
\AgdaOperator{\AgdaField{⟦}}\AgdaSpace{}%
\AgdaBound{t}\AgdaSpace{}%
\AgdaOperator{\AgdaField{⟧ᵀ}}\<%
\\
%
\>[4]\AgdaField{Ret}\AgdaSpace{}%
\AgdaSymbol{(}\AgdaFunction{CC⅋}\AgdaSpace{}%
\AgdaBound{Ref}\AgdaSymbol{)}\AgdaSpace{}%
\AgdaSymbol{(}\AgdaInductiveConstructor{jump}\AgdaSpace{}%
\AgdaBound{ref}\AgdaSpace{}%
\AgdaBound{x}\AgdaSymbol{)}%
\>[32]\AgdaSymbol{=}\AgdaSpace{}%
\AgdaFunction{⊥}\<%
\end{code}
\caption{Effect signature of the sub/jump effect}
\label{fig:alg-eff-ccop}
\end{figure}


The operations and their handler (abbreviated to \af{h}) satisfy the following laws:
\begin{align*}
  \af{h}~\as{(}\af{‵sub}~\as{(λ~\_~→}~\ab{p}\as{)}~\ab{k}\as{)}
  &~\ad{≡}~\af{h}~\ab{p}
  \\
  \af{h}~\as{(}\af{‵sub}~\as{(λ}~\ab{r}~\as{→}~\ab{m}~\af{𝓑}~\af{‵jump}~\ab{r}\as{)}~\ab{k}\as{)}
   &~\ad{≡}~\af{h}~\as{(}\ab{m}~\af{𝓑}~\ab{k}\as{)}
  \\
  \af{h}~\as{(}\af{‵sub}~\ab{p}~\as{(}\af{‵jump}~\ab{r′}\as{))}
  &~\ad{≡}~\af{h}~\as{(}\ab{p}~\ab{r′}\as{)}
  \\
  \af{h}~\as{(}\af{‵sub}~\ab{p}~\ab{q}~\af{𝓑}~\ab{k}\as{)}
 &~\ad{≡}~\af{h}~\as{(}\af{‵sub}~\as{(λ}~\ab{x}~\as{→}~\ab{p}~\ab{x}~\af{𝓑}~\ab{k}~\as{)}~\as{(λ}~\ab{x}~\as{→}~\ab{q}~\ab{x}~\af{𝓑}~\ab{k}\as{))}
\end{align*}
The last law asserts that \af{‵sub} and \af{‵jump} are \emph{algebraic} operations, since their computational sub-terms behave as continuations.
Thus, we encode \af{‵sub} and its handler as an algebraic effect.
%
\begin{code}[hide]%
%
\>[4]\AgdaKeyword{module}\AgdaSpace{}%
\AgdaModule{\AgdaUnderscore{}}\AgdaSpace{}%
\AgdaSymbol{⦃}\AgdaSpace{}%
\AgdaBound{u}\AgdaSpace{}%
\AgdaSymbol{:}\AgdaSpace{}%
\AgdaRecord{Univ}\AgdaSpace{}%
\AgdaSymbol{⦄}\AgdaSpace{}%
\AgdaKeyword{where}\<%
\end{code}
\begin{code}[hide]%
\>[4][@{}l@{\AgdaIndent{1}}]%
\>[6]\AgdaFunction{hCC}\AgdaSpace{}%
\AgdaSymbol{:}\AgdaSpace{}%
\AgdaOperator{\AgdaRecord{⟨}}\AgdaSpace{}%
\AgdaGeneralizable{A}\AgdaSpace{}%
\AgdaOperator{\AgdaRecord{!}}\AgdaSpace{}%
\AgdaSymbol{(}\AgdaFunction{CC}\AgdaSpace{}%
\AgdaSymbol{(λ}\AgdaSpace{}%
\AgdaBound{t}\AgdaSpace{}%
\AgdaSymbol{→}\AgdaSpace{}%
\AgdaOperator{\AgdaField{⟦}}\AgdaSpace{}%
\AgdaBound{t}\AgdaSpace{}%
\AgdaOperator{\AgdaField{⟧ᵀ}}\AgdaSpace{}%
\AgdaSymbol{→}\AgdaSpace{}%
\AgdaDatatype{Free}\AgdaSpace{}%
\AgdaGeneralizable{Δ′}\AgdaSpace{}%
\AgdaGeneralizable{A}\AgdaSymbol{))}\AgdaSpace{}%
\AgdaOperator{\AgdaRecord{⇒}}\AgdaSpace{}%
\AgdaRecord{⊤}\AgdaSpace{}%
\AgdaOperator{\AgdaRecord{⇒}}\AgdaSpace{}%
\AgdaGeneralizable{A}\AgdaSpace{}%
\AgdaOperator{\AgdaRecord{!}}\AgdaSpace{}%
\AgdaGeneralizable{Δ′}\AgdaSpace{}%
\AgdaOperator{\AgdaRecord{⟩}}\<%
\\
%
\>[6]\AgdaField{ret}%
\>[11]\AgdaFunction{hCC}\AgdaSpace{}%
\AgdaBound{a}\AgdaSpace{}%
\AgdaSymbol{\AgdaUnderscore{}}\AgdaSpace{}%
\AgdaSymbol{=}\AgdaSpace{}%
\AgdaInductiveConstructor{pure}\AgdaSpace{}%
\AgdaBound{a}\<%
\\
%
\>[6]\AgdaField{hdl}%
\>[11]\AgdaFunction{hCC}\AgdaSpace{}%
\AgdaSymbol{(}\AgdaInductiveConstructor{sub}%
\>[24]\AgdaOperator{\AgdaInductiveConstructor{,}}%
\>[29]\AgdaBound{k}\AgdaSymbol{)}\AgdaSpace{}%
\AgdaBound{p}\AgdaSpace{}%
\AgdaSymbol{=}\AgdaSpace{}%
\AgdaKeyword{let}\AgdaSpace{}%
\AgdaBound{c}\AgdaSpace{}%
\AgdaSymbol{=}\AgdaSpace{}%
\AgdaFunction{flip}\AgdaSpace{}%
\AgdaBound{k}\AgdaSpace{}%
\AgdaBound{p}\AgdaSpace{}%
\AgdaOperator{\AgdaFunction{∘}}\AgdaSpace{}%
\AgdaInductiveConstructor{inj₂}\<%
\\
\>[6][@{}l@{\AgdaIndent{0}}]%
\>[8]\AgdaKeyword{in}\AgdaSpace{}%
\AgdaBound{k}\AgdaSpace{}%
\AgdaSymbol{(}\AgdaInductiveConstructor{inj₁}\AgdaSpace{}%
\AgdaBound{c}\AgdaSymbol{)}\AgdaSpace{}%
\AgdaBound{p}\<%
\\
%
\>[6]\AgdaField{hdl}%
\>[11]\AgdaFunction{hCC}\AgdaSpace{}%
\AgdaSymbol{(}\AgdaInductiveConstructor{jump}\AgdaSpace{}%
\AgdaBound{ref}\AgdaSpace{}%
\AgdaBound{x}\AgdaSpace{}%
\AgdaOperator{\AgdaInductiveConstructor{,}}\AgdaSpace{}%
\AgdaBound{k}\AgdaSymbol{)}\AgdaSpace{}%
\AgdaSymbol{\AgdaUnderscore{}}\AgdaSpace{}%
\AgdaSymbol{=}\AgdaSpace{}%
\AgdaBound{ref}\AgdaSpace{}%
\AgdaBound{x}\<%
\end{code}
%
\begin{code}[hide]%
%
\>[4]\AgdaKeyword{private}\<%
\\
\>[4][@{}l@{\AgdaIndent{0}}]%
\>[6]\AgdaKeyword{open}\AgdaSpace{}%
\AgdaKeyword{import}\AgdaSpace{}%
\AgdaModule{Data.Nat}\AgdaSpace{}%
\AgdaKeyword{using}\AgdaSpace{}%
\AgdaSymbol{(}\AgdaDatatype{ℕ}\AgdaSymbol{)}\AgdaSpace{}%
\AgdaKeyword{renaming}\AgdaSpace{}%
\AgdaSymbol{(}\AgdaOperator{\AgdaPrimitive{\AgdaUnderscore{}+\AgdaUnderscore{}}}\AgdaSpace{}%
\AgdaSymbol{to}\AgdaSpace{}%
\AgdaOperator{\AgdaPrimitive{\AgdaUnderscore{}ℕ+\AgdaUnderscore{}}}\AgdaSymbol{)}\<%
\\
%
\>[6]\AgdaKeyword{open}\AgdaSpace{}%
\AgdaKeyword{import}\AgdaSpace{}%
\AgdaModule{Effect.Monad}\<%
\\
%
\\[\AgdaEmptyExtraSkip]%
%
\>[6]\AgdaKeyword{data}\AgdaSpace{}%
\AgdaDatatype{NumType}\AgdaSpace{}%
\AgdaSymbol{:}\AgdaSpace{}%
\AgdaPrimitive{Set}\AgdaSpace{}%
\AgdaKeyword{where}\<%
\\
\>[6][@{}l@{\AgdaIndent{0}}]%
\>[8]\AgdaInductiveConstructor{num}\AgdaSpace{}%
\AgdaSymbol{:}\AgdaSpace{}%
\AgdaDatatype{NumType}\<%
\\
%
\\[\AgdaEmptyExtraSkip]%
%
\>[6]\AgdaKeyword{instance}\<%
\\
\>[6][@{}l@{\AgdaIndent{0}}]%
\>[8]\AgdaFunction{NumUniv}\AgdaSpace{}%
\AgdaSymbol{:}\AgdaSpace{}%
\AgdaRecord{Univ}\<%
\\
%
\>[8]\AgdaField{Type}\AgdaSpace{}%
\AgdaSymbol{⦃}\AgdaSpace{}%
\AgdaFunction{NumUniv}\AgdaSpace{}%
\AgdaSymbol{⦄}%
\>[30]\AgdaSymbol{=}\AgdaSpace{}%
\AgdaDatatype{NumType}\<%
\\
%
\>[8]\AgdaOperator{\AgdaField{⟦\AgdaUnderscore{}⟧ᵀ}}%
\>[14]\AgdaSymbol{⦃}\AgdaSpace{}%
\AgdaFunction{NumUniv}\AgdaSpace{}%
\AgdaSymbol{⦄}\AgdaSpace{}%
\AgdaInductiveConstructor{num}%
\>[31]\AgdaSymbol{=}\AgdaSpace{}%
\AgdaDatatype{ℕ}\<%
\\
%
\\[\AgdaEmptyExtraSkip]%
%
\>[6]\AgdaFunction{Cont}\AgdaSpace{}%
\AgdaSymbol{:}\AgdaSpace{}%
\AgdaRecord{Effect}\AgdaSpace{}%
\AgdaSymbol{→}\AgdaSpace{}%
\AgdaPrimitive{Set}\AgdaSpace{}%
\AgdaSymbol{→}\AgdaSpace{}%
\AgdaDatatype{NumType}\AgdaSpace{}%
\AgdaSymbol{→}\AgdaSpace{}%
\AgdaPrimitive{Set}\<%
\\
%
\>[6]\AgdaFunction{Cont}\AgdaSpace{}%
\AgdaBound{Δ}\AgdaSpace{}%
\AgdaBound{A}\AgdaSpace{}%
\AgdaBound{t}\AgdaSpace{}%
\AgdaSymbol{=}\AgdaSpace{}%
\AgdaOperator{\AgdaField{⟦}}\AgdaSpace{}%
\AgdaBound{t}\AgdaSpace{}%
\AgdaOperator{\AgdaField{⟧ᵀ}}\AgdaSpace{}%
\AgdaSymbol{→}\AgdaSpace{}%
\AgdaDatatype{Free}\AgdaSpace{}%
\AgdaBound{Δ}\AgdaSpace{}%
\AgdaBound{A}\<%
\\
%
\\[\AgdaEmptyExtraSkip]%
%
\>[6]\AgdaKeyword{private}\AgdaSpace{}%
\AgdaKeyword{instance}\<%
\\
\>[6][@{}l@{\AgdaIndent{0}}]%
\>[8]\AgdaFunction{x₀}\AgdaSpace{}%
\AgdaSymbol{:}\AgdaSpace{}%
\AgdaFunction{CC}\AgdaSpace{}%
\AgdaSymbol{(}\AgdaFunction{Cont}\AgdaSpace{}%
\AgdaGeneralizable{Δ}\AgdaSpace{}%
\AgdaDatatype{ℕ}\AgdaSymbol{)}\AgdaSpace{}%
\AgdaOperator{\AgdaFunction{≲}}\AgdaSpace{}%
\AgdaSymbol{(}\AgdaFunction{CC}\AgdaSpace{}%
\AgdaSymbol{(}\AgdaFunction{Cont}\AgdaSpace{}%
\AgdaGeneralizable{Δ}\AgdaSpace{}%
\AgdaDatatype{ℕ}\AgdaSymbol{)}\AgdaSpace{}%
\AgdaOperator{\AgdaFunction{⊕}}\AgdaSpace{}%
\AgdaGeneralizable{Δ}\AgdaSymbol{)}\<%
\\
%
\>[8]\AgdaFunction{x₀}\AgdaSpace{}%
\AgdaSymbol{=}\AgdaSpace{}%
\AgdaFunction{≲-left}\<%
\\
%
\\[\AgdaEmptyExtraSkip]%
%
\>[6]\AgdaFunction{ex₀}\AgdaSpace{}%
\AgdaSymbol{:}\AgdaSpace{}%
\AgdaDatatype{Free}\AgdaSpace{}%
\AgdaSymbol{(}\AgdaFunction{CC}\AgdaSpace{}%
\AgdaSymbol{(}\AgdaFunction{Cont}\AgdaSpace{}%
\AgdaGeneralizable{Δ}\AgdaSpace{}%
\AgdaDatatype{ℕ}\AgdaSymbol{)}\AgdaSpace{}%
\AgdaOperator{\AgdaFunction{⊕}}\AgdaSpace{}%
\AgdaGeneralizable{Δ}\AgdaSymbol{)}\AgdaSpace{}%
\AgdaDatatype{ℕ}\<%
\\
%
\>[6]\AgdaFunction{ex₀}\AgdaSpace{}%
\AgdaSymbol{=}\AgdaSpace{}%
\AgdaKeyword{do}\<%
\\
\>[6][@{}l@{\AgdaIndent{0}}]%
\>[8]\AgdaFunction{‵sub}\AgdaSpace{}%
\AgdaSymbol{(λ}\AgdaSpace{}%
\AgdaBound{ref}\AgdaSpace{}%
\AgdaSymbol{→}\AgdaSpace{}%
\AgdaFunction{‵jump}\AgdaSpace{}%
\AgdaBound{ref}\AgdaSpace{}%
\AgdaNumber{41}\AgdaSymbol{)}\AgdaSpace{}%
\AgdaSymbol{(λ}\AgdaSpace{}%
\AgdaBound{x}\AgdaSpace{}%
\AgdaSymbol{→}\AgdaSpace{}%
\AgdaInductiveConstructor{pure}\AgdaSpace{}%
\AgdaSymbol{(}\AgdaBound{x}\AgdaSpace{}%
\AgdaOperator{\AgdaPrimitive{ℕ+}}\AgdaSpace{}%
\AgdaNumber{1}\AgdaSymbol{))}\<%
\\
%
\\[\AgdaEmptyExtraSkip]%
%
\>[6]\AgdaFunction{test₀}\AgdaSpace{}%
\AgdaSymbol{:}\AgdaSpace{}%
\AgdaFunction{un}\AgdaSpace{}%
\AgdaSymbol{((}\AgdaOperator{\AgdaFunction{given}}\AgdaSpace{}%
\AgdaFunction{hCC}\AgdaSpace{}%
\AgdaOperator{\AgdaFunction{handle}}\AgdaSpace{}%
\AgdaFunction{ex₀}\AgdaSymbol{)}\AgdaSpace{}%
\AgdaInductiveConstructor{tt}\AgdaSymbol{)}\AgdaSpace{}%
\AgdaOperator{\AgdaDatatype{≡}}\AgdaSpace{}%
\AgdaNumber{42}\<%
\\
%
\>[6]\AgdaFunction{test₀}\AgdaSpace{}%
\AgdaSymbol{=}\AgdaSpace{}%
\AgdaInductiveConstructor{refl}\<%
\\
%
\\[\AgdaEmptyExtraSkip]%
%
\>[6]\AgdaFunction{ex₁}\AgdaSpace{}%
\AgdaSymbol{:}\AgdaSpace{}%
\AgdaDatatype{Free}\AgdaSpace{}%
\AgdaSymbol{(}\AgdaFunction{CC}\AgdaSpace{}%
\AgdaSymbol{(}\AgdaFunction{Cont}\AgdaSpace{}%
\AgdaGeneralizable{Δ}\AgdaSpace{}%
\AgdaDatatype{ℕ}\AgdaSymbol{)}\AgdaSpace{}%
\AgdaOperator{\AgdaFunction{⊕}}\AgdaSpace{}%
\AgdaGeneralizable{Δ}\AgdaSymbol{)}\AgdaSpace{}%
\AgdaDatatype{ℕ}\<%
\\
%
\>[6]\AgdaFunction{ex₁}\AgdaSpace{}%
\AgdaSymbol{=}\AgdaSpace{}%
\AgdaKeyword{do}\<%
\\
\>[6][@{}l@{\AgdaIndent{0}}]%
\>[8]\AgdaFunction{‵sub}\AgdaSpace{}%
\AgdaSymbol{(λ}\AgdaSpace{}%
\AgdaBound{ref}\AgdaSpace{}%
\AgdaSymbol{→}\AgdaSpace{}%
\AgdaInductiveConstructor{pure}\AgdaSpace{}%
\AgdaNumber{41}\AgdaSymbol{)}\AgdaSpace{}%
\AgdaSymbol{(λ}\AgdaSpace{}%
\AgdaBound{x}\AgdaSpace{}%
\AgdaSymbol{→}\AgdaSpace{}%
\AgdaInductiveConstructor{pure}\AgdaSpace{}%
\AgdaSymbol{(}\AgdaBound{x}\AgdaSpace{}%
\AgdaOperator{\AgdaPrimitive{ℕ+}}\AgdaSpace{}%
\AgdaNumber{1}\AgdaSymbol{))}\<%
\\
%
\\[\AgdaEmptyExtraSkip]%
%
\>[6]\AgdaFunction{test₁}\AgdaSpace{}%
\AgdaSymbol{:}\AgdaSpace{}%
\AgdaFunction{un}\AgdaSpace{}%
\AgdaSymbol{((}\AgdaOperator{\AgdaFunction{given}}\AgdaSpace{}%
\AgdaFunction{hCC}\AgdaSpace{}%
\AgdaOperator{\AgdaFunction{handle}}\AgdaSpace{}%
\AgdaFunction{ex₁}\AgdaSymbol{)}\AgdaSpace{}%
\AgdaInductiveConstructor{tt}\AgdaSymbol{)}\AgdaSpace{}%
\AgdaOperator{\AgdaDatatype{≡}}\AgdaSpace{}%
\AgdaNumber{41}\<%
\\
%
\>[6]\AgdaFunction{test₁}\AgdaSpace{}%
\AgdaSymbol{=}\AgdaSpace{}%
\AgdaInductiveConstructor{refl}\<%
\end{code}


\subsubsection{Optionally Transactional Exception Catching}
\label{sec:optional-transactional}

By using the \af{‵sub} and \af{‵jump} operations in our elaboration of \ad{catch}, we get a semantics of exception catching whose interaction with state depends on the order that the state effect and sub/jump effect is handled.
%
\begin{code}[hide]%
%
\>[2]\AgdaKeyword{module}\AgdaSpace{}%
\AgdaModule{TransactionalCatch}\AgdaSpace{}%
\AgdaKeyword{where}\<%
\\
\>[2][@{}l@{\AgdaIndent{0}}]%
\>[4]\AgdaKeyword{open}\AgdaSpace{}%
\AgdaModule{CCModule}\<%
\\
%
\>[4]\AgdaKeyword{open}\AgdaSpace{}%
\AgdaModule{Abbreviation}\<%
\\
%
\\[\AgdaEmptyExtraSkip]%
%
\>[4]\AgdaKeyword{module}\AgdaSpace{}%
\AgdaModule{\AgdaUnderscore{}}%
\>[1657I]\AgdaSymbol{⦃}\AgdaSpace{}%
\AgdaBound{u}\AgdaSpace{}%
\AgdaSymbol{:}\AgdaSpace{}%
\AgdaRecord{Univ}\AgdaSpace{}%
\AgdaSymbol{⦄}\<%
\\
\>[.][@{}l@{}]\<[1657I]%
\>[13]\AgdaSymbol{\{}\AgdaBound{Ref}\AgdaSpace{}%
\AgdaSymbol{:}\AgdaSpace{}%
\AgdaField{Type}\AgdaSpace{}%
\AgdaSymbol{→}\AgdaSpace{}%
\AgdaPrimitive{Set}\AgdaSymbol{\}}\<%
\\
%
\>[13]\AgdaSymbol{\{}\AgdaBound{unit}\AgdaSpace{}%
\AgdaSymbol{:}\AgdaSpace{}%
\AgdaField{Type}\AgdaSymbol{\}}\<%
\\
%
\>[13]\AgdaSymbol{⦃}\AgdaSpace{}%
\AgdaBound{iso}\AgdaSpace{}%
\AgdaSymbol{:}\AgdaSpace{}%
\AgdaOperator{\AgdaField{⟦}}\AgdaSpace{}%
\AgdaBound{unit}\AgdaSpace{}%
\AgdaOperator{\AgdaField{⟧ᵀ}}\AgdaSpace{}%
\AgdaOperator{\AgdaFunction{↔}}\AgdaSpace{}%
\AgdaRecord{⊤}\AgdaSpace{}%
\AgdaSymbol{⦄}\<%
\\
%
\>[13]\AgdaKeyword{where}\<%
\\
\>[4][@{}l@{\AgdaIndent{0}}]%
\>[6]\AgdaKeyword{open}\AgdaSpace{}%
\AgdaModule{FreeModule}\AgdaSpace{}%
\AgdaKeyword{using}\AgdaSpace{}%
\AgdaSymbol{(}\AgdaOperator{\AgdaFunction{\AgdaUnderscore{}𝓑\AgdaUnderscore{}}}\AgdaSymbol{;}\AgdaSpace{}%
\AgdaOperator{\AgdaFunction{\AgdaUnderscore{}>>\AgdaUnderscore{}}}\AgdaSymbol{)}\<%
\\
%
\>[6]\AgdaKeyword{open}\AgdaSpace{}%
\AgdaModule{ElabModule}\<%
\\
\>[0]\AgdaComment{--\ \ \ \ \ \ open\ Elab}\<%
\\
\>[0][@{}l@{\AgdaIndent{0}}]%
\>[6]\AgdaKeyword{open}\AgdaSpace{}%
\AgdaModule{Inverse}\AgdaSpace{}%
\AgdaSymbol{⦃}\AgdaSpace{}%
\AgdaSymbol{...}\AgdaSpace{}%
\AgdaSymbol{⦄}\<%
\\
%
\\[\AgdaEmptyExtraSkip]%
%
\\[\AgdaEmptyExtraSkip]%
%
\>[6]\AgdaKeyword{module}\AgdaSpace{}%
\AgdaModule{\AgdaUnderscore{}}%
\>[16]\AgdaSymbol{⦃}\AgdaSpace{}%
\AgdaBound{w₁}\AgdaSpace{}%
\AgdaSymbol{:}\AgdaSpace{}%
\AgdaFunction{CC}\AgdaSpace{}%
\AgdaBound{Ref}\AgdaSpace{}%
\AgdaOperator{\AgdaFunction{≲}}\AgdaSpace{}%
\AgdaGeneralizable{Δ}\AgdaSpace{}%
\AgdaSymbol{⦄}\AgdaSpace{}%
\AgdaSymbol{⦃}\AgdaSpace{}%
\AgdaBound{w₂}\AgdaSpace{}%
\AgdaSymbol{:}\AgdaSpace{}%
\AgdaFunction{Throw}\AgdaSpace{}%
\AgdaOperator{\AgdaFunction{≲}}\AgdaSpace{}%
\AgdaGeneralizable{Δ}\AgdaSpace{}%
\AgdaSymbol{⦄}\AgdaSpace{}%
\AgdaKeyword{where}\<%
\\
\>[6][@{}l@{\AgdaIndent{0}}]%
\>[8]\AgdaFunction{eCatchOT}\AgdaSpace{}%
\AgdaSymbol{:}\AgdaSpace{}%
\AgdaFunction{Elaboration}\AgdaSpace{}%
\AgdaFunction{Catch}\AgdaSpace{}%
\AgdaBound{Δ}\<%
\\
%
\>[8]\AgdaField{alg}\AgdaSpace{}%
\AgdaFunction{eCatchOT}\AgdaSpace{}%
\AgdaSymbol{(}\AgdaInductiveConstructor{catch}\AgdaSpace{}%
\AgdaBound{x}\AgdaSpace{}%
\AgdaOperator{\AgdaInductiveConstructor{,}}\AgdaSpace{}%
\AgdaBound{k}\AgdaSpace{}%
\AgdaOperator{\AgdaInductiveConstructor{,}}\AgdaSpace{}%
\AgdaBound{ψ}\AgdaSymbol{)}\AgdaSpace{}%
\AgdaSymbol{=}\AgdaSpace{}%
\AgdaKeyword{let}\AgdaSpace{}%
\AgdaBound{m₁}\AgdaSpace{}%
\AgdaSymbol{=}\AgdaSpace{}%
\AgdaBound{ψ}\AgdaSpace{}%
\AgdaInductiveConstructor{true}\AgdaSymbol{;}\AgdaSpace{}%
\AgdaBound{m₂}\AgdaSpace{}%
\AgdaSymbol{=}\AgdaSpace{}%
\AgdaBound{ψ}\AgdaSpace{}%
\AgdaInductiveConstructor{false}\AgdaSpace{}%
\AgdaKeyword{in}\<%
\\
\>[8][@{}l@{\AgdaIndent{0}}]%
\>[10]\AgdaFunction{‵sub}%
\>[16]\AgdaSymbol{(λ}\AgdaSpace{}%
\AgdaBound{r}\AgdaSpace{}%
\AgdaSymbol{→}\AgdaSpace{}%
\AgdaSymbol{(}\AgdaOperator{\AgdaFunction{♯}}\AgdaSpace{}%
\AgdaSymbol{((}\AgdaOperator{\AgdaFunction{given}}\AgdaSpace{}%
\AgdaFunction{hThrow}\AgdaSpace{}%
\AgdaOperator{\AgdaFunction{handle}}\AgdaSpace{}%
\AgdaBound{m₁}\AgdaSymbol{)}\AgdaSpace{}%
\AgdaInductiveConstructor{tt}\AgdaSymbol{))}\AgdaSpace{}%
\AgdaOperator{\AgdaFunction{𝓑}}\AgdaSpace{}%
\AgdaFunction{maybe}\AgdaSpace{}%
\AgdaBound{k}\AgdaSpace{}%
\AgdaSymbol{(}\AgdaFunction{‵jump}\AgdaSpace{}%
\AgdaBound{r}\AgdaSpace{}%
\AgdaSymbol{(}\AgdaField{from}\AgdaSpace{}%
\AgdaInductiveConstructor{tt}\AgdaSymbol{)))}\<%
\\
%
\>[16]\AgdaSymbol{(λ}\AgdaSpace{}%
\AgdaBound{\AgdaUnderscore{}}\AgdaSpace{}%
\AgdaSymbol{→}\AgdaSpace{}%
\AgdaBound{m₂}\AgdaSpace{}%
\AgdaOperator{\AgdaFunction{𝓑}}\AgdaSpace{}%
\AgdaBound{k}\AgdaSymbol{)}\<%
\\
%
\>[10]\AgdaKeyword{where}\AgdaSpace{}%
\AgdaKeyword{instance}\AgdaSpace{}%
\AgdaSymbol{\AgdaUnderscore{}}\AgdaSpace{}%
\AgdaSymbol{=}\AgdaSpace{}%
\AgdaSymbol{\AgdaUnderscore{}}\AgdaSpace{}%
\AgdaOperator{\AgdaInductiveConstructor{,}}\AgdaSpace{}%
\AgdaFunction{∙-comm}\AgdaSpace{}%
\AgdaSymbol{(}\AgdaBound{w₂}\AgdaSpace{}%
\AgdaSymbol{.}\AgdaField{proj₂}\AgdaSymbol{)}\<%
\end{code}
\begin{code}%
%
\>[8]\AgdaFunction{eCatchOT⅋}\AgdaSpace{}%
\AgdaSymbol{:}\AgdaSpace{}%
\AgdaSymbol{⦃}\AgdaSpace{}%
\AgdaBound{w₁}\AgdaSpace{}%
\AgdaSymbol{:}\AgdaSpace{}%
\AgdaFunction{CC}\AgdaSpace{}%
\AgdaBound{Ref}\AgdaSpace{}%
\AgdaOperator{\AgdaPostulate{≲⅋}}\AgdaSpace{}%
\AgdaBound{Δ}\AgdaSpace{}%
\AgdaSymbol{⦄}\AgdaSpace{}%
\AgdaSymbol{⦃}\AgdaSpace{}%
\AgdaBound{w₂}\AgdaSpace{}%
\AgdaSymbol{:}\AgdaSpace{}%
\AgdaFunction{Throw}\AgdaSpace{}%
\AgdaOperator{\AgdaPostulate{≲⅋}}\AgdaSpace{}%
\AgdaBound{Δ}\AgdaSpace{}%
\AgdaSymbol{⦄}\AgdaSpace{}%
\AgdaSymbol{→}\AgdaSpace{}%
\AgdaFunction{Elaboration}\AgdaSpace{}%
\AgdaFunction{Catch}\AgdaSpace{}%
\AgdaBound{Δ}\<%
\\
%
\>[8]\AgdaField{alg}\AgdaSpace{}%
\AgdaFunction{eCatchOT⅋}\AgdaSpace{}%
\AgdaSymbol{(}\AgdaInductiveConstructor{catch}\AgdaSpace{}%
\AgdaBound{x}\AgdaSpace{}%
\AgdaOperator{\AgdaInductiveConstructor{,}}\AgdaSpace{}%
\AgdaBound{k}\AgdaSpace{}%
\AgdaOperator{\AgdaInductiveConstructor{,}}\AgdaSpace{}%
\AgdaBound{ψ}\AgdaSymbol{)}\AgdaSpace{}%
\AgdaSymbol{=}\AgdaSpace{}%
\AgdaKeyword{let}\AgdaSpace{}%
\AgdaBound{m₁}\AgdaSpace{}%
\AgdaSymbol{=}\AgdaSpace{}%
\AgdaBound{ψ}\AgdaSpace{}%
\AgdaInductiveConstructor{true}\AgdaSymbol{;}\AgdaSpace{}%
\AgdaBound{m₂}\AgdaSpace{}%
\AgdaSymbol{=}\AgdaSpace{}%
\AgdaBound{ψ}\AgdaSpace{}%
\AgdaInductiveConstructor{false}\AgdaSpace{}%
\AgdaKeyword{in}\<%
\\
\>[8][@{}l@{\AgdaIndent{0}}]%
\>[10]\AgdaFunction{‵sub}%
\>[16]\AgdaSymbol{(λ}\AgdaSpace{}%
\AgdaBound{r}\AgdaSpace{}%
\AgdaSymbol{→}\AgdaSpace{}%
\AgdaSymbol{(}\AgdaOperator{\AgdaFunction{♯}}\AgdaSpace{}%
\AgdaSymbol{((}\AgdaOperator{\AgdaFunction{given}}\AgdaSpace{}%
\AgdaFunction{hThrow}\AgdaSpace{}%
\AgdaOperator{\AgdaFunction{handle}}\AgdaSpace{}%
\AgdaBound{m₁}\AgdaSymbol{)}\AgdaSpace{}%
\AgdaInductiveConstructor{tt}\AgdaSymbol{))}\AgdaSpace{}%
\AgdaOperator{\AgdaFunction{𝓑}}\AgdaSpace{}%
\AgdaFunction{maybe}\AgdaSpace{}%
\AgdaBound{k}\AgdaSpace{}%
\AgdaSymbol{(}\AgdaFunction{‵jump}\AgdaSpace{}%
\AgdaBound{r}\AgdaSpace{}%
\AgdaSymbol{(}\AgdaField{from}\AgdaSpace{}%
\AgdaInductiveConstructor{tt}\AgdaSymbol{)))}\<%
\\
%
\>[16]\AgdaSymbol{(λ}\AgdaSpace{}%
\AgdaBound{\AgdaUnderscore{}}\AgdaSpace{}%
\AgdaSymbol{→}\AgdaSpace{}%
\AgdaBound{m₂}\AgdaSpace{}%
\AgdaOperator{\AgdaFunction{𝓑}}\AgdaSpace{}%
\AgdaBound{k}\AgdaSymbol{)}\<%
\end{code}
\begin{code}[hide]%
%
\>[10]\AgdaKeyword{where}\AgdaSpace{}%
\AgdaKeyword{instance}\AgdaSpace{}%
\AgdaSymbol{\AgdaUnderscore{}}\AgdaSpace{}%
\AgdaSymbol{=}\AgdaSpace{}%
\AgdaSymbol{\AgdaUnderscore{}}\AgdaSpace{}%
\AgdaOperator{\AgdaInductiveConstructor{,}}\AgdaSpace{}%
\AgdaFunction{∙-comm}\AgdaSpace{}%
\AgdaSymbol{(}\AgdaBound{w₂}\AgdaSpace{}%
\AgdaSymbol{.}\AgdaField{proj₂}\AgdaSymbol{)}\<%
\end{code}
%
The elaboration uses \af{‵sub} to capture the continuation of a higher-order \ac{catch} operation.
If no exception is raised, then control flows to the continuation \ab{k} without invoking the continuation of \af{‵sub}.
Otherwise, we jump to the continuation of \af{‵sub}, which runs \ab{m₂} before passing control to \ab{k}.
Capturing the continuation in this way interacts with state because the continuation of \af{‵sub} may have been pre-applied to a state handler that only knows about the ``old'' state.
This happens when we handle the state effect before the sub/jump effect: in this case we get the transactional interpretation of state, so running \af{transact} gives \an{1}.
Otherwise, if we run the sub/jump handler before the state handler, we get the global interpretation of state and the result \an{2}.
%
\begin{code}[hide]%
%
\>[6]\AgdaComment{--\ instance}\<%
\\
%
\>[6]\AgdaComment{--\ \ \ eCatchOT′\ :\ Elab\ Catch\ Δ}\<%
\\
%
\>[6]\AgdaComment{--\ \ \ orate\ eCatchOT′\ =\ eCatchOT}\<%
\\
%
\\[\AgdaEmptyExtraSkip]%
\>[0][@{}l@{\AgdaIndent{3}}]%
\>[4]\AgdaKeyword{module}\AgdaSpace{}%
\AgdaModule{\AgdaUnderscore{}}\AgdaSpace{}%
\AgdaKeyword{where}\<%
\\
\>[4][@{}l@{\AgdaIndent{0}}]%
\>[6]\AgdaKeyword{open}\AgdaSpace{}%
\AgdaModule{HeftyModule}\AgdaSpace{}%
\AgdaKeyword{using}\AgdaSpace{}%
\AgdaSymbol{(}\AgdaOperator{\AgdaFunction{\AgdaUnderscore{}𝓑\AgdaUnderscore{}}}\AgdaSymbol{;}\AgdaSpace{}%
\AgdaOperator{\AgdaFunction{\AgdaUnderscore{}>>\AgdaUnderscore{}}}\AgdaSymbol{)}\<%
\\
%
\>[6]\AgdaKeyword{open}\AgdaSpace{}%
\AgdaKeyword{import}\AgdaSpace{}%
\AgdaModule{Data.Nat}\AgdaSpace{}%
\AgdaKeyword{using}\AgdaSpace{}%
\AgdaSymbol{(}\AgdaDatatype{ℕ}\AgdaSymbol{;}\AgdaSpace{}%
\AgdaOperator{\AgdaPrimitive{\AgdaUnderscore{}+\AgdaUnderscore{}}}\AgdaSymbol{)}\<%
\\
%
\>[6]\AgdaKeyword{open}\AgdaSpace{}%
\AgdaModule{Inverse}\AgdaSpace{}%
\AgdaSymbol{⦃}\AgdaSpace{}%
\AgdaSymbol{...}\AgdaSpace{}%
\AgdaSymbol{⦄}\<%
\\
\>[0]\<%
\\
%
\>[6]\AgdaFunction{transact}%
\>[1832I]\AgdaSymbol{:}\AgdaSpace{}%
\AgdaSymbol{⦃}\AgdaSpace{}%
\AgdaBound{u}\AgdaSpace{}%
\AgdaSymbol{:}\AgdaSpace{}%
\AgdaRecord{Univ}\AgdaSpace{}%
\AgdaSymbol{⦄}\<%
\\
\>[.][@{}l@{}]\<[1832I]%
\>[15]\AgdaSymbol{→}\AgdaSpace{}%
\AgdaSymbol{⦃}\AgdaSpace{}%
\AgdaBound{wₛ}%
\>[23]\AgdaSymbol{:}\AgdaSpace{}%
\AgdaFunction{Lift}\AgdaSpace{}%
\AgdaFunction{State}\AgdaSpace{}%
\AgdaOperator{\AgdaFunction{≲ᴴ}}\AgdaSpace{}%
\AgdaGeneralizable{H}\AgdaSpace{}%
\AgdaSymbol{⦄}\<%
\\
%
\>[15]\AgdaSymbol{→}\AgdaSpace{}%
\AgdaSymbol{⦃}\AgdaSpace{}%
\AgdaBound{wₜ}%
\>[23]\AgdaSymbol{:}\AgdaSpace{}%
\AgdaFunction{Lift}\AgdaSpace{}%
\AgdaFunction{Throw}\AgdaSpace{}%
\AgdaOperator{\AgdaFunction{≲ᴴ}}\AgdaSpace{}%
\AgdaGeneralizable{H}\AgdaSpace{}%
\AgdaSymbol{⦄}\<%
\\
%
\>[15]\AgdaSymbol{→}\AgdaSpace{}%
\AgdaSymbol{⦃}\AgdaSpace{}%
\AgdaBound{w}%
\>[23]\AgdaSymbol{:}\AgdaSpace{}%
\AgdaFunction{Catch}\AgdaSpace{}%
\AgdaOperator{\AgdaFunction{≲ᴴ}}\AgdaSpace{}%
\AgdaGeneralizable{H}\AgdaSpace{}%
\AgdaSymbol{⦄}\<%
\\
%
\>[15]\AgdaSymbol{→}\AgdaSpace{}%
\AgdaSymbol{\{}\AgdaBound{unit}\AgdaSpace{}%
\AgdaSymbol{:}\AgdaSpace{}%
\AgdaField{Type}\AgdaSymbol{\}}\AgdaSpace{}%
\AgdaSymbol{⦃}\AgdaSpace{}%
\AgdaBound{iso}\AgdaSpace{}%
\AgdaSymbol{:}\AgdaSpace{}%
\AgdaRecord{⊤}\AgdaSpace{}%
\AgdaOperator{\AgdaFunction{↔}}\AgdaSpace{}%
\AgdaOperator{\AgdaField{⟦}}\AgdaSpace{}%
\AgdaBound{unit}\AgdaSpace{}%
\AgdaOperator{\AgdaField{⟧ᵀ}}\AgdaSpace{}%
\AgdaSymbol{⦄}\<%
\\
%
\>[15]\AgdaSymbol{→}\AgdaSpace{}%
\AgdaDatatype{Hefty}\AgdaSpace{}%
\AgdaGeneralizable{H}\AgdaSpace{}%
\AgdaDatatype{ℕ}\<%
\\
%
\>[6]\AgdaFunction{transact}\AgdaSpace{}%
\AgdaSymbol{\{}\AgdaArgument{unit}\AgdaSpace{}%
\AgdaSymbol{=}\AgdaSpace{}%
\AgdaBound{unit}\AgdaSymbol{\}}\AgdaSpace{}%
\AgdaSymbol{=}\AgdaSpace{}%
\AgdaKeyword{do}\<%
\\
\>[6][@{}l@{\AgdaIndent{0}}]%
\>[8]\AgdaOperator{\AgdaFunction{↑}}\AgdaSpace{}%
\AgdaSymbol{(}\AgdaInductiveConstructor{put}\AgdaSpace{}%
\AgdaNumber{1}\AgdaSymbol{)}\<%
\\
%
\>[8]\AgdaFunction{‵catch}\AgdaSpace{}%
\AgdaSymbol{(}\AgdaKeyword{do}\AgdaSpace{}%
\AgdaOperator{\AgdaFunction{↑}}\AgdaSpace{}%
\AgdaSymbol{(}\AgdaInductiveConstructor{put}\AgdaSpace{}%
\AgdaNumber{2}\AgdaSymbol{);}\AgdaSpace{}%
\AgdaSymbol{((}\AgdaOperator{\AgdaFunction{↑}}\AgdaSpace{}%
\AgdaInductiveConstructor{throw}\AgdaSymbol{)}\AgdaSpace{}%
\AgdaOperator{\AgdaFunction{𝓑}}\AgdaSpace{}%
\AgdaFunction{⊥-elim}\AgdaSymbol{))}\AgdaSpace{}%
\AgdaSymbol{(}\AgdaInductiveConstructor{pure}\AgdaSpace{}%
\AgdaSymbol{(}\AgdaField{to}\AgdaSpace{}%
\AgdaInductiveConstructor{tt}\AgdaSymbol{))}\<%
\\
%
\>[8]\AgdaOperator{\AgdaFunction{↑}}\AgdaSpace{}%
\AgdaInductiveConstructor{get}\<%
\\
%
\\[\AgdaEmptyExtraSkip]%
%
\>[4]\AgdaKeyword{module}\AgdaSpace{}%
\AgdaModule{CatchExample}\AgdaSpace{}%
\AgdaKeyword{where}\AgdaSpace{}%
\AgdaKeyword{private}\<%
\\
\>[4][@{}l@{\AgdaIndent{0}}]%
\>[6]\AgdaKeyword{open}\AgdaSpace{}%
\AgdaKeyword{import}\AgdaSpace{}%
\AgdaModule{Data.Nat}\AgdaSpace{}%
\AgdaKeyword{using}\AgdaSpace{}%
\AgdaSymbol{(}\AgdaDatatype{ℕ}\AgdaSymbol{)}\<%
\\
%
\>[6]\AgdaKeyword{open}\AgdaSpace{}%
\AgdaModule{ElabModule}\<%
\\
%
\>[6]\AgdaKeyword{open}\AgdaSpace{}%
\AgdaModule{Inverse}\AgdaSpace{}%
\AgdaSymbol{⦃}\AgdaSpace{}%
\AgdaSymbol{...}\AgdaSpace{}%
\AgdaSymbol{⦄}\<%
\\
%
\>[6]\AgdaKeyword{open}\AgdaSpace{}%
\AgdaKeyword{import}\AgdaSpace{}%
\AgdaModule{Function.Construct.Identity}%
\>[49]\AgdaKeyword{using}\AgdaSpace{}%
\AgdaSymbol{(}\AgdaFunction{↔-id}\AgdaSymbol{)}\<%
\\
%
\>[6]\AgdaComment{--\ open\ Elab}\<%
\\
%
\\[\AgdaEmptyExtraSkip]%
%
\>[6]\AgdaKeyword{data}\AgdaSpace{}%
\AgdaDatatype{CatchType}\AgdaSpace{}%
\AgdaSymbol{:}\AgdaSpace{}%
\AgdaPrimitive{Set}\AgdaSpace{}%
\AgdaKeyword{where}\<%
\\
\>[6][@{}l@{\AgdaIndent{0}}]%
\>[8]\AgdaInductiveConstructor{unit}%
\>[15]\AgdaSymbol{:}\AgdaSpace{}%
\AgdaDatatype{CatchType}\<%
\\
%
\>[8]\AgdaInductiveConstructor{num}\AgdaSpace{}%
\AgdaSymbol{:}\AgdaSpace{}%
\AgdaDatatype{CatchType}\<%
\\
%
\\[\AgdaEmptyExtraSkip]%
%
\>[6]\AgdaKeyword{instance}\<%
\\
\>[6][@{}l@{\AgdaIndent{0}}]%
\>[8]\AgdaFunction{CatchUniv}\AgdaSpace{}%
\AgdaSymbol{:}\AgdaSpace{}%
\AgdaRecord{Univ}\<%
\\
%
\>[8]\AgdaField{Type}%
\>[14]\AgdaSymbol{⦃}\AgdaSpace{}%
\AgdaFunction{CatchUniv}\AgdaSpace{}%
\AgdaSymbol{⦄}\AgdaSpace{}%
\AgdaSymbol{=}\AgdaSpace{}%
\AgdaDatatype{CatchType}\<%
\\
%
\>[8]\AgdaOperator{\AgdaField{⟦\AgdaUnderscore{}⟧ᵀ}}\AgdaSpace{}%
\AgdaSymbol{⦃}\AgdaSpace{}%
\AgdaFunction{CatchUniv}\AgdaSpace{}%
\AgdaSymbol{⦄}\AgdaSpace{}%
\AgdaInductiveConstructor{unit}%
\>[34]\AgdaSymbol{=}\AgdaSpace{}%
\AgdaRecord{⊤}\<%
\\
%
\>[8]\AgdaOperator{\AgdaField{⟦\AgdaUnderscore{}⟧ᵀ}}\AgdaSpace{}%
\AgdaSymbol{⦃}\AgdaSpace{}%
\AgdaFunction{CatchUniv}\AgdaSpace{}%
\AgdaSymbol{⦄}\AgdaSpace{}%
\AgdaInductiveConstructor{num}\AgdaSpace{}%
\AgdaSymbol{=}\AgdaSpace{}%
\AgdaDatatype{ℕ}\<%
\\
%
\\[\AgdaEmptyExtraSkip]%
%
\>[8]\AgdaFunction{iso-1}\AgdaSpace{}%
\AgdaSymbol{:}\AgdaSpace{}%
\AgdaRecord{⊤}\AgdaSpace{}%
\AgdaOperator{\AgdaFunction{↔}}\AgdaSpace{}%
\AgdaOperator{\AgdaField{⟦}}\AgdaSpace{}%
\AgdaInductiveConstructor{unit}\AgdaSpace{}%
\AgdaOperator{\AgdaField{⟧ᵀ}}\<%
\\
%
\>[8]\AgdaFunction{iso-1}\AgdaSpace{}%
\AgdaSymbol{=}\AgdaSpace{}%
\AgdaFunction{↔-id}\AgdaSpace{}%
\AgdaSymbol{\AgdaUnderscore{}}\<%
\\
%
\\[\AgdaEmptyExtraSkip]%
%
\>[6]\AgdaKeyword{module}\AgdaSpace{}%
\AgdaModule{\AgdaUnderscore{}}\AgdaSpace{}%
\AgdaKeyword{where}\<%
\\
\>[6][@{}l@{\AgdaIndent{0}}]%
\>[8]\AgdaKeyword{private}\AgdaSpace{}%
\AgdaKeyword{instance}\<%
\\
\>[8][@{}l@{\AgdaIndent{0}}]%
\>[10]\AgdaFunction{x₀}\AgdaSpace{}%
\AgdaSymbol{:}\AgdaSpace{}%
\AgdaFunction{CC}\AgdaSpace{}%
\AgdaSymbol{(λ}\AgdaSpace{}%
\AgdaBound{t}\AgdaSpace{}%
\AgdaSymbol{→}\AgdaSpace{}%
\AgdaOperator{\AgdaField{⟦}}\AgdaSpace{}%
\AgdaBound{t}\AgdaSpace{}%
\AgdaOperator{\AgdaField{⟧ᵀ}}\AgdaSpace{}%
\AgdaSymbol{→}\AgdaSpace{}%
\AgdaDatatype{Free}\AgdaSpace{}%
\AgdaFunction{Nil}\AgdaSpace{}%
\AgdaGeneralizable{A}\AgdaSymbol{)}\AgdaSpace{}%
\AgdaOperator{\AgdaFunction{≲}}\AgdaSpace{}%
\AgdaSymbol{(}\AgdaFunction{CC}\AgdaSpace{}%
\AgdaSymbol{(λ}\AgdaSpace{}%
\AgdaBound{t}\AgdaSpace{}%
\AgdaSymbol{→}\AgdaSpace{}%
\AgdaOperator{\AgdaField{⟦}}\AgdaSpace{}%
\AgdaBound{t}\AgdaSpace{}%
\AgdaOperator{\AgdaField{⟧ᵀ}}\AgdaSpace{}%
\AgdaSymbol{→}\AgdaSpace{}%
\AgdaDatatype{Free}\AgdaSpace{}%
\AgdaFunction{Nil}\AgdaSpace{}%
\AgdaGeneralizable{A}\AgdaSymbol{)}\AgdaSpace{}%
\AgdaOperator{\AgdaFunction{⊕}}\AgdaSpace{}%
\AgdaFunction{State}\AgdaSpace{}%
\AgdaOperator{\AgdaFunction{⊕}}\AgdaSpace{}%
\AgdaFunction{Throw}\AgdaSpace{}%
\AgdaOperator{\AgdaFunction{⊕}}\AgdaSpace{}%
\AgdaFunction{Nil}\AgdaSymbol{)}\<%
\\
%
\>[10]\AgdaFunction{x₀}\AgdaSpace{}%
\AgdaSymbol{=}\AgdaSpace{}%
\AgdaFunction{≲-left}\<%
\\
%
\\[\AgdaEmptyExtraSkip]%
%
\>[10]\AgdaFunction{x₁}\AgdaSpace{}%
\AgdaSymbol{:}\AgdaSpace{}%
\AgdaFunction{State}\AgdaSpace{}%
\AgdaOperator{\AgdaFunction{≲}}\AgdaSpace{}%
\AgdaSymbol{(}\AgdaFunction{CC}\AgdaSpace{}%
\AgdaSymbol{(λ}\AgdaSpace{}%
\AgdaBound{t}\AgdaSpace{}%
\AgdaSymbol{→}\AgdaSpace{}%
\AgdaOperator{\AgdaField{⟦}}\AgdaSpace{}%
\AgdaBound{t}\AgdaSpace{}%
\AgdaOperator{\AgdaField{⟧ᵀ}}\AgdaSpace{}%
\AgdaSymbol{→}\AgdaSpace{}%
\AgdaDatatype{Free}\AgdaSpace{}%
\AgdaFunction{Nil}\AgdaSpace{}%
\AgdaGeneralizable{A}\AgdaSymbol{)}\AgdaSpace{}%
\AgdaOperator{\AgdaFunction{⊕}}\AgdaSpace{}%
\AgdaFunction{State}\AgdaSpace{}%
\AgdaOperator{\AgdaFunction{⊕}}\AgdaSpace{}%
\AgdaFunction{Throw}\AgdaSpace{}%
\AgdaOperator{\AgdaFunction{⊕}}\AgdaSpace{}%
\AgdaFunction{Nil}\AgdaSymbol{)}\<%
\\
%
\>[10]\AgdaFunction{x₁}\AgdaSpace{}%
\AgdaSymbol{=}\AgdaSpace{}%
\AgdaFunction{≲-right}\AgdaSpace{}%
\AgdaSymbol{⦃}\AgdaSpace{}%
\AgdaFunction{≲-left}\AgdaSpace{}%
\AgdaSymbol{⦄}\<%
\\
%
\\[\AgdaEmptyExtraSkip]%
%
\>[10]\AgdaFunction{x₂}\AgdaSpace{}%
\AgdaSymbol{:}\AgdaSpace{}%
\AgdaFunction{Throw}\AgdaSpace{}%
\AgdaOperator{\AgdaFunction{≲}}\AgdaSpace{}%
\AgdaSymbol{(}\AgdaFunction{CC}\AgdaSpace{}%
\AgdaSymbol{(λ}\AgdaSpace{}%
\AgdaBound{t}\AgdaSpace{}%
\AgdaSymbol{→}\AgdaSpace{}%
\AgdaOperator{\AgdaField{⟦}}\AgdaSpace{}%
\AgdaBound{t}\AgdaSpace{}%
\AgdaOperator{\AgdaField{⟧ᵀ}}\AgdaSpace{}%
\AgdaSymbol{→}\AgdaSpace{}%
\AgdaDatatype{Free}\AgdaSpace{}%
\AgdaFunction{Nil}\AgdaSpace{}%
\AgdaGeneralizable{A}\AgdaSymbol{)}\AgdaSpace{}%
\AgdaOperator{\AgdaFunction{⊕}}\AgdaSpace{}%
\AgdaFunction{State}\AgdaSpace{}%
\AgdaOperator{\AgdaFunction{⊕}}\AgdaSpace{}%
\AgdaFunction{Throw}\AgdaSpace{}%
\AgdaOperator{\AgdaFunction{⊕}}\AgdaSpace{}%
\AgdaFunction{Nil}\AgdaSymbol{)}\<%
\\
%
\>[10]\AgdaFunction{x₂}\AgdaSpace{}%
\AgdaSymbol{=}\AgdaSpace{}%
\AgdaFunction{≲-right}\AgdaSpace{}%
\AgdaSymbol{⦃}\AgdaSpace{}%
\AgdaFunction{≲-right}\AgdaSpace{}%
\AgdaSymbol{⦃}\AgdaSpace{}%
\AgdaFunction{≲-left}\AgdaSpace{}%
\AgdaSymbol{⦄}\AgdaSpace{}%
\AgdaSymbol{⦄}\<%
\\
%
\\[\AgdaEmptyExtraSkip]%
%
\>[8]\AgdaFunction{transact-elab₂}\AgdaSpace{}%
\AgdaSymbol{:}%
\>[2029I]\AgdaFunction{Elaboration}\<%
\\
\>[2029I][@{}l@{\AgdaIndent{0}}]%
\>[27]\AgdaSymbol{(}\AgdaFunction{Lift}\AgdaSpace{}%
\AgdaFunction{State}\AgdaSpace{}%
\AgdaOperator{\AgdaFunction{∔}}\AgdaSpace{}%
\AgdaFunction{Lift}\AgdaSpace{}%
\AgdaFunction{Throw}\AgdaSpace{}%
\AgdaOperator{\AgdaFunction{∔}}\AgdaSpace{}%
\AgdaFunction{Catch}\AgdaSpace{}%
\AgdaOperator{\AgdaFunction{∔}}\AgdaSpace{}%
\AgdaFunction{Lift}\AgdaSpace{}%
\AgdaFunction{Nil}\AgdaSymbol{)}\<%
\\
%
\>[27]\AgdaSymbol{(}\AgdaFunction{CC}\AgdaSpace{}%
\AgdaSymbol{(λ}\AgdaSpace{}%
\AgdaBound{t}\AgdaSpace{}%
\AgdaSymbol{→}\AgdaSpace{}%
\AgdaOperator{\AgdaField{⟦}}\AgdaSpace{}%
\AgdaBound{t}\AgdaSpace{}%
\AgdaOperator{\AgdaField{⟧ᵀ}}\AgdaSpace{}%
\AgdaSymbol{→}\AgdaSpace{}%
\AgdaDatatype{Free}\AgdaSpace{}%
\AgdaFunction{Nil}\AgdaSpace{}%
\AgdaGeneralizable{A}\AgdaSymbol{)}\AgdaSpace{}%
\AgdaOperator{\AgdaFunction{⊕}}\AgdaSpace{}%
\AgdaFunction{State}\AgdaSpace{}%
\AgdaOperator{\AgdaFunction{⊕}}\AgdaSpace{}%
\AgdaFunction{Throw}\AgdaSpace{}%
\AgdaOperator{\AgdaFunction{⊕}}\AgdaSpace{}%
\AgdaFunction{Nil}\AgdaSymbol{)}\<%
\\
%
\>[8]\AgdaFunction{transact-elab₂}\AgdaSpace{}%
\AgdaSymbol{=}\AgdaSpace{}%
\AgdaFunction{eLift}\AgdaSpace{}%
\AgdaOperator{\AgdaFunction{⋎}}\AgdaSpace{}%
\AgdaFunction{eLift}\AgdaSpace{}%
\AgdaOperator{\AgdaFunction{⋎}}\AgdaSpace{}%
\AgdaFunction{eCatchOT}\AgdaSpace{}%
\AgdaOperator{\AgdaFunction{⋎}}\AgdaSpace{}%
\AgdaFunction{eNil}\<%
\\
%
\\[\AgdaEmptyExtraSkip]%
%
\>[6]\AgdaKeyword{module}\AgdaSpace{}%
\AgdaModule{\AgdaUnderscore{}}\AgdaSpace{}%
\AgdaKeyword{where}\<%
\\
\>[6][@{}l@{\AgdaIndent{0}}]%
\>[8]\AgdaKeyword{private}\AgdaSpace{}%
\AgdaKeyword{instance}\<%
\\
\>[8][@{}l@{\AgdaIndent{0}}]%
\>[10]\AgdaFunction{x₀}\AgdaSpace{}%
\AgdaSymbol{:}\AgdaSpace{}%
\AgdaFunction{CC}\AgdaSpace{}%
\AgdaSymbol{(λ}\AgdaSpace{}%
\AgdaBound{t}\AgdaSpace{}%
\AgdaSymbol{→}\AgdaSpace{}%
\AgdaOperator{\AgdaField{⟦}}\AgdaSpace{}%
\AgdaBound{t}\AgdaSpace{}%
\AgdaOperator{\AgdaField{⟧ᵀ}}\AgdaSpace{}%
\AgdaSymbol{→}\AgdaSpace{}%
\AgdaDatatype{Free}\AgdaSpace{}%
\AgdaSymbol{(}\AgdaFunction{State}\AgdaSpace{}%
\AgdaOperator{\AgdaFunction{⊕}}\AgdaSpace{}%
\AgdaFunction{Nil}\AgdaSymbol{)}\AgdaSpace{}%
\AgdaGeneralizable{A}\AgdaSymbol{)}\AgdaSpace{}%
\AgdaOperator{\AgdaFunction{≲}}\AgdaSpace{}%
\AgdaSymbol{(}\AgdaFunction{CC}\AgdaSpace{}%
\AgdaSymbol{(λ}\AgdaSpace{}%
\AgdaBound{t}\AgdaSpace{}%
\AgdaSymbol{→}\AgdaSpace{}%
\AgdaOperator{\AgdaField{⟦}}\AgdaSpace{}%
\AgdaBound{t}\AgdaSpace{}%
\AgdaOperator{\AgdaField{⟧ᵀ}}\AgdaSpace{}%
\AgdaSymbol{→}\AgdaSpace{}%
\AgdaDatatype{Free}\AgdaSpace{}%
\AgdaSymbol{(}\AgdaFunction{State}\AgdaSpace{}%
\AgdaOperator{\AgdaFunction{⊕}}\AgdaSpace{}%
\AgdaFunction{Nil}\AgdaSymbol{)}\AgdaSpace{}%
\AgdaGeneralizable{A}\AgdaSymbol{)}\AgdaSpace{}%
\AgdaOperator{\AgdaFunction{⊕}}\AgdaSpace{}%
\AgdaFunction{State}\AgdaSpace{}%
\AgdaOperator{\AgdaFunction{⊕}}\AgdaSpace{}%
\AgdaFunction{Throw}\AgdaSpace{}%
\AgdaOperator{\AgdaFunction{⊕}}\AgdaSpace{}%
\AgdaFunction{Nil}\AgdaSymbol{)}\<%
\\
%
\>[10]\AgdaFunction{x₀}\AgdaSpace{}%
\AgdaSymbol{=}\AgdaSpace{}%
\AgdaFunction{≲-left}\<%
\\
%
\\[\AgdaEmptyExtraSkip]%
%
\>[10]\AgdaFunction{x₁}\AgdaSpace{}%
\AgdaSymbol{:}\AgdaSpace{}%
\AgdaFunction{State}\AgdaSpace{}%
\AgdaOperator{\AgdaFunction{≲}}\AgdaSpace{}%
\AgdaSymbol{(}\AgdaFunction{CC}\AgdaSpace{}%
\AgdaSymbol{(λ}\AgdaSpace{}%
\AgdaBound{t}\AgdaSpace{}%
\AgdaSymbol{→}\AgdaSpace{}%
\AgdaOperator{\AgdaField{⟦}}\AgdaSpace{}%
\AgdaBound{t}\AgdaSpace{}%
\AgdaOperator{\AgdaField{⟧ᵀ}}\AgdaSpace{}%
\AgdaSymbol{→}\AgdaSpace{}%
\AgdaDatatype{Free}\AgdaSpace{}%
\AgdaSymbol{(}\AgdaFunction{State}\AgdaSpace{}%
\AgdaOperator{\AgdaFunction{⊕}}\AgdaSpace{}%
\AgdaFunction{Nil}\AgdaSymbol{)}\AgdaSpace{}%
\AgdaGeneralizable{A}\AgdaSymbol{)}\AgdaSpace{}%
\AgdaOperator{\AgdaFunction{⊕}}\AgdaSpace{}%
\AgdaFunction{State}\AgdaSpace{}%
\AgdaOperator{\AgdaFunction{⊕}}\AgdaSpace{}%
\AgdaFunction{Throw}\AgdaSpace{}%
\AgdaOperator{\AgdaFunction{⊕}}\AgdaSpace{}%
\AgdaFunction{Nil}\AgdaSymbol{)}\<%
\\
%
\>[10]\AgdaFunction{x₁}\AgdaSpace{}%
\AgdaSymbol{=}\AgdaSpace{}%
\AgdaFunction{≲-right}\AgdaSpace{}%
\AgdaSymbol{⦃}\AgdaSpace{}%
\AgdaFunction{≲-left}\AgdaSpace{}%
\AgdaSymbol{⦄}\<%
\\
%
\\[\AgdaEmptyExtraSkip]%
%
\>[10]\AgdaFunction{x₂}\AgdaSpace{}%
\AgdaSymbol{:}\AgdaSpace{}%
\AgdaFunction{Throw}\AgdaSpace{}%
\AgdaOperator{\AgdaFunction{≲}}\AgdaSpace{}%
\AgdaSymbol{(}\AgdaFunction{CC}\AgdaSpace{}%
\AgdaSymbol{(λ}\AgdaSpace{}%
\AgdaBound{t}\AgdaSpace{}%
\AgdaSymbol{→}\AgdaSpace{}%
\AgdaOperator{\AgdaField{⟦}}\AgdaSpace{}%
\AgdaBound{t}\AgdaSpace{}%
\AgdaOperator{\AgdaField{⟧ᵀ}}\AgdaSpace{}%
\AgdaSymbol{→}\AgdaSpace{}%
\AgdaDatatype{Free}\AgdaSpace{}%
\AgdaSymbol{(}\AgdaFunction{State}\AgdaSpace{}%
\AgdaOperator{\AgdaFunction{⊕}}\AgdaSpace{}%
\AgdaFunction{Nil}\AgdaSymbol{)}\AgdaSpace{}%
\AgdaGeneralizable{A}\AgdaSymbol{)}\AgdaSpace{}%
\AgdaOperator{\AgdaFunction{⊕}}\AgdaSpace{}%
\AgdaFunction{State}\AgdaSpace{}%
\AgdaOperator{\AgdaFunction{⊕}}\AgdaSpace{}%
\AgdaFunction{Throw}\AgdaSpace{}%
\AgdaOperator{\AgdaFunction{⊕}}\AgdaSpace{}%
\AgdaFunction{Nil}\AgdaSymbol{)}\<%
\\
%
\>[10]\AgdaFunction{x₂}\AgdaSpace{}%
\AgdaSymbol{=}\AgdaSpace{}%
\AgdaFunction{≲-right}\AgdaSpace{}%
\AgdaSymbol{⦃}\AgdaSpace{}%
\AgdaFunction{≲-right}\AgdaSpace{}%
\AgdaSymbol{⦃}\AgdaSpace{}%
\AgdaFunction{≲-left}\AgdaSpace{}%
\AgdaSymbol{⦄}\AgdaSpace{}%
\AgdaSymbol{⦄}\<%
\\
%
\\[\AgdaEmptyExtraSkip]%
%
\>[8]\AgdaFunction{transact-elab₃}\AgdaSpace{}%
\AgdaSymbol{:}%
\>[2160I]\AgdaFunction{Elaboration}\<%
\\
\>[2160I][@{}l@{\AgdaIndent{0}}]%
\>[27]\AgdaSymbol{(}\AgdaFunction{Lift}\AgdaSpace{}%
\AgdaFunction{State}\AgdaSpace{}%
\AgdaOperator{\AgdaFunction{∔}}\AgdaSpace{}%
\AgdaFunction{Lift}\AgdaSpace{}%
\AgdaFunction{Throw}\AgdaSpace{}%
\AgdaOperator{\AgdaFunction{∔}}\AgdaSpace{}%
\AgdaFunction{Catch}\AgdaSpace{}%
\AgdaOperator{\AgdaFunction{∔}}\AgdaSpace{}%
\AgdaFunction{Lift}\AgdaSpace{}%
\AgdaFunction{Nil}\AgdaSymbol{)}\<%
\\
%
\>[27]\AgdaSymbol{(}\AgdaFunction{CC}\AgdaSpace{}%
\AgdaSymbol{(λ}\AgdaSpace{}%
\AgdaBound{t}\AgdaSpace{}%
\AgdaSymbol{→}\AgdaSpace{}%
\AgdaOperator{\AgdaField{⟦}}\AgdaSpace{}%
\AgdaBound{t}\AgdaSpace{}%
\AgdaOperator{\AgdaField{⟧ᵀ}}\AgdaSpace{}%
\AgdaSymbol{→}\AgdaSpace{}%
\AgdaDatatype{Free}\AgdaSpace{}%
\AgdaSymbol{(}\AgdaFunction{State}\AgdaSpace{}%
\AgdaOperator{\AgdaFunction{⊕}}\AgdaSpace{}%
\AgdaFunction{Nil}\AgdaSymbol{)}\AgdaSpace{}%
\AgdaGeneralizable{A}\AgdaSymbol{)}\AgdaSpace{}%
\AgdaOperator{\AgdaFunction{⊕}}\AgdaSpace{}%
\AgdaFunction{State}\AgdaSpace{}%
\AgdaOperator{\AgdaFunction{⊕}}\AgdaSpace{}%
\AgdaFunction{Throw}\AgdaSpace{}%
\AgdaOperator{\AgdaFunction{⊕}}\AgdaSpace{}%
\AgdaFunction{Nil}\AgdaSymbol{)}\<%
\\
%
\>[8]\AgdaFunction{transact-elab₃}\AgdaSpace{}%
\AgdaSymbol{=}\AgdaSpace{}%
\AgdaFunction{eLift}\AgdaSpace{}%
\AgdaOperator{\AgdaFunction{⋎}}\AgdaSpace{}%
\AgdaFunction{eLift}\AgdaSpace{}%
\AgdaOperator{\AgdaFunction{⋎}}\AgdaSpace{}%
\AgdaFunction{eCatchOT}\AgdaSpace{}%
\AgdaOperator{\AgdaFunction{⋎}}\AgdaSpace{}%
\AgdaFunction{eNil}\<%
\end{code}
\begin{code}[hide]%
%
\>[6]\AgdaComment{--\ module\ \AgdaUnderscore{}\ where}\<%
\\
%
\>[6]\AgdaComment{--\ \ \ private\ instance}\<%
\\
%
\>[6]\AgdaComment{--\ \ \ \ \ x₀\ :\ CC\ (λ\ t\ →\ ⟦\ t\ ⟧ᵀ\ →\ Free\ Nil\ A)\ ≲\ (State\ ⊕\ Throw\ ⊕\ CC\ (λ\ t\ →\ ⟦\ t\ ⟧ᵀ\ →\ Free\ Nil\ A)\ ⊕\ Nil)}\<%
\\
%
\>[6]\AgdaComment{--\ \ \ \ \ x₀\ =\ ≲-right\ ⦃\ ≲-right\ ⦃\ ≲-left\ ⦄\ ⦄}\<%
\\
%
\\[\AgdaEmptyExtraSkip]%
%
\>[6]\AgdaComment{--\ \ \ \ \ x₁\ :\ State\ ≲\ (State\ ⊕\ Throw\ ⊕\ CC\ (λ\ t\ →\ ⟦\ t\ ⟧ᵀ\ →\ Free\ Nil\ A)\ ⊕\ Nil)}\<%
\\
%
\>[6]\AgdaComment{--\ \ \ \ \ x₁\ =\ ≲-left\ ⦄}\<%
\\
%
\\[\AgdaEmptyExtraSkip]%
%
\>[6]\AgdaComment{--\ \ \ \ \ x₂\ :\ Throw\ ≲\ (CC\ (λ\ t\ →\ ⟦\ t\ ⟧ᵀ\ →\ Free\ Nil\ A)\ ⊕\ State\ ⊕\ Throw\ ⊕\ Nil)}\<%
\\
%
\>[6]\AgdaComment{--\ \ \ \ \ x₂\ =\ ≲-right\ ⦃\ ≲-right\ ⦃\ ≲-left\ ⦄\ ⦄}\<%
\\
%
\\[\AgdaEmptyExtraSkip]%
%
\>[6]\AgdaComment{--\ \ \ \ \ y₀\ :\ Lift\ State\ ≲ᴴ\ (Lift\ State\ ∔\ Lift\ Throw\ ∔\ Catch\ ∔\ Lift\ Nil)}\<%
\\
%
\>[6]\AgdaComment{--\ \ \ \ \ y₀\ =\ ≲ᴴ-left}\<%
\\
%
\\[\AgdaEmptyExtraSkip]%
%
\>[6]\AgdaComment{--\ \ \ \ \ y₁\ :\ Lift\ Throw\ ≲ᴴ\ (Lift\ State\ ∔\ Lift\ Throw\ ∔\ Catch\ ∔\ Lift\ Nil)}\<%
\\
%
\>[6]\AgdaComment{--\ \ \ \ \ y₁\ =\ ≲ᴴ-right\ ⦃\ ≲ᴴ-left\ ⦄}\<%
\\
%
\\[\AgdaEmptyExtraSkip]%
%
\>[6]\AgdaComment{--\ \ \ \ \ y₂\ :\ Catch\ ≲ᴴ\ (Lift\ State\ ∔\ Lift\ Throw\ ∔\ Catch\ ∔\ Lift\ Nil)}\<%
\\
%
\>[6]\AgdaComment{--\ \ \ \ \ y₂\ =\ ≲ᴴ-right\ ⦃\ ≲ᴴ-right\ ⦃\ ≲ᴴ-left\ ⦄\ ⦄}\<%
\\
%
\\[\AgdaEmptyExtraSkip]%
%
\>[6]\AgdaComment{--\ \ \ test-transact₂\ :\ \ un}\<%
\\
%
\>[6]\AgdaComment{--\ \ \ \ \ \ \ \ \ \ \ \ \ \ \ \ \ \ \ \ \ \ \ (given\ hCC}\<%
\\
%
\>[6]\AgdaComment{--\ \ \ \ \ \ \ \ \ \ \ \ \ \ \ \ \ \ \ \ \ \ \ \ handle\ (given\ hThrow}\<%
\\
%
\>[6]\AgdaComment{--\ \ \ \ \ \ \ \ \ \ \ \ \ \ \ \ \ \ \ \ \ \ \ \ \ \ \ \ \ \ \ \ handle\ (given\ hSt}\<%
\\
%
\>[6]\AgdaComment{--\ \ \ \ \ \ \ \ \ \ \ \ \ \ \ \ \ \ \ \ \ \ \ \ \ \ \ \ \ \ \ \ \ \ \ \ \ \ \ \ handle\ (elaborate\ transact-elab₂\ transact)\ \$\ 0)}\<%
\\
%
\>[6]\AgdaComment{--\ \ \ \ \ \ \ \ \ \ \ \ \ \ \ \ \ \ \ \ \ \ \ \ \ \ \ \ \ \ \ \$\ tt)}\<%
\\
%
\>[6]\AgdaComment{--\ \ \ \ \ \ \ \ \ \ \ \ \ \ \ \ \ \ \ \ \ \ \ \ \$\ tt)}\<%
\\
%
\>[6]\AgdaComment{--\ \ \ \ \ \ \ \ \ \ \ \ \ \ \ \ \ \ \ \ \ \ \ ≡\ just\ (1\ ,\ 1)}\<%
\\
%
\>[6]\AgdaComment{--\ \ \ test-transact₂\ =\ refl}\<%
\\
%
\\[\AgdaEmptyExtraSkip]%
\>[0]\AgdaComment{--\ \ \ \ \ \ \ test-transact₃\ :\ un\ (given\ hSt}\<%
\\
\>[0]\AgdaComment{--\ \ \ \ \ \ \ \ \ \ \ \ \ \ \ \ \ \ \ \ \ \ \ \ \ \ \ \ handle\ (given\ hCC}\<%
\\
\>[0]\AgdaComment{--\ \ \ \ \ \ \ \ \ \ \ \ \ \ \ \ \ \ \ \ \ \ \ \ \ \ \ \ \ \ \ \ \ \ \ \ handle\ (given\ hThrow}\<%
\\
\>[0]\AgdaComment{--\ \ \ \ \ \ \ \ \ \ \ \ \ \ \ \ \ \ \ \ \ \ \ \ \ \ \ \ \ \ \ \ \ \ \ \ \ \ \ \ \ \ \ \ handle\ (elaborate\ transact-elab₃\ transact)}\<%
\\
\>[0]\AgdaComment{--\ \ \ \ \ \ \ \ \ \ \ \ \ \ \ \ \ \ \ \ \ \ \ \ \ \ \ \ \ \ \ \ \ \ \ \ \ \ \ \ \ \ \ \$\ tt)}\<%
\\
\>[0]\AgdaComment{--\ \ \ \ \ \ \ \ \ \ \ \ \ \ \ \ \ \ \ \ \ \ \ \ \ \ \ \ \ \ \ \ \ \ \ \$\ tt)}\<%
\\
\>[0]\AgdaComment{--\ \ \ \ \ \ \ \ \ \ \ \ \ \ \ \ \ \ \ \ \ \ \ \ \ \ \ \$\ 0)\ ≡\ (just\ 2\ ,\ 2)}\<%
\\
\>[0]\AgdaComment{--\ \ \ \ \ \ \ test-transact₃\ =\ refl}\<%
\end{code}
\begin{code}[hide]%
\>[0]\AgdaComment{--\ \ \ \ \ \ \ transact′\ :\ ⦃\ wₛ\ :\ H\ ∼\ Lift\ State\ ▹\ H′\ ⦄\ ⦃\ wₜ\ :\ H\ ∼\ \ Lift\ Throw\ ▹\ H″\ ⦄\ ⦃\ w\ \ :\ H\ ∼\ Catch\ ▹\ H‴\ ⦄}\<%
\\
\>[0]\AgdaComment{--\ \ \ \ \ \ \ \ \ \ \ \ \ \ \ \ \ →\ Hefty\ H\ ℕ}\<%
\\
\>[0]\AgdaComment{--\ \ \ \ \ \ \ transact′\ =\ do}\<%
\\
\>[0]\AgdaComment{--\ \ \ \ \ \ \ \ \ ↑\ put\ 1}\<%
\\
\>[0]\AgdaComment{--\ \ \ \ \ \ \ \ \ ‵catch\ (do\ ↑\ put\ 2)\ (pure\ (from\ tt))}\<%
\\
\>[0]\AgdaComment{--\ \ \ \ \ \ \ \ \ ↑\ get}\<%
\\
\>[0]\AgdaComment{--\ \ \ \ \ \ \ \ \ where\ open\ HeftyModule\ using\ (\AgdaUnderscore{}>>\AgdaUnderscore{})}\<%
\\
\>[0]\AgdaComment{--\ }\<%
\\
\>[0]\AgdaComment{--\ \ \ \ \ \ \ test-transact₂′\ :\ un\ (given\ hCC}\<%
\\
\>[0]\AgdaComment{--\ \ \ \ \ \ \ \ \ \ \ \ \ \ \ \ \ \ \ \ \ \ \ \ \ \ \ \ \ handle\ (given\ hThrow}\<%
\\
\>[0]\AgdaComment{--\ \ \ \ \ \ \ \ \ \ \ \ \ \ \ \ \ \ \ \ \ \ \ \ \ \ \ \ \ \ \ \ \ \ \ \ \ handle\ (given\ hSt}\<%
\\
\>[0]\AgdaComment{--\ \ \ \ \ \ \ \ \ \ \ \ \ \ \ \ \ \ \ \ \ \ \ \ \ \ \ \ \ \ \ \ \ \ \ \ \ \ \ \ \ \ \ \ \ handle\ (elaborate\ transact-elab₂\ transact′)\ \$\ 0)}\<%
\\
\>[0]\AgdaComment{--\ \ \ \ \ \ \ \ \ \ \ \ \ \ \ \ \ \ \ \ \ \ \ \ \ \ \ \ \ \ \ \ \ \ \ \ \$\ tt)}\<%
\\
\>[0]\AgdaComment{--\ \ \ \ \ \ \ \ \ \ \ \ \ \ \ \ \ \ \ \ \ \ \ \ \ \ \ \ \$\ tt)\ ≡\ just\ (2\ ,\ 2)}\<%
\\
\>[0]\AgdaComment{--\ \ \ \ \ \ \ test-transact₂′\ =\ refl}\<%
\\
\>[0]\AgdaComment{--\ }\<%
\\
\>[0]\AgdaComment{--\ \ \ \ \ \ \ test-transact₃′\ :\ un\ (given\ hSt}\<%
\\
\>[0]\AgdaComment{--\ \ \ \ \ \ \ \ \ \ \ \ \ \ \ \ \ \ \ \ \ \ \ \ \ \ \ \ handle\ (given\ hCC}\<%
\\
\>[0]\AgdaComment{--\ \ \ \ \ \ \ \ \ \ \ \ \ \ \ \ \ \ \ \ \ \ \ \ \ \ \ \ \ \ \ \ \ \ \ \ handle\ (given\ hThrow}\<%
\\
\>[0]\AgdaComment{--\ \ \ \ \ \ \ \ \ \ \ \ \ \ \ \ \ \ \ \ \ \ \ \ \ \ \ \ \ \ \ \ \ \ \ \ \ \ \ \ \ \ \ \ handle\ (elaborate\ transact-elab₃\ transact′)}\<%
\\
\>[0]\AgdaComment{--\ \ \ \ \ \ \ \ \ \ \ \ \ \ \ \ \ \ \ \ \ \ \ \ \ \ \ \ \ \ \ \ \ \ \ \ \ \ \ \ \ \ \ \$\ tt)}\<%
\\
\>[0]\AgdaComment{--\ \ \ \ \ \ \ \ \ \ \ \ \ \ \ \ \ \ \ \ \ \ \ \ \ \ \ \ \ \ \ \ \ \ \ \$\ tt)}\<%
\\
\>[0]\AgdaComment{--\ \ \ \ \ \ \ \ \ \ \ \ \ \ \ \ \ \ \ \ \ \ \ \ \ \ \ \$\ 0)\ ≡\ (just\ 2\ ,\ 2)}\<%
\\
\>[0]\AgdaComment{--\ \ \ \ \ \ \ test-transact₃′\ =\ refl}\<%
\\
\>[0]\AgdaComment{--\ }\<%
\\
\>[0]\AgdaComment{--\ }\<%
\\
\>[0]\AgdaComment{--\ \ \ \ \ \ \ transact″\ :\ ⦃\ wₛ\ :\ H\ ∼\ Lift\ State\ ▹\ H′\ ⦄\ ⦃\ wₜ\ :\ H\ ∼\ \ Lift\ Throw\ ▹\ H″\ ⦄\ ⦃\ w\ \ :\ H\ ∼\ Catch\ ▹\ H‴\ ⦄}\<%
\\
\>[0]\AgdaComment{--\ \ \ \ \ \ \ \ \ \ \ \ \ \ \ \ \ →\ Hefty\ H\ ℕ}\<%
\\
\>[0]\AgdaComment{--\ \ \ \ \ \ \ transact″\ =\ do}\<%
\\
\>[0]\AgdaComment{--\ \ \ \ \ \ \ \ \ ↑\ put\ 1}\<%
\\
\>[0]\AgdaComment{--\ \ \ \ \ \ \ \ \ ‵catch\ (do\ ↑\ put\ 2;\ ‵throwᴴ)\ (↑\ get)}\<%
\\
\>[0]\AgdaComment{--\ \ \ \ \ \ \ \ \ where\ open\ HeftyModule\ using\ (\AgdaUnderscore{}>>\AgdaUnderscore{})}\<%
\\
\>[0]\AgdaComment{--\ \ \ \ \ \ \ \ \ }\<%
\\
\>[0]\AgdaComment{--\ \ \ \ \ \ \ test-transact₂″\ :\ un\ (given\ hCC}\<%
\\
\>[0]\AgdaComment{--\ \ \ \ \ \ \ \ \ \ \ \ \ \ \ \ \ \ \ \ \ \ \ \ \ \ \ \ \ handle\ (given\ hThrow}\<%
\\
\>[0]\AgdaComment{--\ \ \ \ \ \ \ \ \ \ \ \ \ \ \ \ \ \ \ \ \ \ \ \ \ \ \ \ \ \ \ \ \ \ \ \ \ handle\ (given\ hSt}\<%
\\
\>[0]\AgdaComment{--\ \ \ \ \ \ \ \ \ \ \ \ \ \ \ \ \ \ \ \ \ \ \ \ \ \ \ \ \ \ \ \ \ \ \ \ \ \ \ \ \ \ \ \ \ handle\ (elaborate\ transact-elab₂\ transact″)\ \$\ 0)}\<%
\\
\>[0]\AgdaComment{--\ \ \ \ \ \ \ \ \ \ \ \ \ \ \ \ \ \ \ \ \ \ \ \ \ \ \ \ \ \ \ \ \ \ \ \ \$\ tt)}\<%
\\
\>[0]\AgdaComment{--\ \ \ \ \ \ \ \ \ \ \ \ \ \ \ \ \ \ \ \ \ \ \ \ \ \ \ \ \$\ tt)\ ≡\ just\ (1\ ,\ 1)}\<%
\\
\>[0]\AgdaComment{--\ \ \ \ \ \ \ test-transact₂″\ =\ refl}\<%
\\
\>[0]\AgdaComment{--\ }\<%
\\
\>[0]\AgdaComment{--\ \ \ \ \ \ \ test-transact₃″\ :\ un\ (given\ hSt}\<%
\\
\>[0]\AgdaComment{--\ \ \ \ \ \ \ \ \ \ \ \ \ \ \ \ \ \ \ \ \ \ \ \ \ \ \ \ handle\ (given\ hCC}\<%
\\
\>[0]\AgdaComment{--\ \ \ \ \ \ \ \ \ \ \ \ \ \ \ \ \ \ \ \ \ \ \ \ \ \ \ \ \ \ \ \ \ \ \ \ handle\ (given\ hThrow}\<%
\\
\>[0]\AgdaComment{--\ \ \ \ \ \ \ \ \ \ \ \ \ \ \ \ \ \ \ \ \ \ \ \ \ \ \ \ \ \ \ \ \ \ \ \ \ \ \ \ \ \ \ \ handle\ (elaborate\ transact-elab₃\ transact″)}\<%
\\
\>[0]\AgdaComment{--\ \ \ \ \ \ \ \ \ \ \ \ \ \ \ \ \ \ \ \ \ \ \ \ \ \ \ \ \ \ \ \ \ \ \ \ \ \ \ \ \ \ \ \$\ tt)}\<%
\\
\>[0]\AgdaComment{--\ \ \ \ \ \ \ \ \ \ \ \ \ \ \ \ \ \ \ \ \ \ \ \ \ \ \ \ \ \ \ \ \ \ \ \$\ tt)}\<%
\\
\>[0]\AgdaComment{--\ \ \ \ \ \ \ \ \ \ \ \ \ \ \ \ \ \ \ \ \ \ \ \ \ \ \ \$\ 0)\ ≡\ (just\ 2\ ,\ 2)}\<%
\\
\>[0]\AgdaComment{--\ \ \ \ \ \ \ test-transact₃″\ =\ refl}\<%
\end{code}

The sub/jump elaboration above is more involved than the scoped effect handler that we considered in \cref{sec:scoped-effects}.
However, the complicated encoding does not pollute the higher-order effect interface, and only turns up if we strictly want or need effect interaction.


\subsection{Logic Programming}

Following \citet{DBLP:conf/ppdp/SchrijversWDD14,WuSH14,YangPWBS22} we can define a non-deterministic choice operation (\af{\_‵or\_}) as an algebraic effect, to provide support for expressing the kind of non-deterministic search for solutions that is common in logic programming.
We can also define a \af{‵fail} operation which indicates that the search in the current branch was unsuccessful.
The effect signature for \ad{Choice} is given in \cref{fig:choice-sig}.
The following smart constructors are the lifted higher-order counterparts to the \af{‵or} and \af{‵fail} operations:
\begin{code}[hide]%
\>[0][@{}l@{\AgdaIndent{1}}]%
\>[2]\AgdaKeyword{module}\AgdaSpace{}%
\AgdaModule{ChoiceModule}\AgdaSpace{}%
\AgdaKeyword{where}\<%
\\
\>[2][@{}l@{\AgdaIndent{0}}]%
\>[4]\AgdaKeyword{open}\AgdaSpace{}%
\AgdaModule{Abbreviation}\<%
\\
%
\>[4]\AgdaKeyword{open}\AgdaSpace{}%
\AgdaModule{Algᴴ}\<%
\\
%
\>[4]\AgdaKeyword{open}\AgdaSpace{}%
\AgdaModule{ElabModule}\<%
\\
\>[0]\AgdaComment{--\ \ \ \ open\ Elab}\<%
\end{code}
\begin{figure}
\begin{minipage}{0.495\linewidth}
\begin{code}%
\>[0][@{}l@{\AgdaIndent{1}}]%
\>[4]\AgdaKeyword{data}\AgdaSpace{}%
\AgdaDatatype{ChoiceOp}\AgdaSpace{}%
\AgdaSymbol{:}\AgdaSpace{}%
\AgdaPrimitive{Set}\AgdaSpace{}%
\AgdaKeyword{where}\<%
\\
\>[4][@{}l@{\AgdaIndent{0}}]%
\>[6]\AgdaInductiveConstructor{or}%
\>[12]\AgdaSymbol{:}\AgdaSpace{}%
\AgdaDatatype{ChoiceOp}\<%
\\
%
\>[6]\AgdaInductiveConstructor{fail}%
\>[12]\AgdaSymbol{:}\AgdaSpace{}%
\AgdaDatatype{ChoiceOp}\<%
\end{code}
  \end{minipage}
  \hfill\vline\hfill
  \begin{minipage}{0.495\linewidth}
\begin{code}%
%
\>[4]\AgdaFunction{Choice}\AgdaSpace{}%
\AgdaSymbol{:}\AgdaSpace{}%
\AgdaRecord{Effect}\<%
\\
%
\>[4]\AgdaField{Op}%
\>[8]\AgdaFunction{Choice}\AgdaSpace{}%
\AgdaSymbol{=}\AgdaSpace{}%
\AgdaDatatype{ChoiceOp}\<%
\\
%
\>[4]\AgdaField{Ret}\AgdaSpace{}%
\AgdaFunction{Choice}\AgdaSpace{}%
\AgdaInductiveConstructor{or}\AgdaSpace{}%
\AgdaSymbol{=}\AgdaSpace{}%
\AgdaDatatype{Bool}\<%
\\
%
\>[4]\AgdaField{Ret}\AgdaSpace{}%
\AgdaFunction{Choice}\AgdaSpace{}%
\AgdaInductiveConstructor{fail}\AgdaSpace{}%
\AgdaSymbol{=}\AgdaSpace{}%
\AgdaFunction{⊥}\<%
\end{code}
\end{minipage}
\caption{Effect signature of the choice effect}
\label{fig:choice-sig}
\end{figure}
\begin{code}[hide]%
%
\>[4]\AgdaFunction{‵fail}\AgdaSpace{}%
\AgdaSymbol{:}\AgdaSpace{}%
\AgdaSymbol{⦃}\AgdaSpace{}%
\AgdaFunction{Choice}\AgdaSpace{}%
\AgdaOperator{\AgdaFunction{≲}}\AgdaSpace{}%
\AgdaGeneralizable{Δ}\AgdaSpace{}%
\AgdaSymbol{⦄}\AgdaSpace{}%
\AgdaSymbol{→}\AgdaSpace{}%
\AgdaDatatype{Free}\AgdaSpace{}%
\AgdaGeneralizable{Δ}\AgdaSpace{}%
\AgdaGeneralizable{A}\<%
\\
%
\>[4]\AgdaComment{--\ \AgdaUnderscore{}‵or\AgdaUnderscore{}\ :\ ⦃\ Δ\ ∼\ Choice\ ▸\ Δ′\ ⦄\ →\ Free\ Δ\ A\ →\ Free\ Δ\ A\ →\ Free\ Δ\ A}\<%
\end{code}
\begin{code}[hide]%
%
\>[4]\AgdaComment{--\ \AgdaUnderscore{}‵or\AgdaUnderscore{}\ ⦃\ w\ ⦄\ m₁\ m₂\ =\ impure\ (inj▸ₗ\ or)\ ((if\AgdaUnderscore{}then\ m₁\ else\ m₂)\ ∘\ proj-ret▸ₗ\ ⦃\ w\ ⦄)}\<%
\\
%
\>[4]\AgdaFunction{‵fail}\AgdaSpace{}%
\AgdaSymbol{⦃}\AgdaSpace{}%
\AgdaBound{w}\AgdaSpace{}%
\AgdaSymbol{⦄}\AgdaSpace{}%
\AgdaSymbol{=}\AgdaSpace{}%
\AgdaInductiveConstructor{impure}\<%
\\
\>[4][@{}l@{\AgdaIndent{0}}]%
\>[6]\AgdaSymbol{(}\AgdaFunction{inj}\AgdaSpace{}%
\AgdaSymbol{(}\AgdaInductiveConstructor{fail}\AgdaSpace{}%
\AgdaOperator{\AgdaInductiveConstructor{,}}\AgdaSpace{}%
\AgdaSymbol{λ}\AgdaSpace{}%
\AgdaSymbol{()))}\<%
\\
%
\>[6]\AgdaComment{--\ (inj▸ₗ\ fail\ ,\ ⊥-elim\ ∘\ proj-ret▸ₗ\ ⦃\ w\ ⦄)}\<%
\end{code}
\begin{code}[hide]%
%
\>[4]\AgdaKeyword{module}\AgdaSpace{}%
\AgdaModule{\AgdaUnderscore{}}\AgdaSpace{}%
\AgdaKeyword{where}\<%
\\
\>[4][@{}l@{\AgdaIndent{0}}]%
\>[6]\AgdaKeyword{open}\AgdaSpace{}%
\AgdaModule{FreeModule}\AgdaSpace{}%
\AgdaKeyword{using}\AgdaSpace{}%
\AgdaSymbol{(}\AgdaOperator{\AgdaFunction{\AgdaUnderscore{}𝓑\AgdaUnderscore{}}}\AgdaSymbol{;}\AgdaSpace{}%
\AgdaOperator{\AgdaFunction{\AgdaUnderscore{}>>\AgdaUnderscore{}}}\AgdaSymbol{)}\<%
\\
%
\>[6]\AgdaKeyword{open}\AgdaSpace{}%
\AgdaModule{ElabModule}\<%
\\
%
\\[\AgdaEmptyExtraSkip]%
%
\>[6]\AgdaKeyword{private}\AgdaSpace{}%
\AgdaOperator{\AgdaFunction{\AgdaUnderscore{}>>=\AgdaUnderscore{}}}\AgdaSpace{}%
\AgdaSymbol{=}\AgdaSpace{}%
\AgdaOperator{\AgdaFunction{\AgdaUnderscore{}𝓑\AgdaUnderscore{}}}\<%
\\
%
\\[\AgdaEmptyExtraSkip]%
%
\>[6]\AgdaFunction{hChoice}\AgdaSpace{}%
\AgdaSymbol{:}\AgdaSpace{}%
\AgdaOperator{\AgdaRecord{⟨}}\AgdaSpace{}%
\AgdaGeneralizable{A}\AgdaSpace{}%
\AgdaOperator{\AgdaRecord{!}}\AgdaSpace{}%
\AgdaFunction{Choice}\AgdaSpace{}%
\AgdaOperator{\AgdaRecord{⇒}}\AgdaSpace{}%
\AgdaRecord{⊤}\AgdaSpace{}%
\AgdaOperator{\AgdaRecord{⇒}}\AgdaSpace{}%
\AgdaDatatype{List}\AgdaSpace{}%
\AgdaGeneralizable{A}\AgdaSpace{}%
\AgdaOperator{\AgdaRecord{!}}\AgdaSpace{}%
\AgdaGeneralizable{Δ}\AgdaSpace{}%
\AgdaOperator{\AgdaRecord{⟩}}\<%
\\
%
\>[6]\AgdaField{ret}\AgdaSpace{}%
\AgdaFunction{hChoice}\AgdaSpace{}%
\AgdaBound{a}\AgdaSpace{}%
\AgdaSymbol{\AgdaUnderscore{}}\AgdaSpace{}%
\AgdaSymbol{=}\AgdaSpace{}%
\AgdaInductiveConstructor{pure}\AgdaSpace{}%
\AgdaSymbol{(}\AgdaBound{a}\AgdaSpace{}%
\AgdaOperator{\AgdaInductiveConstructor{∷}}\AgdaSpace{}%
\AgdaInductiveConstructor{[]}\AgdaSymbol{)}\<%
\\
%
\>[6]\AgdaField{hdl}\AgdaSpace{}%
\AgdaFunction{hChoice}\AgdaSpace{}%
\AgdaSymbol{(}\AgdaInductiveConstructor{or}\AgdaSpace{}%
\AgdaOperator{\AgdaInductiveConstructor{,}}\AgdaSpace{}%
\AgdaBound{k}\AgdaSymbol{)}\AgdaSpace{}%
\AgdaBound{p}\AgdaSpace{}%
\AgdaSymbol{=}\AgdaSpace{}%
\AgdaKeyword{do}\<%
\\
\>[6][@{}l@{\AgdaIndent{0}}]%
\>[8]\AgdaBound{l₁}\AgdaSpace{}%
\AgdaOperator{\AgdaFunction{←}}\AgdaSpace{}%
\AgdaBound{k}\AgdaSpace{}%
\AgdaInductiveConstructor{true}%
\>[22]\AgdaBound{p}\<%
\\
%
\>[8]\AgdaBound{l₂}\AgdaSpace{}%
\AgdaOperator{\AgdaFunction{←}}\AgdaSpace{}%
\AgdaBound{k}\AgdaSpace{}%
\AgdaInductiveConstructor{false}%
\>[22]\AgdaBound{p}\<%
\\
%
\>[8]\AgdaInductiveConstructor{pure}\AgdaSpace{}%
\AgdaSymbol{(}\AgdaBound{l₁}\AgdaSpace{}%
\AgdaOperator{\AgdaFunction{++}}\AgdaSpace{}%
\AgdaBound{l₂}\AgdaSymbol{)}\<%
\\
%
\>[6]\AgdaField{hdl}\AgdaSpace{}%
\AgdaFunction{hChoice}\AgdaSpace{}%
\AgdaSymbol{(}\AgdaInductiveConstructor{fail}\AgdaSpace{}%
\AgdaOperator{\AgdaInductiveConstructor{,}}\AgdaSpace{}%
\AgdaBound{k}\AgdaSymbol{)}\AgdaSpace{}%
\AgdaSymbol{\AgdaUnderscore{}}\AgdaSpace{}%
\AgdaSymbol{=}\AgdaSpace{}%
\AgdaInductiveConstructor{pure}\AgdaSpace{}%
\AgdaInductiveConstructor{[]}\<%
\end{code}
\begin{figure}
  \begin{minipage}{0.495\linewidth}
\begin{code}%
%
\>[6]\AgdaKeyword{data}\AgdaSpace{}%
\AgdaDatatype{OnceOp}\AgdaSpace{}%
\AgdaSymbol{⦃}\AgdaSpace{}%
\AgdaBound{u}\AgdaSpace{}%
\AgdaSymbol{:}\AgdaSpace{}%
\AgdaRecord{Univ}\AgdaSpace{}%
\AgdaSymbol{⦄}\AgdaSpace{}%
\AgdaSymbol{:}\AgdaSpace{}%
\AgdaPrimitive{Set}\AgdaSpace{}%
\AgdaKeyword{where}\<%
\\
\>[6][@{}l@{\AgdaIndent{0}}]%
\>[8]\AgdaInductiveConstructor{once}\AgdaSpace{}%
\AgdaSymbol{:}\AgdaSpace{}%
\AgdaSymbol{\{}\AgdaBound{t}\AgdaSpace{}%
\AgdaSymbol{:}\AgdaSpace{}%
\AgdaField{Type}\AgdaSymbol{\}}\AgdaSpace{}%
\AgdaSymbol{→}\AgdaSpace{}%
\AgdaDatatype{OnceOp}\<%
\end{code}
\end{minipage}
\hfill\vline\hfill
\begin{minipage}{0.495\linewidth}
\begin{code}%
%
\>[6]\AgdaFunction{Once}\AgdaSpace{}%
\AgdaSymbol{:}\AgdaSpace{}%
\AgdaSymbol{⦃}\AgdaSpace{}%
\AgdaBound{u}\AgdaSpace{}%
\AgdaSymbol{:}\AgdaSpace{}%
\AgdaRecord{Univ}\AgdaSpace{}%
\AgdaSymbol{⦄}\AgdaSpace{}%
\AgdaSymbol{→}\AgdaSpace{}%
\AgdaRecord{Effectᴴ}\<%
\\
%
\>[6]\AgdaField{Opᴴ}%
\>[13]\AgdaFunction{Once}%
\>[27]\AgdaSymbol{=}\AgdaSpace{}%
\AgdaDatatype{OnceOp}\<%
\\
%
\>[6]\AgdaField{Retᴴ}%
\>[13]\AgdaFunction{Once}\AgdaSpace{}%
\AgdaSymbol{(}\AgdaInductiveConstructor{once}\AgdaSpace{}%
\AgdaSymbol{\{}\AgdaBound{t}\AgdaSymbol{\})}\AgdaSpace{}%
\AgdaSymbol{=}\AgdaSpace{}%
\AgdaOperator{\AgdaField{⟦}}\AgdaSpace{}%
\AgdaBound{t}\AgdaSpace{}%
\AgdaOperator{\AgdaField{⟧ᵀ}}\<%
\\
%
\>[6]\AgdaField{Fork}%
\>[13]\AgdaFunction{Once}\AgdaSpace{}%
\AgdaSymbol{(}\AgdaInductiveConstructor{once}\AgdaSpace{}%
\AgdaSymbol{\{}\AgdaBound{t}\AgdaSymbol{\})}\AgdaSpace{}%
\AgdaSymbol{=}\AgdaSpace{}%
\AgdaRecord{⊤}\<%
\\
%
\>[6]\AgdaField{Ty}%
\>[13]\AgdaFunction{Once}\AgdaSpace{}%
\AgdaSymbol{\{}\AgdaInductiveConstructor{once}\AgdaSpace{}%
\AgdaSymbol{\{}\AgdaBound{t}\AgdaSymbol{\}\}}\AgdaSpace{}%
\AgdaSymbol{\AgdaUnderscore{}}\AgdaSpace{}%
\AgdaSymbol{=}\AgdaSpace{}%
\AgdaOperator{\AgdaField{⟦}}\AgdaSpace{}%
\AgdaBound{t}\AgdaSpace{}%
\AgdaOperator{\AgdaField{⟧ᵀ}}\<%
\end{code}
\end{minipage}
\caption{Higher-order effect signature of the once effect}
\label{fig:once-ho-sig}
\end{figure}
\begin{code}%
%
\>[6]\AgdaOperator{\AgdaFunction{\AgdaUnderscore{}‵orᴴ\AgdaUnderscore{}}}%
\>[14]\AgdaSymbol{:}\AgdaSpace{}%
\AgdaSymbol{⦃}\AgdaSpace{}%
\AgdaFunction{Lift}\AgdaSpace{}%
\AgdaFunction{Choice}\AgdaSpace{}%
\AgdaOperator{\AgdaFunction{≲ᴴ}}\AgdaSpace{}%
\AgdaGeneralizable{H}\AgdaSpace{}%
\AgdaSymbol{⦄}\AgdaSpace{}%
\AgdaSymbol{→}\AgdaSpace{}%
\AgdaDatatype{Hefty}\AgdaSpace{}%
\AgdaGeneralizable{H}\AgdaSpace{}%
\AgdaGeneralizable{A}\AgdaSpace{}%
\AgdaSymbol{→}\AgdaSpace{}%
\AgdaDatatype{Hefty}\AgdaSpace{}%
\AgdaGeneralizable{H}\AgdaSpace{}%
\AgdaGeneralizable{A}%
\>[62]\AgdaSymbol{→}\AgdaSpace{}%
\AgdaDatatype{Hefty}\AgdaSpace{}%
\AgdaGeneralizable{H}\AgdaSpace{}%
\AgdaGeneralizable{A}\<%
\\
%
\>[6]\AgdaFunction{‵failᴴ}%
\>[14]\AgdaSymbol{:}\AgdaSpace{}%
\AgdaSymbol{⦃}\AgdaSpace{}%
\AgdaFunction{Lift}\AgdaSpace{}%
\AgdaFunction{Choice}\AgdaSpace{}%
\AgdaOperator{\AgdaFunction{≲ᴴ}}\AgdaSpace{}%
\AgdaGeneralizable{H}\AgdaSpace{}%
\AgdaSymbol{⦄}%
\>[62]\AgdaSymbol{→}\AgdaSpace{}%
\AgdaDatatype{Hefty}\AgdaSpace{}%
\AgdaGeneralizable{H}\AgdaSpace{}%
\AgdaGeneralizable{A}\<%
\end{code}
\begin{code}[hide]%
%
\>[6]\AgdaOperator{\AgdaFunction{\AgdaUnderscore{}‵orᴴ\AgdaUnderscore{}}}\AgdaSpace{}%
\AgdaSymbol{⦃}\AgdaSpace{}%
\AgdaBound{w}\AgdaSpace{}%
\AgdaSymbol{⦄}\AgdaSpace{}%
\AgdaBound{m₁}\AgdaSpace{}%
\AgdaBound{m₂}\AgdaSpace{}%
\AgdaSymbol{=}\AgdaSpace{}%
\AgdaSymbol{(}\AgdaOperator{\AgdaFunction{↑}}\AgdaSpace{}%
\AgdaInductiveConstructor{or}\AgdaSymbol{)}\AgdaSpace{}%
\AgdaOperator{\AgdaFunction{𝓑'}}\AgdaSpace{}%
\AgdaSymbol{(}\AgdaOperator{\AgdaFunction{if\AgdaUnderscore{}then}}\AgdaSpace{}%
\AgdaBound{m₁}\AgdaSpace{}%
\AgdaOperator{\AgdaFunction{else}}\AgdaSpace{}%
\AgdaBound{m₂}\AgdaSymbol{)}\<%
\\
\>[6][@{}l@{\AgdaIndent{0}}]%
\>[8]\AgdaKeyword{where}\AgdaSpace{}%
\AgdaKeyword{open}\AgdaSpace{}%
\AgdaModule{HeftyModule}\AgdaSpace{}%
\AgdaKeyword{renaming}\AgdaSpace{}%
\AgdaSymbol{(}\AgdaOperator{\AgdaFunction{\AgdaUnderscore{}𝓑\AgdaUnderscore{}}}\AgdaSpace{}%
\AgdaSymbol{to}\AgdaSpace{}%
\AgdaOperator{\AgdaFunction{\AgdaUnderscore{}𝓑'\AgdaUnderscore{}}}\AgdaSymbol{)}\<%
\\
%
\\[\AgdaEmptyExtraSkip]%
%
\>[6]\AgdaFunction{‵failᴴ}\AgdaSpace{}%
\AgdaSymbol{⦃}\AgdaSpace{}%
\AgdaBound{w}\AgdaSpace{}%
\AgdaSymbol{⦄}\AgdaSpace{}%
\AgdaSymbol{=}\AgdaSpace{}%
\AgdaSymbol{(}\AgdaOperator{\AgdaFunction{↑}}\AgdaSpace{}%
\AgdaInductiveConstructor{fail}\AgdaSymbol{)}\AgdaSpace{}%
\AgdaOperator{\AgdaFunction{𝓑'}}\AgdaSpace{}%
\AgdaFunction{⊥-elim}\<%
\\
\>[6][@{}l@{\AgdaIndent{0}}]%
\>[8]\AgdaKeyword{where}\AgdaSpace{}%
\AgdaKeyword{open}\AgdaSpace{}%
\AgdaModule{HeftyModule}\AgdaSpace{}%
\AgdaKeyword{renaming}\AgdaSpace{}%
\AgdaSymbol{(}\AgdaOperator{\AgdaFunction{\AgdaUnderscore{}𝓑\AgdaUnderscore{}}}\AgdaSpace{}%
\AgdaSymbol{to}\AgdaSpace{}%
\AgdaOperator{\AgdaFunction{\AgdaUnderscore{}𝓑'\AgdaUnderscore{}}}\AgdaSymbol{)}\<%
\\
%
\\[\AgdaEmptyExtraSkip]%
%
\>[6]\AgdaKeyword{module}\AgdaSpace{}%
\AgdaModule{\AgdaUnderscore{}}\AgdaSpace{}%
\AgdaSymbol{⦃}\AgdaSpace{}%
\AgdaBound{u}\AgdaSpace{}%
\AgdaSymbol{:}\AgdaSpace{}%
\AgdaRecord{Univ}\AgdaSpace{}%
\AgdaSymbol{⦄}\AgdaSpace{}%
\AgdaKeyword{where}\<%
\end{code}
A useful operator for cutting non-deterministic search short when a solution is found is the \af{‵once} operator.
The \af{‵once} operator, whose higher-order effect signature is given in \cref{fig:once-ho-sig}, is not an algebraic effect, but a scoped (and thus higher-order) effect.
\begin{code}%
\>[6][@{}l@{\AgdaIndent{1}}]%
\>[7]\AgdaFunction{‵once}\AgdaSpace{}%
\AgdaSymbol{:}\AgdaSpace{}%
\AgdaSymbol{⦃}\AgdaSpace{}%
\AgdaBound{w}\AgdaSpace{}%
\AgdaSymbol{:}\AgdaSpace{}%
\AgdaFunction{Once}\AgdaSpace{}%
\AgdaOperator{\AgdaFunction{≲ᴴ}}\AgdaSpace{}%
\AgdaGeneralizable{H}\AgdaSpace{}%
\AgdaSymbol{⦄}\AgdaSpace{}%
\AgdaSymbol{\{}\AgdaBound{t}\AgdaSpace{}%
\AgdaSymbol{:}\AgdaSpace{}%
\AgdaField{Type}\AgdaSymbol{\}}\AgdaSpace{}%
\AgdaSymbol{→}\AgdaSpace{}%
\AgdaDatatype{Hefty}\AgdaSpace{}%
\AgdaGeneralizable{H}\AgdaSpace{}%
\AgdaOperator{\AgdaField{⟦}}\AgdaSpace{}%
\AgdaBound{t}\AgdaSpace{}%
\AgdaOperator{\AgdaField{⟧ᵀ}}\AgdaSpace{}%
\AgdaSymbol{→}\AgdaSpace{}%
\AgdaDatatype{Hefty}\AgdaSpace{}%
\AgdaGeneralizable{H}\AgdaSpace{}%
\AgdaOperator{\AgdaField{⟦}}\AgdaSpace{}%
\AgdaBound{t}\AgdaSpace{}%
\AgdaOperator{\AgdaField{⟧ᵀ}}\<%
\end{code}
\begin{code}[hide]%
%
\>[7]\AgdaFunction{‵once}\AgdaSpace{}%
\AgdaSymbol{⦃}\AgdaSpace{}%
\AgdaBound{w}\AgdaSpace{}%
\AgdaSymbol{⦄}\AgdaSpace{}%
\AgdaSymbol{\{}\AgdaBound{t}\AgdaSymbol{\}}\AgdaSpace{}%
\AgdaBound{b}\AgdaSpace{}%
\AgdaSymbol{=}\AgdaSpace{}%
\AgdaInductiveConstructor{impure}\<%
\\
\>[7][@{}l@{\AgdaIndent{0}}]%
\>[9]\AgdaSymbol{(}\AgdaFunction{injᴴ}\AgdaSpace{}%
\AgdaSymbol{\{}\AgdaArgument{M}\AgdaSpace{}%
\AgdaSymbol{=}\AgdaSpace{}%
\AgdaDatatype{Hefty}\AgdaSpace{}%
\AgdaSymbol{\AgdaUnderscore{}\}}\AgdaSpace{}%
\AgdaSymbol{(}\AgdaInductiveConstructor{once}\AgdaSpace{}%
\AgdaOperator{\AgdaInductiveConstructor{,}}\AgdaSpace{}%
\AgdaInductiveConstructor{pure}\AgdaSpace{}%
\AgdaOperator{\AgdaInductiveConstructor{,}}\AgdaSpace{}%
\AgdaSymbol{λ}\AgdaSpace{}%
\AgdaBound{\AgdaUnderscore{}}\AgdaSpace{}%
\AgdaSymbol{→}\AgdaSpace{}%
\AgdaBound{b}\AgdaSymbol{))}\<%
\\
%
\\[\AgdaEmptyExtraSkip]%
%
\>[6]\AgdaKeyword{module}\AgdaSpace{}%
\AgdaModule{\AgdaUnderscore{}}\AgdaSpace{}%
\AgdaSymbol{⦃}\AgdaSpace{}%
\AgdaBound{u}\AgdaSpace{}%
\AgdaSymbol{:}\AgdaSpace{}%
\AgdaRecord{Univ}\AgdaSpace{}%
\AgdaSymbol{⦄}\AgdaSpace{}%
\AgdaSymbol{⦃}\AgdaSpace{}%
\AgdaBound{w}\AgdaSpace{}%
\AgdaSymbol{:}\AgdaSpace{}%
\AgdaFunction{Choice}\AgdaSpace{}%
\AgdaOperator{\AgdaFunction{≲}}\AgdaSpace{}%
\AgdaGeneralizable{Δ}\AgdaSpace{}%
\AgdaSymbol{⦄}\AgdaSpace{}%
\AgdaKeyword{where}\<%
\end{code}
We can define the meaning of the \af{once} operator as the following elaboration:
\begin{code}%
\>[6][@{}l@{\AgdaIndent{1}}]%
\>[8]\AgdaFunction{eOnce}\AgdaSpace{}%
\AgdaSymbol{:}\AgdaSpace{}%
\AgdaSymbol{⦃}\AgdaSpace{}%
\AgdaFunction{Choice}\AgdaSpace{}%
\AgdaOperator{\AgdaPostulate{≲⅋}}\AgdaSpace{}%
\AgdaBound{Δ}\AgdaSpace{}%
\AgdaSymbol{⦄}\AgdaSpace{}%
\AgdaSymbol{→}\AgdaSpace{}%
\AgdaFunction{Elaboration}\AgdaSpace{}%
\AgdaFunction{Once}\AgdaSpace{}%
\AgdaBound{Δ}\<%
\\
%
\>[8]\AgdaField{alg}\AgdaSpace{}%
\AgdaFunction{eOnce}\AgdaSpace{}%
\AgdaSymbol{(}\AgdaInductiveConstructor{once}\AgdaSpace{}%
\AgdaOperator{\AgdaInductiveConstructor{,}}\AgdaSpace{}%
\AgdaBound{k}\AgdaSpace{}%
\AgdaOperator{\AgdaInductiveConstructor{,}}\AgdaSpace{}%
\AgdaBound{ψ}\AgdaSymbol{)}\AgdaSpace{}%
\AgdaSymbol{=}\AgdaSpace{}%
\AgdaKeyword{do}\<%
\\
\>[8][@{}l@{\AgdaIndent{0}}]%
\>[10]\AgdaBound{l}\AgdaSpace{}%
\AgdaOperator{\AgdaFunction{←}}\AgdaSpace{}%
\AgdaOperator{\AgdaFunction{♯}}\AgdaSpace{}%
\AgdaSymbol{((}\AgdaOperator{\AgdaFunction{given}}\AgdaSpace{}%
\AgdaFunction{hChoice}\AgdaSpace{}%
\AgdaOperator{\AgdaFunction{handle}}\AgdaSpace{}%
\AgdaSymbol{(}\AgdaBound{ψ}\AgdaSpace{}%
\AgdaInductiveConstructor{tt}\AgdaSymbol{))}\AgdaSpace{}%
\AgdaInductiveConstructor{tt}\AgdaSymbol{)}\<%
\\
%
\>[10]\AgdaFunction{maybe}\AgdaSpace{}%
\AgdaBound{k}\AgdaSpace{}%
\AgdaFunction{‵fail}\AgdaSpace{}%
\AgdaSymbol{(}\AgdaFunction{head}\AgdaSpace{}%
\AgdaBound{l}\AgdaSymbol{)}\<%
\end{code}
\begin{code}[hide]%
%
\>[10]\AgdaKeyword{where}\AgdaSpace{}%
\AgdaKeyword{instance}\AgdaSpace{}%
\AgdaSymbol{\AgdaUnderscore{}}\AgdaSpace{}%
\AgdaSymbol{=}\AgdaSpace{}%
\AgdaSymbol{\AgdaUnderscore{}}\AgdaSpace{}%
\AgdaOperator{\AgdaInductiveConstructor{,}}\AgdaSpace{}%
\AgdaFunction{∙-comm}\AgdaSpace{}%
\AgdaSymbol{(}\AgdaBound{w}\AgdaSpace{}%
\AgdaSymbol{.}\AgdaField{proj₂}\AgdaSymbol{)}\<%
\end{code}
The elaboration runs the branch (\ab{ψ}) of \ac{once} under the \af{hChoice} handler to compute a list of all solutions of \ab{ψ}.
It then tries to choose the first solution and pass that to the continuation \ab{k}.
If the branch has no solutions, we fail.
%
Under a strict evaluation order, the elaboration computes all possible solutions which is doing more work than needed.
Agda 2.6.2.2 does not have a specified evaluation strategy, but does compile to Haskell which is lazy.
In Haskell, the solutions would be lazily computed, such that the \ac{once} operator cuts search short as intended.

\begin{code}[hide]%
\>[0]\AgdaComment{--\ \ \ \ \ module\ OnceExample\ where}\<%
\\
\>[0]\AgdaComment{--\ \ \ \ \ \ \ open\ import\ Data.Nat\ using\ (ℕ;\ \AgdaUnderscore{}≡ᵇ\AgdaUnderscore{})}\<%
\\
\>[0]\AgdaComment{--\ \ \ \ \ \ \ open\ HeftyModule\ using\ (\AgdaUnderscore{}𝓑\AgdaUnderscore{};\ \AgdaUnderscore{}>>\AgdaUnderscore{})}\<%
\\
\>[0]\AgdaComment{--\ \ \ \ \ \ \ open\ ElabModule}\<%
\\
\>[0]\AgdaComment{--\ }\<%
\\
\>[0]\AgdaComment{--\ \ \ \ \ \ \ private\ \AgdaUnderscore{}>>=\AgdaUnderscore{}\ =\ \AgdaUnderscore{}𝓑\AgdaUnderscore{}}\<%
\\
\>[0]\AgdaComment{--\ }\<%
\\
\>[0]\AgdaComment{--\ \ \ \ \ \ \ data\ OnceType\ :\ Set\ where}\<%
\\
\>[0]\AgdaComment{--\ \ \ \ \ \ \ \ \ num\ \ \ :\ OnceType}\<%
\\
\>[0]\AgdaComment{--\ \ \ \ \ \ \ \ \ unit\ \ :\ OnceType}\<%
\\
\>[0]\AgdaComment{--\ }\<%
\\
\>[0]\AgdaComment{--\ \ \ \ \ \ \ private\ instance}\<%
\\
\>[0]\AgdaComment{--\ \ \ \ \ \ \ \ \ OnceUniv\ :\ Univ}\<%
\\
\>[0]\AgdaComment{--\ \ \ \ \ \ \ \ \ Ty\ ⦃\ OnceUniv\ ⦄\ =\ OnceType}\<%
\\
\>[0]\AgdaComment{--\ \ \ \ \ \ \ \ \ ⟦\AgdaUnderscore{}⟧ᵀ\ ⦃\ OnceUniv\ ⦄\ num\ =\ ℕ}\<%
\\
\>[0]\AgdaComment{--\ \ \ \ \ \ \ \ \ ⟦\AgdaUnderscore{}⟧ᵀ\ ⦃\ OnceUniv\ ⦄\ unit\ =\ ⊤}\<%
\\
\>[0]\AgdaComment{--\ }\<%
\\
\>[0]\AgdaComment{--\ \ \ \ \ \ \ ex-0or1\ :\ Hefty\ (Lift\ Choice\ ∔\ Once\ ∔\ Lift\ Nil)\ ℕ}\<%
\\
\>[0]\AgdaComment{--\ \ \ \ \ \ \ ex-0or1\ =\ (pure\ 0)\ ‵orᴴ\ (pure\ 1)}\<%
\\
\>[0]\AgdaComment{--\ }\<%
\\
\>[0]\AgdaComment{--\ \ \ \ \ \ \ ex-fail-once\ :\ Hefty\ (Lift\ Choice\ ∔\ Once\ ∔\ Lift\ Nil)\ ℕ}\<%
\\
\>[0]\AgdaComment{--\ \ \ \ \ \ \ ex-fail-once\ =\ do}\<%
\\
\>[0]\AgdaComment{--\ \ \ \ \ \ \ \ \ r\ ←\ ‵once\ ex-0or1}\<%
\\
\>[0]\AgdaComment{--\ \ \ \ \ \ \ \ \ if\ r\ ≡ᵇ\ 0\ then\ ‵failᴴ\ else\ pure\ r}\<%
\\
\>[0]\AgdaComment{--\ \ \ \ \ \ \ \ \ \ \ \ \ \ \ \ \ \ \ \ \ \ \ \ \ \ \ \ \ \ \ \ \ \ \ \ \ \ \ \ \ }\<%
\\
\>[0]\AgdaComment{--\ \ \ \ \ \ \ once-elab\ :\ Elaboration\ (Lift\ Choice\ ∔\ Once\ ∔\ Lift\ Nil)\ (Choice\ ⊕\ Nil)}\<%
\\
\>[0]\AgdaComment{--\ \ \ \ \ \ \ once-elab\ =\ eLift\ ⋎\ eOnce\ ⋎\ eNil}\<%
\\
\>[0]\AgdaComment{--\ }\<%
\\
\>[0]\AgdaComment{--\ \ \ \ \ \ \ test-ex-0or1\ :\ un\ (given\ hChoice\ handle\ (elaborate\ once-elab\ ex-0or1)\ \$\ tt)\ ≡\ 0\ ∷\ 1\ ∷\ []}\<%
\\
\>[0]\AgdaComment{--\ \ \ \ \ \ \ test-ex-0or1\ =\ refl}\<%
\\
\>[0]\AgdaComment{--\ }\<%
\\
\>[0]\AgdaComment{--\ \ \ \ \ \ \ test-fail-once\ :\ un\ (given\ hChoice\ handle\ (elaborate\ once-elab\ ex-fail-once)\ \$\ tt)\ ≡\ []}\<%
\\
\>[0]\AgdaComment{--\ \ \ \ \ \ \ test-fail-once\ =\ refl}\<%
\end{code}


\subsection{Concurrency}

Finally, we consider how to define higher-order operations for concurrency, inspired by \citeauthor{YangPWBS22}'s~[\citeyear{YangPWBS22}] \emph{resumption monad}~\citep{Claessen99,Schmidt1986denotational,PirogG14} defined using scoped effects.
We summarize our encoding and compare it with the resumption monad. The goal is to define the two operations, whose higher-order effect signature is given in \cref{fig:concurrency-ho-sig}, and summarized by these smart constructors:
%
%Our goal is to define two higher-order operations:
%
\begin{code}[hide]%
\>[0][@{}l@{\AgdaIndent{1}}]%
\>[2]\AgdaKeyword{module}\AgdaSpace{}%
\AgdaModule{\AgdaUnderscore{}}\AgdaSpace{}%
\AgdaSymbol{⦃}\AgdaSpace{}%
\AgdaBound{u}\AgdaSpace{}%
\AgdaSymbol{:}\AgdaSpace{}%
\AgdaRecord{Univ}\AgdaSpace{}%
\AgdaSymbol{⦄}\AgdaSpace{}%
\AgdaKeyword{where}\<%
\\
\>[2][@{}l@{\AgdaIndent{0}}]%
\>[4]\AgdaKeyword{postulate}\<%
\end{code}
\begin{code}%
\>[4][@{}l@{\AgdaIndent{1}}]%
\>[6]\AgdaPostulate{‵spawn⅋}%
\>[16]\AgdaSymbol{:}\AgdaSpace{}%
\AgdaSymbol{\{}\AgdaBound{t}\AgdaSpace{}%
\AgdaSymbol{:}\AgdaSpace{}%
\AgdaField{Type}\AgdaSymbol{\}}\AgdaSpace{}%
\AgdaSymbol{→}\AgdaSpace{}%
\AgdaSymbol{(}\AgdaBound{m₁}\AgdaSpace{}%
\AgdaBound{m₂}\AgdaSpace{}%
\AgdaSymbol{:}\AgdaSpace{}%
\AgdaDatatype{Hefty}\AgdaSpace{}%
\AgdaGeneralizable{H}\AgdaSpace{}%
\AgdaOperator{\AgdaField{⟦}}\AgdaSpace{}%
\AgdaBound{t}\AgdaSpace{}%
\AgdaOperator{\AgdaField{⟧ᵀ}}\AgdaSymbol{)}%
\>[57]\AgdaSymbol{→}\AgdaSpace{}%
\AgdaDatatype{Hefty}\AgdaSpace{}%
\AgdaGeneralizable{H}\AgdaSpace{}%
\AgdaOperator{\AgdaField{⟦}}\AgdaSpace{}%
\AgdaBound{t}\AgdaSpace{}%
\AgdaOperator{\AgdaField{⟧ᵀ}}\<%
\\
%
\>[6]\AgdaPostulate{‵atomic⅋}%
\>[16]\AgdaSymbol{:}\AgdaSpace{}%
\AgdaSymbol{\{}\AgdaBound{t}\AgdaSpace{}%
\AgdaSymbol{:}\AgdaSpace{}%
\AgdaField{Type}\AgdaSymbol{\}}\AgdaSpace{}%
\AgdaSymbol{→}\AgdaSpace{}%
\AgdaDatatype{Hefty}\AgdaSpace{}%
\AgdaGeneralizable{H}\AgdaSpace{}%
\AgdaOperator{\AgdaField{⟦}}\AgdaSpace{}%
\AgdaBound{t}\AgdaSpace{}%
\AgdaOperator{\AgdaField{⟧ᵀ}}%
\>[57]\AgdaSymbol{→}\AgdaSpace{}%
\AgdaDatatype{Hefty}\AgdaSpace{}%
\AgdaGeneralizable{H}\AgdaSpace{}%
\AgdaOperator{\AgdaField{⟦}}\AgdaSpace{}%
\AgdaBound{t}\AgdaSpace{}%
\AgdaOperator{\AgdaField{⟧ᵀ}}\<%
\end{code}
%
The operation \af{‵spawn}~\ab{m₁}~\ab{m₂} spawns two threads that run concurrently, and returns the value produced by \ab{m₁} after both have finished.
The operation \af{‵atomic}~\ab{m} represents a block to be executed atomically; i.e., no other threads run before the block finishes executing.

We elaborate \ad{‵spawn} by interleaving the sub-trees of its computations.
To this end, we use a dedicated function which interleaves the operations in two trees and yields as output the value of the left input tree (the first computation parameter):
%
\begin{code}[hide]%
%
\>[2]\AgdaKeyword{module}\AgdaSpace{}%
\AgdaModule{\AgdaUnderscore{}}\AgdaSpace{}%
\AgdaSymbol{⦃}\AgdaSpace{}%
\AgdaBound{u}\AgdaSpace{}%
\AgdaSymbol{:}\AgdaSpace{}%
\AgdaRecord{Univ}\AgdaSpace{}%
\AgdaSymbol{⦄}\AgdaSpace{}%
\AgdaKeyword{where}\<%
\\
\>[2][@{}l@{\AgdaIndent{0}}]%
\>[4]\AgdaKeyword{open}\AgdaSpace{}%
\AgdaModule{CCModule}\<%
\\
%
\>[4]\AgdaKeyword{postulate}\<%
\end{code}
\begin{code}%
\>[4][@{}l@{\AgdaIndent{1}}]%
\>[6]\AgdaPostulate{interleaveₗ⅋}%
\>[20]\AgdaSymbol{:}%
\>[23]\AgdaSymbol{\{}\AgdaBound{Ref}\AgdaSpace{}%
\AgdaSymbol{:}\AgdaSpace{}%
\AgdaField{Type}\AgdaSpace{}%
\AgdaSymbol{→}\AgdaSpace{}%
\AgdaPrimitive{Set}\AgdaSymbol{\}}\AgdaSpace{}%
\AgdaSymbol{→}\AgdaSpace{}%
\AgdaDatatype{Free}\AgdaSpace{}%
\AgdaSymbol{(}\AgdaFunction{CC}\AgdaSpace{}%
\AgdaBound{Ref}\AgdaSpace{}%
\AgdaOperator{\AgdaFunction{⊕}}\AgdaSpace{}%
\AgdaGeneralizable{Δ}\AgdaSymbol{)}\AgdaSpace{}%
\AgdaGeneralizable{A}\AgdaSpace{}%
\AgdaSymbol{→}\AgdaSpace{}%
\AgdaDatatype{Free}\AgdaSpace{}%
\AgdaSymbol{(}\AgdaFunction{CC}\AgdaSpace{}%
\AgdaBound{Ref}\AgdaSpace{}%
\AgdaOperator{\AgdaFunction{⊕}}\AgdaSpace{}%
\AgdaGeneralizable{Δ}\AgdaSymbol{)}\AgdaSpace{}%
\AgdaGeneralizable{B}\<%
\\
%
\>[20]\AgdaSymbol{→}%
\>[23]\AgdaDatatype{Free}\AgdaSpace{}%
\AgdaSymbol{(}\AgdaFunction{CC}\AgdaSpace{}%
\AgdaBound{Ref}\AgdaSpace{}%
\AgdaOperator{\AgdaFunction{⊕}}\AgdaSpace{}%
\AgdaGeneralizable{Δ}\AgdaSymbol{)}\AgdaSpace{}%
\AgdaGeneralizable{A}\<%
\end{code}
%
\begin{code}[hide]%
%
\>[2]\AgdaKeyword{module}\AgdaSpace{}%
\AgdaModule{ResumptionModule}\AgdaSpace{}%
\AgdaKeyword{where}\<%
\\
%
\\[\AgdaEmptyExtraSkip]%
\>[2][@{}l@{\AgdaIndent{0}}]%
\>[4]\AgdaKeyword{module}\AgdaSpace{}%
\AgdaModule{\AgdaUnderscore{}}\AgdaSpace{}%
\AgdaKeyword{where}\<%
\\
\>[4][@{}l@{\AgdaIndent{0}}]%
\>[6]\AgdaKeyword{open}\AgdaSpace{}%
\AgdaModule{FreeModule}\<%
\\
%
\>[6]\AgdaKeyword{open}\AgdaSpace{}%
\AgdaModule{ElabModule}\<%
\\
%
\>[6]\AgdaKeyword{open}\AgdaSpace{}%
\AgdaModule{CCModule}\<%
\\
\>[0]\AgdaComment{--\ \ \ \ \ \ open\ Elab}\<%
\\
%
\\[\AgdaEmptyExtraSkip]%
%
\\[\AgdaEmptyExtraSkip]%
\>[0][@{}l@{\AgdaIndent{0}}]%
\>[6]\AgdaComment{--\ interleaving\ interleaves\ two\ trees,\ except\ for\ sub-scopes\ of\ atomic\ blocks}\<%
\\
%
\\[\AgdaEmptyExtraSkip]%
%
\>[6]\AgdaFunction{interleaveₗ}%
\>[2570I]\AgdaSymbol{:}\AgdaSpace{}%
\AgdaSymbol{⦃}\AgdaSpace{}%
\AgdaBound{u}\AgdaSpace{}%
\AgdaSymbol{:}\AgdaSpace{}%
\AgdaRecord{Univ}\AgdaSpace{}%
\AgdaSymbol{⦄}\AgdaSpace{}%
\AgdaSymbol{\{}\AgdaBound{Ref}\AgdaSpace{}%
\AgdaSymbol{:}\AgdaSpace{}%
\AgdaField{Type}\AgdaSpace{}%
\AgdaSymbol{→}\AgdaSpace{}%
\AgdaPrimitive{Set}\AgdaSymbol{\}}\AgdaSpace{}%
\AgdaComment{\{-⦃\ w\ :\ Δ\ ∼\ Choice\ ▸\ Δ′\ ⦄-\}}\<%
\\
\>[.][@{}l@{}]\<[2570I]%
\>[18]\AgdaSymbol{→}\AgdaSpace{}%
\AgdaDatatype{Free}\AgdaSpace{}%
\AgdaSymbol{(}\AgdaFunction{CC}\AgdaSpace{}%
\AgdaBound{Ref}\AgdaSpace{}%
\AgdaOperator{\AgdaFunction{⊕}}\AgdaSpace{}%
\AgdaGeneralizable{Δ}\AgdaSymbol{)}\AgdaSpace{}%
\AgdaGeneralizable{A}\AgdaSpace{}%
\AgdaSymbol{→}\AgdaSpace{}%
\AgdaDatatype{Free}\AgdaSpace{}%
\AgdaSymbol{(}\AgdaFunction{CC}\AgdaSpace{}%
\AgdaBound{Ref}\AgdaSpace{}%
\AgdaOperator{\AgdaFunction{⊕}}\AgdaSpace{}%
\AgdaGeneralizable{Δ}\AgdaSymbol{)}\AgdaSpace{}%
\AgdaGeneralizable{B}\AgdaSpace{}%
\AgdaSymbol{→}\AgdaSpace{}%
\AgdaDatatype{Free}\AgdaSpace{}%
\AgdaSymbol{(}\AgdaFunction{CC}\AgdaSpace{}%
\AgdaBound{Ref}\AgdaSpace{}%
\AgdaOperator{\AgdaFunction{⊕}}\AgdaSpace{}%
\AgdaGeneralizable{Δ}\AgdaSymbol{)}\AgdaSpace{}%
\AgdaGeneralizable{A}\<%
\\
%
\>[6]\AgdaFunction{interleaveₗ}\AgdaSpace{}%
\AgdaSymbol{(}\AgdaInductiveConstructor{pure}\AgdaSpace{}%
\AgdaBound{x}\AgdaSymbol{)}\AgdaSpace{}%
\AgdaSymbol{(}\AgdaInductiveConstructor{pure}\AgdaSpace{}%
\AgdaSymbol{\AgdaUnderscore{})}\AgdaSpace{}%
\AgdaSymbol{=}\AgdaSpace{}%
\AgdaInductiveConstructor{pure}\AgdaSpace{}%
\AgdaBound{x}\<%
\\
%
\>[6]\AgdaCatchallClause{\AgdaFunction{interleaveₗ}}\AgdaSpace{}%
\AgdaCatchallClause{\AgdaSymbol{(}}\AgdaCatchallClause{\AgdaInductiveConstructor{pure}}\AgdaSpace{}%
\AgdaCatchallClause{\AgdaBound{x}}\AgdaCatchallClause{\AgdaSymbol{)}}\AgdaSpace{}%
\AgdaCatchallClause{\AgdaBound{m₂}}\AgdaSpace{}%
\AgdaSymbol{=}\AgdaSpace{}%
\AgdaFunction{fmap}\AgdaSpace{}%
\AgdaSymbol{(λ}\AgdaSpace{}%
\AgdaBound{\AgdaUnderscore{}}\AgdaSpace{}%
\AgdaSymbol{→}\AgdaSpace{}%
\AgdaBound{x}\AgdaSymbol{)}\AgdaSpace{}%
\AgdaBound{m₂}\<%
\\
%
\>[6]\AgdaCatchallClause{\AgdaFunction{interleaveₗ}}\AgdaSpace{}%
\AgdaCatchallClause{\AgdaBound{m₁}}\AgdaSpace{}%
\AgdaCatchallClause{\AgdaSymbol{(}}\AgdaCatchallClause{\AgdaInductiveConstructor{pure}}\AgdaSpace{}%
\AgdaCatchallClause{\AgdaBound{x}}\AgdaCatchallClause{\AgdaSymbol{)}}\AgdaSpace{}%
\AgdaSymbol{=}\AgdaSpace{}%
\AgdaBound{m₁}\<%
\\
%
\>[6]\AgdaCatchallClause{\AgdaFunction{interleaveₗ}}\AgdaSpace{}%
\AgdaCatchallClause{\AgdaSymbol{(}}\AgdaCatchallClause{\AgdaInductiveConstructor{impure}}\AgdaSpace{}%
\AgdaCatchallClause{\AgdaSymbol{(}}\AgdaCatchallClause{\AgdaInductiveConstructor{inj₁}}\AgdaSpace{}%
\AgdaCatchallClause{\AgdaSymbol{(}}\AgdaCatchallClause{\AgdaInductiveConstructor{jump}}\AgdaSpace{}%
\AgdaCatchallClause{\AgdaBound{ref}}\AgdaSpace{}%
\AgdaCatchallClause{\AgdaBound{x}}\AgdaCatchallClause{\AgdaSymbol{)}}\AgdaSpace{}%
\AgdaCatchallClause{\AgdaOperator{\AgdaInductiveConstructor{,}}}\AgdaSpace{}%
\AgdaCatchallClause{\AgdaSymbol{\AgdaUnderscore{}))}}\AgdaSpace{}%
\AgdaCatchallClause{\AgdaBound{m₂}}\AgdaSpace{}%
\AgdaSymbol{=}\AgdaSpace{}%
\AgdaKeyword{do}\<%
\\
\>[6][@{}l@{\AgdaIndent{0}}]%
\>[8]\AgdaBound{m₂}\<%
\\
%
\>[8]\AgdaFunction{‵jump}\AgdaSpace{}%
\AgdaSymbol{⦃}\AgdaSpace{}%
\AgdaSymbol{\AgdaUnderscore{}}\AgdaSpace{}%
\AgdaSymbol{⦄}\AgdaSpace{}%
\AgdaSymbol{⦃}\AgdaSpace{}%
\AgdaFunction{≲-left}\AgdaSpace{}%
\AgdaSymbol{⦄}\AgdaSpace{}%
\AgdaBound{ref}\AgdaSpace{}%
\AgdaBound{x}\<%
\\
%
\>[6]\AgdaCatchallClause{\AgdaFunction{interleaveₗ}}\AgdaSpace{}%
\AgdaCatchallClause{\AgdaBound{m₁}}\AgdaSpace{}%
\AgdaCatchallClause{\AgdaSymbol{(}}\AgdaCatchallClause{\AgdaInductiveConstructor{impure}}\AgdaSpace{}%
\AgdaCatchallClause{\AgdaSymbol{(}}\AgdaCatchallClause{\AgdaInductiveConstructor{inj₁}}\AgdaSpace{}%
\AgdaCatchallClause{\AgdaSymbol{(}}\AgdaCatchallClause{\AgdaInductiveConstructor{jump}}\AgdaSpace{}%
\AgdaCatchallClause{\AgdaBound{ref}}\AgdaSpace{}%
\AgdaCatchallClause{\AgdaBound{x}}\AgdaCatchallClause{\AgdaSymbol{)}}\AgdaSpace{}%
\AgdaCatchallClause{\AgdaOperator{\AgdaInductiveConstructor{,}}}\AgdaSpace{}%
\AgdaCatchallClause{\AgdaSymbol{\AgdaUnderscore{}))}}\AgdaSpace{}%
\AgdaSymbol{=}\AgdaSpace{}%
\AgdaKeyword{do}\<%
\\
\>[6][@{}l@{\AgdaIndent{0}}]%
\>[8]\AgdaBound{m₁}\<%
\\
%
\>[8]\AgdaFunction{‵jump}\AgdaSpace{}%
\AgdaSymbol{⦃}\AgdaSpace{}%
\AgdaSymbol{\AgdaUnderscore{}}\AgdaSpace{}%
\AgdaSymbol{⦄}\AgdaSpace{}%
\AgdaSymbol{⦃}\AgdaSpace{}%
\AgdaFunction{≲-left}\AgdaSpace{}%
\AgdaSymbol{⦄}\AgdaSpace{}%
\AgdaBound{ref}\AgdaSpace{}%
\AgdaBound{x}\<%
\\
%
\>[6]\AgdaFunction{interleaveₗ}\AgdaSpace{}%
\AgdaSymbol{(}\AgdaInductiveConstructor{impure}\AgdaSpace{}%
\AgdaSymbol{(}\AgdaInductiveConstructor{inj₁}\AgdaSpace{}%
\AgdaInductiveConstructor{sub}\AgdaSpace{}%
\AgdaOperator{\AgdaInductiveConstructor{,}}\AgdaSpace{}%
\AgdaBound{k₁}\AgdaSymbol{))}\AgdaSpace{}%
\AgdaSymbol{(}\AgdaInductiveConstructor{impure}\AgdaSpace{}%
\AgdaSymbol{(}\AgdaInductiveConstructor{inj₁}\AgdaSpace{}%
\AgdaInductiveConstructor{sub}\AgdaSpace{}%
\AgdaOperator{\AgdaInductiveConstructor{,}}\AgdaSpace{}%
\AgdaBound{k₂}\AgdaSymbol{))}\AgdaSpace{}%
\AgdaSymbol{=}\<%
\\
\>[6][@{}l@{\AgdaIndent{0}}]%
\>[8]\AgdaInductiveConstructor{impure}\<%
\\
\>[8][@{}l@{\AgdaIndent{0}}]%
\>[10]\AgdaSymbol{(}\AgdaInductiveConstructor{inj₁}\AgdaSpace{}%
\AgdaInductiveConstructor{sub}\AgdaSpace{}%
\AgdaOperator{\AgdaInductiveConstructor{,}}\<%
\\
%
\>[10]\AgdaSymbol{(λ\{}\AgdaSpace{}%
\AgdaSymbol{(}\AgdaInductiveConstructor{inj₁}\AgdaSpace{}%
\AgdaBound{x}\AgdaSymbol{)}\AgdaSpace{}%
\AgdaSymbol{→}\AgdaSpace{}%
\AgdaBound{k₁}\AgdaSpace{}%
\AgdaSymbol{(}\AgdaInductiveConstructor{inj₁}\AgdaSpace{}%
\AgdaBound{x}\AgdaSymbol{)}\<%
\\
\>[10][@{}l@{\AgdaIndent{0}}]%
\>[12]\AgdaSymbol{;}%
\>[2679I]\AgdaSymbol{(}\AgdaInductiveConstructor{inj₂}\AgdaSpace{}%
\AgdaBound{y}\AgdaSymbol{)}\AgdaSpace{}%
\AgdaSymbol{→}\<%
\\
\>[.][@{}l@{}]\<[2679I]%
\>[14]\AgdaInductiveConstructor{impure}\<%
\\
\>[14][@{}l@{\AgdaIndent{0}}]%
\>[16]\AgdaSymbol{(}\AgdaInductiveConstructor{inj₁}\AgdaSpace{}%
\AgdaInductiveConstructor{sub}\AgdaSpace{}%
\AgdaOperator{\AgdaInductiveConstructor{,}}\<%
\\
%
\>[16]\AgdaSymbol{(λ\{}\AgdaSpace{}%
\AgdaSymbol{(}\AgdaInductiveConstructor{inj₁}\AgdaSpace{}%
\AgdaBound{x}\AgdaSymbol{)}\AgdaSpace{}%
\AgdaSymbol{→}\AgdaSpace{}%
\AgdaBound{k₂}\AgdaSpace{}%
\AgdaSymbol{(}\AgdaInductiveConstructor{inj₁}\AgdaSpace{}%
\AgdaBound{x}\AgdaSymbol{)}\AgdaSpace{}%
\AgdaOperator{\AgdaFunction{𝓑}}\AgdaSpace{}%
\AgdaSymbol{λ}\AgdaSpace{}%
\AgdaBound{\AgdaUnderscore{}}\AgdaSpace{}%
\AgdaSymbol{→}\AgdaSpace{}%
\AgdaBound{k₁}\AgdaSpace{}%
\AgdaSymbol{(}\AgdaInductiveConstructor{inj₂}\AgdaSpace{}%
\AgdaBound{y}\AgdaSymbol{)}\<%
\\
\>[16][@{}l@{\AgdaIndent{0}}]%
\>[18]\AgdaSymbol{;}\AgdaSpace{}%
\AgdaSymbol{(}\AgdaInductiveConstructor{inj₂}\AgdaSpace{}%
\AgdaBound{z}\AgdaSymbol{)}\AgdaSpace{}%
\AgdaSymbol{→}\AgdaSpace{}%
\AgdaFunction{interleaveₗ}\AgdaSpace{}%
\AgdaSymbol{(}\AgdaBound{k₁}\AgdaSpace{}%
\AgdaSymbol{(}\AgdaInductiveConstructor{inj₂}\AgdaSpace{}%
\AgdaBound{y}\AgdaSymbol{))}\AgdaSpace{}%
\AgdaSymbol{(}\AgdaBound{k₂}\AgdaSpace{}%
\AgdaSymbol{(}\AgdaInductiveConstructor{inj₂}\AgdaSpace{}%
\AgdaBound{z}\AgdaSymbol{))}\AgdaSpace{}%
\AgdaSymbol{\}))}\AgdaSpace{}%
\AgdaSymbol{\}))}\<%
\\
%
\>[6]\AgdaFunction{interleaveₗ}\AgdaSpace{}%
\AgdaSymbol{(}\AgdaInductiveConstructor{impure}\AgdaSpace{}%
\AgdaSymbol{(}\AgdaInductiveConstructor{inj₁}\AgdaSpace{}%
\AgdaInductiveConstructor{sub}\AgdaSpace{}%
\AgdaOperator{\AgdaInductiveConstructor{,}}\AgdaSpace{}%
\AgdaBound{k₁}\AgdaSymbol{))}\AgdaSpace{}%
\AgdaSymbol{(}\AgdaInductiveConstructor{impure}\AgdaSpace{}%
\AgdaSymbol{(}\AgdaInductiveConstructor{inj₂}\AgdaSpace{}%
\AgdaBound{op₂}\AgdaSpace{}%
\AgdaOperator{\AgdaInductiveConstructor{,}}\AgdaSpace{}%
\AgdaBound{k₂}\AgdaSymbol{))}\AgdaSpace{}%
\AgdaSymbol{=}\AgdaSpace{}%
\AgdaKeyword{do}\<%
\\
\>[6][@{}l@{\AgdaIndent{0}}]%
\>[8]\AgdaInductiveConstructor{impure}\<%
\\
\>[8][@{}l@{\AgdaIndent{0}}]%
\>[10]\AgdaSymbol{(}\AgdaInductiveConstructor{inj₁}\AgdaSpace{}%
\AgdaInductiveConstructor{sub}\AgdaSpace{}%
\AgdaOperator{\AgdaInductiveConstructor{,}}\<%
\\
%
\>[10]\AgdaSymbol{(λ\{}\AgdaSpace{}%
\AgdaSymbol{(}\AgdaInductiveConstructor{inj₁}\AgdaSpace{}%
\AgdaBound{x}\AgdaSymbol{)}\AgdaSpace{}%
\AgdaSymbol{→}\AgdaSpace{}%
\AgdaBound{k₁}\AgdaSpace{}%
\AgdaSymbol{(}\AgdaInductiveConstructor{inj₁}\AgdaSpace{}%
\AgdaBound{x}\AgdaSymbol{)}\<%
\\
\>[10][@{}l@{\AgdaIndent{0}}]%
\>[12]\AgdaSymbol{;}%
\>[2729I]\AgdaSymbol{(}\AgdaInductiveConstructor{inj₂}\AgdaSpace{}%
\AgdaBound{y}\AgdaSymbol{)}\AgdaSpace{}%
\AgdaSymbol{→}\<%
\\
\>[.][@{}l@{}]\<[2729I]%
\>[14]\AgdaInductiveConstructor{impure}\<%
\\
\>[14][@{}l@{\AgdaIndent{0}}]%
\>[16]\AgdaSymbol{(}\AgdaInductiveConstructor{inj₂}\AgdaSpace{}%
\AgdaBound{op₂}\AgdaSpace{}%
\AgdaOperator{\AgdaInductiveConstructor{,}}\<%
\\
%
\>[16]\AgdaSymbol{(λ}\AgdaSpace{}%
\AgdaBound{z}\AgdaSpace{}%
\AgdaSymbol{→}\AgdaSpace{}%
\AgdaFunction{interleaveₗ}\AgdaSpace{}%
\AgdaSymbol{(}\AgdaBound{k₁}\AgdaSpace{}%
\AgdaSymbol{(}\AgdaInductiveConstructor{inj₂}\AgdaSpace{}%
\AgdaBound{y}\AgdaSymbol{))}\AgdaSpace{}%
\AgdaSymbol{(}\AgdaBound{k₂}\AgdaSpace{}%
\AgdaBound{z}\AgdaSymbol{)))}\AgdaSpace{}%
\AgdaSymbol{\}))}\<%
\\
%
\>[6]\AgdaFunction{interleaveₗ}\AgdaSpace{}%
\AgdaSymbol{(}\AgdaInductiveConstructor{impure}\AgdaSpace{}%
\AgdaSymbol{(}\AgdaInductiveConstructor{inj₂}\AgdaSpace{}%
\AgdaBound{op₁}\AgdaSpace{}%
\AgdaOperator{\AgdaInductiveConstructor{,}}\AgdaSpace{}%
\AgdaBound{k₁}\AgdaSymbol{))}\AgdaSpace{}%
\AgdaSymbol{(}\AgdaInductiveConstructor{impure}\AgdaSpace{}%
\AgdaSymbol{(}\AgdaInductiveConstructor{inj₁}\AgdaSpace{}%
\AgdaInductiveConstructor{sub}\AgdaSpace{}%
\AgdaOperator{\AgdaInductiveConstructor{,}}\AgdaSpace{}%
\AgdaBound{k₂}\AgdaSymbol{))}\AgdaSpace{}%
\AgdaSymbol{=}\<%
\\
\>[6][@{}l@{\AgdaIndent{0}}]%
\>[8]\AgdaInductiveConstructor{impure}\<%
\\
\>[8][@{}l@{\AgdaIndent{0}}]%
\>[10]\AgdaSymbol{(}\AgdaInductiveConstructor{inj₂}\AgdaSpace{}%
\AgdaBound{op₁}\AgdaSpace{}%
\AgdaOperator{\AgdaInductiveConstructor{,}}\<%
\\
%
\>[10]\AgdaSymbol{(λ}\AgdaSpace{}%
\AgdaBound{x}\AgdaSpace{}%
\AgdaSymbol{→}\<%
\\
\>[10][@{}l@{\AgdaIndent{0}}]%
\>[12]\AgdaInductiveConstructor{impure}\<%
\\
\>[12][@{}l@{\AgdaIndent{0}}]%
\>[14]\AgdaSymbol{(}\AgdaInductiveConstructor{inj₁}\AgdaSpace{}%
\AgdaInductiveConstructor{sub}\AgdaSpace{}%
\AgdaOperator{\AgdaInductiveConstructor{,}}\<%
\\
%
\>[14]\AgdaSymbol{(λ\{}\AgdaSpace{}%
\AgdaSymbol{(}\AgdaInductiveConstructor{inj₁}\AgdaSpace{}%
\AgdaBound{y}\AgdaSymbol{)}\AgdaSpace{}%
\AgdaSymbol{→}\AgdaSpace{}%
\AgdaBound{k₂}\AgdaSpace{}%
\AgdaSymbol{(}\AgdaInductiveConstructor{inj₁}\AgdaSpace{}%
\AgdaBound{y}\AgdaSymbol{)}\AgdaSpace{}%
\AgdaOperator{\AgdaFunction{𝓑}}\AgdaSpace{}%
\AgdaSymbol{λ}\AgdaSpace{}%
\AgdaBound{\AgdaUnderscore{}}\AgdaSpace{}%
\AgdaSymbol{→}\AgdaSpace{}%
\AgdaBound{k₁}\AgdaSpace{}%
\AgdaBound{x}\<%
\\
\>[14][@{}l@{\AgdaIndent{0}}]%
\>[16]\AgdaSymbol{;}\AgdaSpace{}%
\AgdaSymbol{(}\AgdaInductiveConstructor{inj₂}\AgdaSpace{}%
\AgdaBound{z}\AgdaSymbol{)}\AgdaSpace{}%
\AgdaSymbol{→}\AgdaSpace{}%
\AgdaFunction{interleaveₗ}\AgdaSpace{}%
\AgdaSymbol{(}\AgdaBound{k₁}\AgdaSpace{}%
\AgdaBound{x}\AgdaSymbol{)}\AgdaSpace{}%
\AgdaSymbol{(}\AgdaBound{k₂}\AgdaSpace{}%
\AgdaSymbol{(}\AgdaInductiveConstructor{inj₂}\AgdaSpace{}%
\AgdaBound{z}\AgdaSymbol{))}\AgdaSpace{}%
\AgdaSymbol{\}))))}\<%
\\
%
\>[6]\AgdaFunction{interleaveₗ}\AgdaSpace{}%
\AgdaSymbol{(}\AgdaInductiveConstructor{impure}\AgdaSpace{}%
\AgdaSymbol{(}\AgdaInductiveConstructor{inj₂}\AgdaSpace{}%
\AgdaBound{op₁}\AgdaSpace{}%
\AgdaOperator{\AgdaInductiveConstructor{,}}\AgdaSpace{}%
\AgdaBound{k₁}\AgdaSymbol{))}\AgdaSpace{}%
\AgdaSymbol{(}\AgdaInductiveConstructor{impure}\AgdaSpace{}%
\AgdaSymbol{(}\AgdaInductiveConstructor{inj₂}\AgdaSpace{}%
\AgdaBound{op₂}\AgdaSpace{}%
\AgdaOperator{\AgdaInductiveConstructor{,}}\AgdaSpace{}%
\AgdaBound{k₂}\AgdaSymbol{))}\AgdaSpace{}%
\AgdaSymbol{=}\<%
\\
\>[6][@{}l@{\AgdaIndent{0}}]%
\>[8]\AgdaInductiveConstructor{impure}\AgdaSpace{}%
\AgdaSymbol{(}\AgdaInductiveConstructor{inj₂}\AgdaSpace{}%
\AgdaBound{op₁}\AgdaSpace{}%
\AgdaOperator{\AgdaInductiveConstructor{,}}\AgdaSpace{}%
\AgdaSymbol{λ}\AgdaSpace{}%
\AgdaBound{x₁}\AgdaSpace{}%
\AgdaSymbol{→}\AgdaSpace{}%
\AgdaInductiveConstructor{impure}\AgdaSpace{}%
\AgdaSymbol{(}\AgdaInductiveConstructor{inj₂}\AgdaSpace{}%
\AgdaBound{op₂}\AgdaSpace{}%
\AgdaOperator{\AgdaInductiveConstructor{,}}\AgdaSpace{}%
\AgdaSymbol{λ}\AgdaSpace{}%
\AgdaBound{x₂}\AgdaSpace{}%
\AgdaSymbol{→}\AgdaSpace{}%
\AgdaFunction{interleaveₗ}\AgdaSpace{}%
\AgdaSymbol{(}\AgdaBound{k₁}\AgdaSpace{}%
\AgdaBound{x₁}\AgdaSymbol{)}\AgdaSpace{}%
\AgdaSymbol{(}\AgdaBound{k₂}\AgdaSpace{}%
\AgdaBound{x₂}\AgdaSymbol{)))}\<%
\\
%
\\[\AgdaEmptyExtraSkip]%
%
\\[\AgdaEmptyExtraSkip]%
%
\>[6]\AgdaComment{--\ higher-order\ operation\ for\ concurrency\ that\ desugars\ into\ interleaving\ and\ atomic}\<%
\end{code}
\begin{figure}[t]
\begin{minipage}{0.545\linewidth}
\begin{code}%
%
\>[6]\AgdaKeyword{data}\AgdaSpace{}%
\AgdaDatatype{ConcurOp}\AgdaSpace{}%
\AgdaSymbol{⦃}\AgdaSpace{}%
\AgdaBound{u}\AgdaSpace{}%
\AgdaSymbol{:}\AgdaSpace{}%
\AgdaRecord{Univ}\AgdaSpace{}%
\AgdaSymbol{⦄}\AgdaSpace{}%
\AgdaSymbol{:}\AgdaSpace{}%
\AgdaPrimitive{Set}\AgdaSpace{}%
\AgdaKeyword{where}\<%
\\
\>[6][@{}l@{\AgdaIndent{0}}]%
\>[8]\AgdaInductiveConstructor{spawn}%
\>[16]\AgdaSymbol{:}\AgdaSpace{}%
\AgdaSymbol{(}\AgdaBound{t}\AgdaSpace{}%
\AgdaSymbol{:}\AgdaSpace{}%
\AgdaField{Type}\AgdaSymbol{)}\AgdaSpace{}%
\AgdaSymbol{→}\AgdaSpace{}%
\AgdaDatatype{ConcurOp}\<%
\\
%
\>[8]\AgdaInductiveConstructor{atomic}%
\>[16]\AgdaSymbol{:}\AgdaSpace{}%
\AgdaSymbol{(}\AgdaBound{t}\AgdaSpace{}%
\AgdaSymbol{:}\AgdaSpace{}%
\AgdaField{Type}\AgdaSymbol{)}\AgdaSpace{}%
\AgdaSymbol{→}\AgdaSpace{}%
\AgdaDatatype{ConcurOp}\<%
\end{code}
\end{minipage}
\hfill\vline\hfill
\begin{minipage}{0.445\linewidth}
\begin{code}%
%
\>[6]\AgdaFunction{Concur}\AgdaSpace{}%
\AgdaSymbol{:}\AgdaSpace{}%
\AgdaSymbol{⦃}\AgdaSpace{}%
\AgdaBound{u}\AgdaSpace{}%
\AgdaSymbol{:}\AgdaSpace{}%
\AgdaRecord{Univ}\AgdaSpace{}%
\AgdaSymbol{⦄}\AgdaSpace{}%
\AgdaSymbol{→}\AgdaSpace{}%
\AgdaRecord{Effectᴴ}\<%
\\
%
\>[6]\AgdaField{Opᴴ}\AgdaSpace{}%
\AgdaFunction{Concur}%
\>[20]\AgdaSymbol{=}\AgdaSpace{}%
\AgdaDatatype{ConcurOp}\<%
\\
%
\>[6]\AgdaField{Retᴴ}\AgdaSpace{}%
\AgdaFunction{Concur}\AgdaSpace{}%
\AgdaSymbol{(}\AgdaInductiveConstructor{spawn}\AgdaSpace{}%
\AgdaBound{t}\AgdaSymbol{)}\AgdaSpace{}%
\AgdaSymbol{=}\AgdaSpace{}%
\AgdaOperator{\AgdaField{⟦}}\AgdaSpace{}%
\AgdaBound{t}\AgdaSpace{}%
\AgdaOperator{\AgdaField{⟧ᵀ}}\<%
\\
%
\>[6]\AgdaField{Retᴴ}\AgdaSpace{}%
\AgdaFunction{Concur}\AgdaSpace{}%
\AgdaSymbol{(}\AgdaInductiveConstructor{atomic}\AgdaSpace{}%
\AgdaBound{t}\AgdaSymbol{)}%
\>[32]\AgdaSymbol{=}\AgdaSpace{}%
\AgdaOperator{\AgdaField{⟦}}\AgdaSpace{}%
\AgdaBound{t}\AgdaSpace{}%
\AgdaOperator{\AgdaField{⟧ᵀ}}\<%
\\
\>[0]\<%
\\
%
\>[6]\AgdaField{Fork}\AgdaSpace{}%
\AgdaFunction{Concur}\AgdaSpace{}%
\AgdaSymbol{(}\AgdaInductiveConstructor{spawn}\AgdaSpace{}%
\AgdaBound{t}\AgdaSymbol{)}\AgdaSpace{}%
\AgdaSymbol{=}\AgdaSpace{}%
\AgdaDatatype{Bool}\<%
\\
%
\>[6]\AgdaField{Fork}\AgdaSpace{}%
\AgdaFunction{Concur}\AgdaSpace{}%
\AgdaSymbol{(}\AgdaInductiveConstructor{atomic}\AgdaSpace{}%
\AgdaBound{t}\AgdaSymbol{)}%
\>[31]\AgdaSymbol{=}\AgdaSpace{}%
\AgdaRecord{⊤}\<%
\\
%
\>[6]\AgdaField{Ty}%
\>[11]\AgdaFunction{Concur}\AgdaSpace{}%
\AgdaSymbol{\{}\AgdaInductiveConstructor{spawn}\AgdaSpace{}%
\AgdaBound{t}\AgdaSymbol{\}}\AgdaSpace{}%
\AgdaSymbol{\AgdaUnderscore{}}\AgdaSpace{}%
\AgdaSymbol{=}\AgdaSpace{}%
\AgdaOperator{\AgdaField{⟦}}\AgdaSpace{}%
\AgdaBound{t}\AgdaSpace{}%
\AgdaOperator{\AgdaField{⟧ᵀ}}\<%
\\
%
\>[6]\AgdaField{Ty}%
\>[11]\AgdaFunction{Concur}\AgdaSpace{}%
\AgdaSymbol{\{}\AgdaInductiveConstructor{atomic}\AgdaSpace{}%
\AgdaBound{t}\AgdaSymbol{\}}\AgdaSpace{}%
\AgdaSymbol{\AgdaUnderscore{}}\AgdaSpace{}%
\AgdaSymbol{=}\AgdaSpace{}%
\AgdaOperator{\AgdaField{⟦}}\AgdaSpace{}%
\AgdaBound{t}\AgdaSpace{}%
\AgdaOperator{\AgdaField{⟧ᵀ}}\<%
\end{code}
\end{minipage}
\caption{Higher-order effect signature of the concur effect}
\label{fig:concur-ho-sig}
\end{figure}
\begin{code}[hide]%
%
\>[6]\AgdaKeyword{module}\AgdaSpace{}%
\AgdaModule{\AgdaUnderscore{}}\AgdaSpace{}%
\AgdaSymbol{⦃}\AgdaSpace{}%
\AgdaBound{u}\AgdaSpace{}%
\AgdaSymbol{:}\AgdaSpace{}%
\AgdaRecord{Univ}\AgdaSpace{}%
\AgdaSymbol{⦄}\AgdaSpace{}%
\AgdaKeyword{where}\<%
\\
\>[6][@{}l@{\AgdaIndent{0}}]%
\>[8]\AgdaFunction{‵spawn}%
\>[2883I]\AgdaSymbol{:}\AgdaSpace{}%
\AgdaSymbol{⦃}\AgdaSpace{}%
\AgdaBound{w}\AgdaSpace{}%
\AgdaSymbol{:}\AgdaSpace{}%
\AgdaFunction{Concur}\AgdaSpace{}%
\AgdaOperator{\AgdaFunction{≲ᴴ}}\AgdaSpace{}%
\AgdaGeneralizable{H}\AgdaSpace{}%
\AgdaSymbol{⦄}\AgdaSpace{}%
\AgdaSymbol{\{}\AgdaBound{t}\AgdaSpace{}%
\AgdaSymbol{:}\AgdaSpace{}%
\AgdaField{Type}\AgdaSymbol{\}}\<%
\\
\>[.][@{}l@{}]\<[2883I]%
\>[15]\AgdaSymbol{→}\AgdaSpace{}%
\AgdaDatatype{Hefty}\AgdaSpace{}%
\AgdaGeneralizable{H}\AgdaSpace{}%
\AgdaOperator{\AgdaField{⟦}}\AgdaSpace{}%
\AgdaBound{t}\AgdaSpace{}%
\AgdaOperator{\AgdaField{⟧ᵀ}}\AgdaSpace{}%
\AgdaSymbol{→}\AgdaSpace{}%
\AgdaDatatype{Hefty}\AgdaSpace{}%
\AgdaGeneralizable{H}\AgdaSpace{}%
\AgdaOperator{\AgdaField{⟦}}\AgdaSpace{}%
\AgdaBound{t}\AgdaSpace{}%
\AgdaOperator{\AgdaField{⟧ᵀ}}\AgdaSpace{}%
\AgdaSymbol{→}\AgdaSpace{}%
\AgdaDatatype{Hefty}\AgdaSpace{}%
\AgdaGeneralizable{H}\AgdaSpace{}%
\AgdaOperator{\AgdaField{⟦}}\AgdaSpace{}%
\AgdaBound{t}\AgdaSpace{}%
\AgdaOperator{\AgdaField{⟧ᵀ}}\<%
\\
%
\>[8]\AgdaFunction{‵spawn}\AgdaSpace{}%
\AgdaSymbol{⦃}\AgdaSpace{}%
\AgdaArgument{w}\AgdaSpace{}%
\AgdaSymbol{=}\AgdaSpace{}%
\AgdaBound{w}\AgdaSpace{}%
\AgdaSymbol{⦄}\AgdaSpace{}%
\AgdaSymbol{\{}\AgdaBound{t}\AgdaSymbol{\}}\AgdaSpace{}%
\AgdaBound{m₁}\AgdaSpace{}%
\AgdaBound{m₂}\AgdaSpace{}%
\AgdaSymbol{=}\<%
\\
\>[8][@{}l@{\AgdaIndent{0}}]%
\>[10]\AgdaInductiveConstructor{impure}\AgdaSpace{}%
\AgdaSymbol{(}\AgdaFunction{injᴴ}\AgdaSpace{}%
\AgdaSymbol{\{}\AgdaArgument{M}\AgdaSpace{}%
\AgdaSymbol{=}\AgdaSpace{}%
\AgdaDatatype{Hefty}\AgdaSpace{}%
\AgdaSymbol{\AgdaUnderscore{}\}}\AgdaSpace{}%
\AgdaSymbol{(}\AgdaInductiveConstructor{spawn}\AgdaSpace{}%
\AgdaBound{t}\AgdaSpace{}%
\AgdaOperator{\AgdaInductiveConstructor{,}}\AgdaSpace{}%
\AgdaInductiveConstructor{pure}\AgdaSpace{}%
\AgdaOperator{\AgdaInductiveConstructor{,}}\AgdaSpace{}%
\AgdaSymbol{(}\AgdaOperator{\AgdaFunction{if\AgdaUnderscore{}then}}\AgdaSpace{}%
\AgdaBound{m₁}\AgdaSpace{}%
\AgdaOperator{\AgdaFunction{else}}\AgdaSpace{}%
\AgdaBound{m₂}\AgdaSymbol{)))}\<%
\\
%
\\[\AgdaEmptyExtraSkip]%
%
\>[8]\AgdaFunction{‵atomic}%
\>[2934I]\AgdaSymbol{:}\AgdaSpace{}%
\AgdaSymbol{⦃}\AgdaSpace{}%
\AgdaBound{w}\AgdaSpace{}%
\AgdaSymbol{:}\AgdaSpace{}%
\AgdaFunction{Concur}\AgdaSpace{}%
\AgdaOperator{\AgdaFunction{≲ᴴ}}\AgdaSpace{}%
\AgdaGeneralizable{H}\AgdaSpace{}%
\AgdaSymbol{⦄}\AgdaSpace{}%
\AgdaSymbol{\{}\AgdaBound{t}\AgdaSpace{}%
\AgdaSymbol{:}\AgdaSpace{}%
\AgdaField{Type}\AgdaSymbol{\}}\<%
\\
\>[2934I][@{}l@{\AgdaIndent{0}}]%
\>[17]\AgdaSymbol{→}\AgdaSpace{}%
\AgdaDatatype{Hefty}\AgdaSpace{}%
\AgdaGeneralizable{H}\AgdaSpace{}%
\AgdaOperator{\AgdaField{⟦}}\AgdaSpace{}%
\AgdaBound{t}\AgdaSpace{}%
\AgdaOperator{\AgdaField{⟧ᵀ}}\AgdaSpace{}%
\AgdaSymbol{→}\AgdaSpace{}%
\AgdaDatatype{Hefty}\AgdaSpace{}%
\AgdaGeneralizable{H}\AgdaSpace{}%
\AgdaOperator{\AgdaField{⟦}}\AgdaSpace{}%
\AgdaBound{t}\AgdaSpace{}%
\AgdaOperator{\AgdaField{⟧ᵀ}}\<%
\\
%
\>[8]\AgdaFunction{‵atomic}\AgdaSpace{}%
\AgdaSymbol{⦃}\AgdaSpace{}%
\AgdaArgument{w}\AgdaSpace{}%
\AgdaSymbol{=}\AgdaSpace{}%
\AgdaBound{w}\AgdaSpace{}%
\AgdaSymbol{⦄}\AgdaSpace{}%
\AgdaSymbol{\{}\AgdaBound{t}\AgdaSymbol{\}}\AgdaSpace{}%
\AgdaBound{m}\AgdaSpace{}%
\AgdaSymbol{=}\AgdaSpace{}%
\AgdaInductiveConstructor{impure}\<%
\\
\>[8][@{}l@{\AgdaIndent{0}}]%
\>[10]\AgdaSymbol{(}\AgdaFunction{injᴴ}\AgdaSpace{}%
\AgdaSymbol{\{}\AgdaArgument{M}\AgdaSpace{}%
\AgdaSymbol{=}\AgdaSpace{}%
\AgdaDatatype{Hefty}\AgdaSpace{}%
\AgdaSymbol{\AgdaUnderscore{}\}}\AgdaSpace{}%
\AgdaSymbol{(}\AgdaInductiveConstructor{atomic}\AgdaSpace{}%
\AgdaBound{t}\AgdaSpace{}%
\AgdaOperator{\AgdaInductiveConstructor{,}}\AgdaSpace{}%
\AgdaInductiveConstructor{pure}\AgdaSpace{}%
\AgdaOperator{\AgdaInductiveConstructor{,}}\AgdaSpace{}%
\AgdaSymbol{λ}\AgdaSpace{}%
\AgdaBound{\AgdaUnderscore{}}\AgdaSpace{}%
\AgdaSymbol{→}\AgdaSpace{}%
\AgdaBound{m}\AgdaSymbol{))}\<%
\\
%
\\[\AgdaEmptyExtraSkip]%
%
\>[8]\AgdaKeyword{module}\AgdaSpace{}%
\AgdaModule{\AgdaUnderscore{}}\AgdaSpace{}%
\AgdaSymbol{\{}\AgdaBound{Ref}\AgdaSpace{}%
\AgdaSymbol{:}\AgdaSpace{}%
\AgdaField{Type}\AgdaSpace{}%
\AgdaSymbol{→}\AgdaSpace{}%
\AgdaPrimitive{Set}\AgdaSymbol{\}}\AgdaSpace{}%
\AgdaSymbol{⦃}\AgdaSpace{}%
\AgdaBound{w}\AgdaSpace{}%
\AgdaSymbol{:}\AgdaSpace{}%
\AgdaFunction{CC}\AgdaSpace{}%
\AgdaBound{Ref}\AgdaSpace{}%
\AgdaOperator{\AgdaFunction{≲}}\AgdaSpace{}%
\AgdaGeneralizable{Δ}\AgdaSpace{}%
\AgdaSymbol{⦄}\AgdaSpace{}%
\AgdaKeyword{where}\<%
\\
\>[8][@{}l@{\AgdaIndent{0}}]%
\>[10]\AgdaKeyword{private}\AgdaSpace{}%
\AgdaKeyword{instance}\<%
\\
\>[10][@{}l@{\AgdaIndent{0}}]%
\>[12]\AgdaFunction{\AgdaUnderscore{}}\AgdaSpace{}%
\AgdaSymbol{:}\AgdaSpace{}%
\AgdaFunction{CC}\AgdaSpace{}%
\AgdaBound{Ref}\AgdaSpace{}%
\AgdaOperator{\AgdaRecord{∙}}\AgdaSpace{}%
\AgdaField{proj₁}\AgdaSpace{}%
\AgdaBound{w}\AgdaSpace{}%
\AgdaOperator{\AgdaRecord{≈}}\AgdaSpace{}%
\AgdaBound{Δ}\<%
\\
%
\>[12]\AgdaSymbol{\AgdaUnderscore{}}\AgdaSpace{}%
\AgdaSymbol{=}\AgdaSpace{}%
\AgdaBound{w}\AgdaSpace{}%
\AgdaSymbol{.}\AgdaField{proj₂}\<%
\\
%
\\[\AgdaEmptyExtraSkip]%
%
\>[10]\AgdaFunction{eConcur}\AgdaSpace{}%
\AgdaSymbol{:}\AgdaSpace{}%
\AgdaFunction{Elaboration}\AgdaSpace{}%
\AgdaFunction{Concur}\AgdaSpace{}%
\AgdaBound{Δ}\<%
\\
%
\>[10]\AgdaField{alg}\AgdaSpace{}%
\AgdaFunction{eConcur}\AgdaSpace{}%
\AgdaSymbol{(}\AgdaInductiveConstructor{spawn}\AgdaSpace{}%
\AgdaBound{t}\AgdaSpace{}%
\AgdaOperator{\AgdaInductiveConstructor{,}}\AgdaSpace{}%
\AgdaBound{k}\AgdaSpace{}%
\AgdaOperator{\AgdaInductiveConstructor{,}}\AgdaSpace{}%
\AgdaBound{ψ}\AgdaSymbol{)}%
\>[41]\AgdaSymbol{=}\<%
\\
\>[10][@{}l@{\AgdaIndent{0}}]%
\>[12]\AgdaFunction{from-front}\AgdaSpace{}%
\AgdaSymbol{(}\AgdaFunction{interleaveₗ}\AgdaSpace{}%
\AgdaSymbol{(}\AgdaFunction{to-front}\AgdaSpace{}%
\AgdaSymbol{(}\AgdaBound{ψ}\AgdaSpace{}%
\AgdaInductiveConstructor{true}\AgdaSymbol{))}\AgdaSpace{}%
\AgdaSymbol{(}\AgdaFunction{to-front}\AgdaSpace{}%
\AgdaSymbol{(}\AgdaBound{ψ}\AgdaSpace{}%
\AgdaInductiveConstructor{false}\AgdaSymbol{)))}\AgdaSpace{}%
\AgdaOperator{\AgdaFunction{𝓑}}\AgdaSpace{}%
\AgdaBound{k}\<%
\\
%
\>[10]\AgdaField{alg}\AgdaSpace{}%
\AgdaFunction{eConcur}\AgdaSpace{}%
\AgdaSymbol{(}\AgdaInductiveConstructor{atomic}\AgdaSpace{}%
\AgdaBound{t}\AgdaSpace{}%
\AgdaOperator{\AgdaInductiveConstructor{,}}\AgdaSpace{}%
\AgdaBound{k}\AgdaSpace{}%
\AgdaOperator{\AgdaInductiveConstructor{,}}\AgdaSpace{}%
\AgdaBound{ψ}\AgdaSymbol{)}%
\>[42]\AgdaSymbol{=}\AgdaSpace{}%
\AgdaFunction{‵sub}\AgdaSpace{}%
\AgdaSymbol{(λ}\AgdaSpace{}%
\AgdaBound{ref}\AgdaSpace{}%
\AgdaSymbol{→}\AgdaSpace{}%
\AgdaBound{ψ}\AgdaSpace{}%
\AgdaInductiveConstructor{tt}\AgdaSpace{}%
\AgdaOperator{\AgdaFunction{𝓑}}\AgdaSpace{}%
\AgdaFunction{‵jump}\AgdaSpace{}%
\AgdaBound{ref}\AgdaSymbol{)}\AgdaSpace{}%
\AgdaBound{k}\<%
\end{code}
%
%
Here, the \ad{CC} effect is the sub/jump effect that we also used in \cref{sec:optional-transactional}.
The \af{interleaveₗ} function ensures atomic execution by only interleaving code that is not wrapped in a \af{‵sub} operation.
We elaborate \ad{Concur} into \ad{CC} as follows, where the \af{to-front} and \af{from-front} functions use the row insertion witness \ab{wₐ} to move the \ad{CC} effect to the front of the row and back again:
%
\begin{code}%
%
\>[10]\AgdaFunction{eConcur⅋}\AgdaSpace{}%
\AgdaSymbol{:}\AgdaSpace{}%
\AgdaSymbol{⦃}\AgdaSpace{}%
\AgdaBound{w}\AgdaSpace{}%
\AgdaSymbol{:}\AgdaSpace{}%
\AgdaFunction{CC}\AgdaSpace{}%
\AgdaBound{Ref}\AgdaSpace{}%
\AgdaOperator{\AgdaPostulate{≲⅋}}\AgdaSpace{}%
\AgdaBound{Δ}\AgdaSpace{}%
\AgdaSymbol{⦄}\AgdaSpace{}%
\AgdaSymbol{→}\AgdaSpace{}%
\AgdaFunction{Elaboration}\AgdaSpace{}%
\AgdaFunction{Concur}\AgdaSpace{}%
\AgdaBound{Δ}\<%
\\
%
\>[10]\AgdaField{alg}\AgdaSpace{}%
\AgdaFunction{eConcur⅋}\AgdaSpace{}%
\AgdaSymbol{(}\AgdaInductiveConstructor{spawn}\AgdaSpace{}%
\AgdaBound{t}\AgdaSpace{}%
\AgdaOperator{\AgdaInductiveConstructor{,}}\AgdaSpace{}%
\AgdaBound{k}\AgdaSpace{}%
\AgdaOperator{\AgdaInductiveConstructor{,}}\AgdaSpace{}%
\AgdaBound{ψ}\AgdaSymbol{)}%
\>[42]\AgdaSymbol{=}\<%
\\
\>[10][@{}l@{\AgdaIndent{0}}]%
\>[12]\AgdaFunction{from-front}\AgdaSpace{}%
\AgdaSymbol{(}\AgdaFunction{interleaveₗ}\AgdaSpace{}%
\AgdaSymbol{(}\AgdaFunction{to-front}\AgdaSpace{}%
\AgdaSymbol{(}\AgdaBound{ψ}\AgdaSpace{}%
\AgdaInductiveConstructor{true}\AgdaSymbol{))}\AgdaSpace{}%
\AgdaSymbol{(}\AgdaFunction{to-front}\AgdaSpace{}%
\AgdaSymbol{(}\AgdaBound{ψ}\AgdaSpace{}%
\AgdaInductiveConstructor{false}\AgdaSymbol{)))}\AgdaSpace{}%
\AgdaOperator{\AgdaFunction{𝓑}}\AgdaSpace{}%
\AgdaBound{k}\<%
\\
%
\>[10]\AgdaField{alg}\AgdaSpace{}%
\AgdaFunction{eConcur⅋}\AgdaSpace{}%
\AgdaSymbol{(}\AgdaInductiveConstructor{atomic}\AgdaSpace{}%
\AgdaBound{t}\AgdaSpace{}%
\AgdaOperator{\AgdaInductiveConstructor{,}}\AgdaSpace{}%
\AgdaBound{k}\AgdaSpace{}%
\AgdaOperator{\AgdaInductiveConstructor{,}}\AgdaSpace{}%
\AgdaBound{ψ}\AgdaSymbol{)}%
\>[43]\AgdaSymbol{=}\AgdaSpace{}%
\AgdaFunction{‵sub}\AgdaSpace{}%
\AgdaSymbol{(λ}\AgdaSpace{}%
\AgdaBound{ref}\AgdaSpace{}%
\AgdaSymbol{→}\AgdaSpace{}%
\AgdaBound{ψ}\AgdaSpace{}%
\AgdaInductiveConstructor{tt}\AgdaSpace{}%
\AgdaOperator{\AgdaFunction{𝓑}}\AgdaSpace{}%
\AgdaFunction{‵jump}\AgdaSpace{}%
\AgdaBound{ref}\AgdaSymbol{)}\AgdaSpace{}%
\AgdaBound{k}\<%
\end{code}
%
The elaboration uses \af{‵sub} as a delimiter for blocks that should not be interleaved, such that the \af{interleaveₗ} function only interleaves code that does not reside in atomic blocks.
At the end of an \ac{atomic} block, we \af{‵jump} to the (possibly interleaved) computation context, \ab{k}.
By using \af{‵sub} to explicitly delimit blocks that should not be interleaved, we have encoded what \citet[\S{}~7]{WuSH14} call \emph{scoped syntax}.

\paragraph*{Example.}
  Below is an example program that spawns two threads that use the \ad{Output} effect.
  The first thread prints \an{0}, \an{1}, and \an{2}; the second prints \an{3} and \an{4}.
%
\begin{code}[hide]%
\>[0][@{}l@{\AgdaIndent{5}}]%
\>[4]\AgdaKeyword{module}\AgdaSpace{}%
\AgdaModule{ConcurExample}\AgdaSpace{}%
\AgdaKeyword{where}\<%
\\
\>[4][@{}l@{\AgdaIndent{0}}]%
\>[6]\AgdaKeyword{open}\AgdaSpace{}%
\AgdaKeyword{import}\AgdaSpace{}%
\AgdaModule{Data.Nat}\AgdaSpace{}%
\AgdaKeyword{using}\AgdaSpace{}%
\AgdaSymbol{(}\AgdaDatatype{ℕ}\AgdaSymbol{)}\<%
\\
%
\>[6]\AgdaComment{--\ open\ OutModule}\<%
\\
%
\>[6]\AgdaKeyword{open}\AgdaSpace{}%
\AgdaModule{HeftyModule}\<%
\\
%
\>[6]\AgdaKeyword{open}\AgdaSpace{}%
\AgdaModule{FreeModule}\AgdaSpace{}%
\AgdaKeyword{hiding}\AgdaSpace{}%
\AgdaSymbol{(}\AgdaOperator{\AgdaFunction{\AgdaUnderscore{}𝓑\AgdaUnderscore{}}}\AgdaSymbol{;}\AgdaSpace{}%
\AgdaOperator{\AgdaFunction{\AgdaUnderscore{}>>\AgdaUnderscore{}}}\AgdaSymbol{)}\<%
\\
%
\>[6]\AgdaKeyword{open}\AgdaSpace{}%
\AgdaModule{ElabModule}\<%
\\
%
\>[6]\AgdaKeyword{open}\AgdaSpace{}%
\AgdaModule{CCModule}\<%
\\
%
\>[6]\AgdaComment{--\ open\ Elab}\<%
\\
%
\\[\AgdaEmptyExtraSkip]%
%
\>[6]\AgdaKeyword{data}\AgdaSpace{}%
\AgdaDatatype{ConcurType}\AgdaSpace{}%
\AgdaSymbol{:}\AgdaSpace{}%
\AgdaPrimitive{Set}\AgdaSpace{}%
\AgdaKeyword{where}\<%
\\
\>[6][@{}l@{\AgdaIndent{0}}]%
\>[8]\AgdaInductiveConstructor{unit}\AgdaSpace{}%
\AgdaSymbol{:}\AgdaSpace{}%
\AgdaDatatype{ConcurType}\<%
\\
%
\>[8]\AgdaInductiveConstructor{num}\AgdaSpace{}%
\AgdaSymbol{:}\AgdaSpace{}%
\AgdaDatatype{ConcurType}\<%
\\
%
\\[\AgdaEmptyExtraSkip]%
%
\>[6]\AgdaKeyword{instance}\<%
\\
\>[6][@{}l@{\AgdaIndent{0}}]%
\>[8]\AgdaFunction{ConcurUniv}\AgdaSpace{}%
\AgdaSymbol{:}\AgdaSpace{}%
\AgdaRecord{Univ}\<%
\\
%
\>[8]\AgdaField{Type}\AgdaSpace{}%
\AgdaSymbol{⦃}\AgdaSpace{}%
\AgdaFunction{ConcurUniv}\AgdaSpace{}%
\AgdaSymbol{⦄}\AgdaSpace{}%
\AgdaSymbol{=}\AgdaSpace{}%
\AgdaDatatype{ConcurType}\<%
\\
%
\>[8]\AgdaOperator{\AgdaField{⟦\AgdaUnderscore{}⟧ᵀ}}\AgdaSpace{}%
\AgdaSymbol{⦃}\AgdaSpace{}%
\AgdaFunction{ConcurUniv}\AgdaSpace{}%
\AgdaSymbol{⦄}\AgdaSpace{}%
\AgdaInductiveConstructor{unit}\AgdaSpace{}%
\AgdaSymbol{=}\AgdaSpace{}%
\AgdaRecord{⊤}\<%
\\
%
\>[8]\AgdaOperator{\AgdaField{⟦\AgdaUnderscore{}⟧ᵀ}}\AgdaSpace{}%
\AgdaSymbol{⦃}\AgdaSpace{}%
\AgdaFunction{ConcurUniv}\AgdaSpace{}%
\AgdaSymbol{⦄}\AgdaSpace{}%
\AgdaInductiveConstructor{num}\AgdaSpace{}%
\AgdaSymbol{=}\AgdaSpace{}%
\AgdaDatatype{ℕ}\<%
\\
%
\\[\AgdaEmptyExtraSkip]%
%
\>[6]\AgdaKeyword{module}\AgdaSpace{}%
\AgdaModule{\AgdaUnderscore{}}\AgdaSpace{}%
\AgdaKeyword{where}\<%
\\
\>[6][@{}l@{\AgdaIndent{0}}]%
\>[8]\AgdaKeyword{private}\AgdaSpace{}%
\AgdaKeyword{instance}\<%
\\
\>[8][@{}l@{\AgdaIndent{0}}]%
\>[10]\AgdaFunction{x₀}\AgdaSpace{}%
\AgdaSymbol{:}\AgdaSpace{}%
\AgdaFunction{Lift}\AgdaSpace{}%
\AgdaFunction{Output}\AgdaSpace{}%
\AgdaOperator{\AgdaFunction{≲ᴴ}}\AgdaSpace{}%
\AgdaSymbol{(}\AgdaFunction{Lift}\AgdaSpace{}%
\AgdaFunction{Output}\AgdaSpace{}%
\AgdaOperator{\AgdaFunction{∔}}\AgdaSpace{}%
\AgdaFunction{Concur}\AgdaSpace{}%
\AgdaOperator{\AgdaFunction{∔}}\AgdaSpace{}%
\AgdaFunction{Lift}\AgdaSpace{}%
\AgdaFunction{Nil}\AgdaSymbol{)}\<%
\\
%
\>[10]\AgdaFunction{x₀}\AgdaSpace{}%
\AgdaSymbol{=}\AgdaSpace{}%
\AgdaFunction{≲ᴴ-left}\<%
\\
%
\\[\AgdaEmptyExtraSkip]%
%
\>[10]\AgdaFunction{x₁}\AgdaSpace{}%
\AgdaSymbol{:}\AgdaSpace{}%
\AgdaFunction{Concur}\AgdaSpace{}%
\AgdaOperator{\AgdaFunction{≲ᴴ}}\AgdaSpace{}%
\AgdaSymbol{(}\AgdaFunction{Lift}\AgdaSpace{}%
\AgdaFunction{Output}\AgdaSpace{}%
\AgdaOperator{\AgdaFunction{∔}}\AgdaSpace{}%
\AgdaFunction{Concur}\AgdaSpace{}%
\AgdaOperator{\AgdaFunction{∔}}\AgdaSpace{}%
\AgdaFunction{Lift}\AgdaSpace{}%
\AgdaFunction{Nil}\AgdaSymbol{)}\<%
\\
%
\>[10]\AgdaFunction{x₁}\AgdaSpace{}%
\AgdaSymbol{=}\AgdaSpace{}%
\AgdaFunction{≲ᴴ-right}\AgdaSpace{}%
\AgdaSymbol{⦃}\AgdaSpace{}%
\AgdaFunction{≲ᴴ-left}\AgdaSpace{}%
\AgdaSymbol{⦄}\<%
\end{code}
\begin{code}%
%
\>[8]\AgdaFunction{ex-01234}\AgdaSpace{}%
\AgdaSymbol{:}\AgdaSpace{}%
\AgdaDatatype{Hefty}\AgdaSpace{}%
\AgdaSymbol{(}\AgdaFunction{Lift}\AgdaSpace{}%
\AgdaFunction{Output}\AgdaSpace{}%
\AgdaOperator{\AgdaFunction{∔}}\AgdaSpace{}%
\AgdaFunction{Concur}\AgdaSpace{}%
\AgdaOperator{\AgdaFunction{∔}}\AgdaSpace{}%
\AgdaFunction{Lift}\AgdaSpace{}%
\AgdaFunction{Nil}\AgdaSymbol{)}\AgdaSpace{}%
\AgdaDatatype{ℕ}\<%
\\
%
\>[8]\AgdaFunction{ex-01234}\AgdaSpace{}%
\AgdaSymbol{=}\AgdaSpace{}%
\AgdaFunction{‵spawn}%
\>[27]\AgdaSymbol{(}\AgdaKeyword{do}\AgdaSpace{}%
\AgdaOperator{\AgdaFunction{↑}}\AgdaSpace{}%
\AgdaInductiveConstructor{out}\AgdaSpace{}%
\AgdaString{"0"}\AgdaSymbol{;}\AgdaSpace{}%
\AgdaOperator{\AgdaFunction{↑}}\AgdaSpace{}%
\AgdaInductiveConstructor{out}\AgdaSpace{}%
\AgdaString{"1"}\AgdaSymbol{;}\AgdaSpace{}%
\AgdaOperator{\AgdaFunction{↑}}\AgdaSpace{}%
\AgdaInductiveConstructor{out}\AgdaSpace{}%
\AgdaString{"2"}\AgdaSymbol{;}\AgdaSpace{}%
\AgdaInductiveConstructor{pure}\AgdaSpace{}%
\AgdaNumber{0}\AgdaSymbol{)}\<%
\\
%
\>[27]\AgdaSymbol{(}\AgdaKeyword{do}\AgdaSpace{}%
\AgdaOperator{\AgdaFunction{↑}}\AgdaSpace{}%
\AgdaInductiveConstructor{out}\AgdaSpace{}%
\AgdaString{"3"}\AgdaSymbol{;}\AgdaSpace{}%
\AgdaOperator{\AgdaFunction{↑}}\AgdaSpace{}%
\AgdaInductiveConstructor{out}\AgdaSpace{}%
\AgdaString{"4"}\AgdaSymbol{;}\AgdaSpace{}%
\AgdaInductiveConstructor{pure}\AgdaSpace{}%
\AgdaNumber{0}\AgdaSymbol{)}\<%
\end{code}
%
Since the \ad{Concur} effect is elaborated to interleave the effects of the two threads, the printed output appears in interleaved order:
%
\begin{code}[hide]%
%
\>[6]\AgdaKeyword{module}\AgdaSpace{}%
\AgdaModule{\AgdaUnderscore{}}\AgdaSpace{}%
\AgdaKeyword{where}\<%
\\
\>[6][@{}l@{\AgdaIndent{0}}]%
\>[8]\AgdaKeyword{private}\AgdaSpace{}%
\AgdaKeyword{instance}\<%
\\
\>[8][@{}l@{\AgdaIndent{0}}]%
\>[10]\AgdaFunction{x₀}\AgdaSpace{}%
\AgdaSymbol{:}\AgdaSpace{}%
\AgdaFunction{CC}\AgdaSpace{}%
\AgdaSymbol{(λ}\AgdaSpace{}%
\AgdaBound{t}\AgdaSpace{}%
\AgdaSymbol{→}\AgdaSpace{}%
\AgdaOperator{\AgdaField{⟦}}\AgdaSpace{}%
\AgdaBound{t}\AgdaSpace{}%
\AgdaOperator{\AgdaField{⟧ᵀ}}\AgdaSpace{}%
\AgdaSymbol{→}\AgdaSpace{}%
\AgdaDatatype{Free}\AgdaSpace{}%
\AgdaSymbol{(}\AgdaFunction{Output}\AgdaSpace{}%
\AgdaOperator{\AgdaFunction{⊕}}\AgdaSpace{}%
\AgdaFunction{Nil}\AgdaSymbol{)}\AgdaSpace{}%
\AgdaDatatype{ℕ}\AgdaSymbol{)}\AgdaSpace{}%
\AgdaOperator{\AgdaFunction{≲}}\AgdaSpace{}%
\AgdaSymbol{(}\AgdaFunction{CC}\AgdaSpace{}%
\AgdaSymbol{(λ}\AgdaSpace{}%
\AgdaBound{t}\AgdaSpace{}%
\AgdaSymbol{→}\AgdaSpace{}%
\AgdaOperator{\AgdaField{⟦}}\AgdaSpace{}%
\AgdaBound{t}\AgdaSpace{}%
\AgdaOperator{\AgdaField{⟧ᵀ}}\AgdaSpace{}%
\AgdaSymbol{→}\AgdaSpace{}%
\AgdaDatatype{Free}\AgdaSpace{}%
\AgdaSymbol{(}\AgdaFunction{Output}\AgdaSpace{}%
\AgdaOperator{\AgdaFunction{⊕}}\AgdaSpace{}%
\AgdaFunction{Nil}\AgdaSymbol{)}\AgdaSpace{}%
\AgdaDatatype{ℕ}\AgdaSymbol{)}\AgdaSpace{}%
\AgdaOperator{\AgdaFunction{⊕}}\AgdaSpace{}%
\AgdaFunction{Output}\AgdaSpace{}%
\AgdaOperator{\AgdaFunction{⊕}}\AgdaSpace{}%
\AgdaFunction{Nil}\AgdaSymbol{)}\<%
\\
%
\>[10]\AgdaFunction{x₀}\AgdaSpace{}%
\AgdaSymbol{=}\AgdaSpace{}%
\AgdaFunction{≲-left}\<%
\\
%
\\[\AgdaEmptyExtraSkip]%
%
\>[10]\AgdaFunction{x₁}\AgdaSpace{}%
\AgdaSymbol{:}\AgdaSpace{}%
\AgdaFunction{Output}\AgdaSpace{}%
\AgdaOperator{\AgdaFunction{≲}}\AgdaSpace{}%
\AgdaSymbol{(}\AgdaFunction{CC}\AgdaSpace{}%
\AgdaSymbol{(λ}\AgdaSpace{}%
\AgdaBound{t}\AgdaSpace{}%
\AgdaSymbol{→}\AgdaSpace{}%
\AgdaOperator{\AgdaField{⟦}}\AgdaSpace{}%
\AgdaBound{t}\AgdaSpace{}%
\AgdaOperator{\AgdaField{⟧ᵀ}}\AgdaSpace{}%
\AgdaSymbol{→}\AgdaSpace{}%
\AgdaDatatype{Free}\AgdaSpace{}%
\AgdaSymbol{(}\AgdaFunction{Output}\AgdaSpace{}%
\AgdaOperator{\AgdaFunction{⊕}}\AgdaSpace{}%
\AgdaFunction{Nil}\AgdaSymbol{)}\AgdaSpace{}%
\AgdaDatatype{ℕ}\AgdaSymbol{)}\AgdaSpace{}%
\AgdaOperator{\AgdaFunction{⊕}}\AgdaSpace{}%
\AgdaFunction{Output}\AgdaSpace{}%
\AgdaOperator{\AgdaFunction{⊕}}\AgdaSpace{}%
\AgdaFunction{Nil}\AgdaSymbol{)}\<%
\\
%
\>[10]\AgdaFunction{x₁}\AgdaSpace{}%
\AgdaSymbol{=}\AgdaSpace{}%
\AgdaFunction{≲-right}\AgdaSpace{}%
\AgdaSymbol{⦃}\AgdaSpace{}%
\AgdaFunction{≲-left}\AgdaSpace{}%
\AgdaSymbol{⦄}\<%
\\
%
\\[\AgdaEmptyExtraSkip]%
%
\>[10]\AgdaFunction{x₂}\AgdaSpace{}%
\AgdaSymbol{:}\AgdaSpace{}%
\AgdaFunction{Output}\AgdaSpace{}%
\AgdaOperator{\AgdaFunction{≲}}\AgdaSpace{}%
\AgdaField{proj₁}\AgdaSpace{}%
\AgdaFunction{x₀}\<%
\\
%
\>[10]\AgdaFunction{x₂}\AgdaSpace{}%
\AgdaSymbol{=}\AgdaSpace{}%
\AgdaSymbol{\AgdaUnderscore{}}\AgdaSpace{}%
\AgdaOperator{\AgdaInductiveConstructor{,}}\AgdaSpace{}%
\AgdaFunction{∙-refl}\<%
\\
\>[0]\<%
\\
%
\>[8]\AgdaFunction{concur-elab}\AgdaSpace{}%
\AgdaSymbol{:}%
\>[3262I]\AgdaFunction{Elaboration}\<%
\\
\>[3262I][@{}l@{\AgdaIndent{0}}]%
\>[27]\AgdaSymbol{(}\AgdaFunction{Lift}\AgdaSpace{}%
\AgdaFunction{Output}\AgdaSpace{}%
\AgdaOperator{\AgdaFunction{∔}}\AgdaSpace{}%
\AgdaFunction{Concur}\AgdaSpace{}%
\AgdaOperator{\AgdaFunction{∔}}\AgdaSpace{}%
\AgdaFunction{Lift}\AgdaSpace{}%
\AgdaFunction{Nil}\AgdaSymbol{)}\<%
\\
%
\>[27]\AgdaSymbol{(}%
\>[30]\AgdaFunction{CC}\AgdaSpace{}%
\AgdaSymbol{(λ}\AgdaSpace{}%
\AgdaBound{t}\AgdaSpace{}%
\AgdaSymbol{→}\AgdaSpace{}%
\AgdaOperator{\AgdaField{⟦}}\AgdaSpace{}%
\AgdaBound{t}\AgdaSpace{}%
\AgdaOperator{\AgdaField{⟧ᵀ}}\AgdaSpace{}%
\AgdaSymbol{→}\AgdaSpace{}%
\AgdaDatatype{Free}\AgdaSpace{}%
\AgdaSymbol{(}\AgdaFunction{Output}\AgdaSpace{}%
\AgdaOperator{\AgdaFunction{⊕}}\AgdaSpace{}%
\AgdaFunction{Nil}\AgdaSymbol{)}\AgdaSpace{}%
\AgdaDatatype{ℕ}\AgdaSymbol{)}\<%
\\
%
\>[27]\AgdaOperator{\AgdaFunction{⊕}}\AgdaSpace{}%
\AgdaFunction{Output}\<%
\\
%
\>[27]\AgdaOperator{\AgdaFunction{⊕}}\AgdaSpace{}%
\AgdaFunction{Nil}\AgdaSpace{}%
\AgdaSymbol{)}\<%
\\
%
\>[8]\AgdaFunction{concur-elab}\AgdaSpace{}%
\AgdaSymbol{=}\AgdaSpace{}%
\AgdaFunction{eLift}\AgdaSpace{}%
\AgdaOperator{\AgdaFunction{⋎}}\AgdaSpace{}%
\AgdaFunction{eConcur}\AgdaSpace{}%
\AgdaOperator{\AgdaFunction{⋎}}\AgdaSpace{}%
\AgdaFunction{eNil}\<%
\end{code}
\begin{code}%
%
\>[8]\AgdaFunction{test-ex-01234}\AgdaSpace{}%
\AgdaSymbol{:}%
\>[25]\AgdaFunction{un}\AgdaSpace{}%
\AgdaSymbol{(}%
\>[31]\AgdaSymbol{(}%
\>[34]\AgdaOperator{\AgdaFunction{given}}\AgdaSpace{}%
\AgdaFunction{hOut}\<%
\\
%
\>[34]\AgdaOperator{\AgdaFunction{handle}}\AgdaSpace{}%
\AgdaSymbol{(}%
\>[44]\AgdaSymbol{(}%
\>[47]\AgdaOperator{\AgdaFunction{given}}\AgdaSpace{}%
\AgdaFunction{hCC}\<%
\\
%
\>[47]\AgdaOperator{\AgdaFunction{handle}}\AgdaSpace{}%
\AgdaSymbol{(}\AgdaFunction{elaborate}\AgdaSpace{}%
\AgdaFunction{concur-elab}\AgdaSpace{}%
\AgdaFunction{ex-01234}\AgdaSymbol{)}\<%
\\
%
\>[44]\AgdaSymbol{)}\AgdaSpace{}%
\AgdaInductiveConstructor{tt}\AgdaSpace{}%
\AgdaSymbol{)}\AgdaSpace{}%
\AgdaSymbol{)}\AgdaSpace{}%
\AgdaInductiveConstructor{tt}\AgdaSpace{}%
\AgdaSymbol{)}\AgdaSpace{}%
\AgdaOperator{\AgdaDatatype{≡}}\AgdaSpace{}%
\AgdaSymbol{(}\AgdaNumber{0}\AgdaSpace{}%
\AgdaOperator{\AgdaInductiveConstructor{,}}\AgdaSpace{}%
\AgdaString{"03142"}\AgdaSymbol{)}\<%
\\
%
\>[8]\AgdaFunction{test-ex-01234}\AgdaSpace{}%
\AgdaSymbol{=}\AgdaSpace{}%
\AgdaInductiveConstructor{refl}\<%
\end{code}
%
The following program spawns an additional thread with an \ad{‵atomic} block
%
\begin{code}[hide]%
%
\>[6]\AgdaKeyword{module}\AgdaSpace{}%
\AgdaModule{\AgdaUnderscore{}}\AgdaSpace{}%
\AgdaKeyword{where}\<%
\\
\>[6][@{}l@{\AgdaIndent{0}}]%
\>[8]\AgdaKeyword{private}\AgdaSpace{}%
\AgdaKeyword{instance}\<%
\\
\>[8][@{}l@{\AgdaIndent{0}}]%
\>[10]\AgdaFunction{x₀}\AgdaSpace{}%
\AgdaSymbol{:}\AgdaSpace{}%
\AgdaFunction{Lift}\AgdaSpace{}%
\AgdaFunction{Output}\AgdaSpace{}%
\AgdaOperator{\AgdaFunction{≲ᴴ}}\AgdaSpace{}%
\AgdaSymbol{(}\AgdaFunction{Lift}\AgdaSpace{}%
\AgdaFunction{Output}\AgdaSpace{}%
\AgdaOperator{\AgdaFunction{∔}}\AgdaSpace{}%
\AgdaFunction{Concur}\AgdaSpace{}%
\AgdaOperator{\AgdaFunction{∔}}\AgdaSpace{}%
\AgdaFunction{Lift}\AgdaSpace{}%
\AgdaFunction{Nil}\AgdaSymbol{)}\<%
\\
%
\>[10]\AgdaFunction{x₀}\AgdaSpace{}%
\AgdaSymbol{=}\AgdaSpace{}%
\AgdaFunction{≲ᴴ-left}\<%
\\
%
\\[\AgdaEmptyExtraSkip]%
%
\>[10]\AgdaFunction{x₁}\AgdaSpace{}%
\AgdaSymbol{:}\AgdaSpace{}%
\AgdaFunction{Concur}\AgdaSpace{}%
\AgdaOperator{\AgdaFunction{≲ᴴ}}\AgdaSpace{}%
\AgdaSymbol{(}\AgdaFunction{Lift}\AgdaSpace{}%
\AgdaFunction{Output}\AgdaSpace{}%
\AgdaOperator{\AgdaFunction{∔}}\AgdaSpace{}%
\AgdaFunction{Concur}\AgdaSpace{}%
\AgdaOperator{\AgdaFunction{∔}}\AgdaSpace{}%
\AgdaFunction{Lift}\AgdaSpace{}%
\AgdaFunction{Nil}\AgdaSymbol{)}\<%
\\
%
\>[10]\AgdaFunction{x₁}\AgdaSpace{}%
\AgdaSymbol{=}\AgdaSpace{}%
\AgdaFunction{≲ᴴ-right}\AgdaSpace{}%
\AgdaSymbol{⦃}\AgdaSpace{}%
\AgdaFunction{≲ᴴ-left}\AgdaSpace{}%
\AgdaSymbol{⦄}\<%
\\
%
\\[\AgdaEmptyExtraSkip]%
%
\>[10]\AgdaFunction{y₀}\AgdaSpace{}%
\AgdaSymbol{:}\AgdaSpace{}%
\AgdaFunction{CC}\AgdaSpace{}%
\AgdaSymbol{(λ}\AgdaSpace{}%
\AgdaBound{t}\AgdaSpace{}%
\AgdaSymbol{→}\AgdaSpace{}%
\AgdaOperator{\AgdaField{⟦}}\AgdaSpace{}%
\AgdaBound{t}\AgdaSpace{}%
\AgdaOperator{\AgdaField{⟧ᵀ}}\AgdaSpace{}%
\AgdaSymbol{→}\AgdaSpace{}%
\AgdaDatatype{Free}\AgdaSpace{}%
\AgdaSymbol{(}\AgdaFunction{Output}\AgdaSpace{}%
\AgdaOperator{\AgdaFunction{⊕}}\AgdaSpace{}%
\AgdaFunction{Nil}\AgdaSymbol{)}\AgdaSpace{}%
\AgdaDatatype{ℕ}\AgdaSymbol{)}\AgdaSpace{}%
\AgdaOperator{\AgdaFunction{≲}}\AgdaSpace{}%
\AgdaSymbol{(}\AgdaFunction{CC}\AgdaSpace{}%
\AgdaSymbol{(λ}\AgdaSpace{}%
\AgdaBound{t}\AgdaSpace{}%
\AgdaSymbol{→}\AgdaSpace{}%
\AgdaOperator{\AgdaField{⟦}}\AgdaSpace{}%
\AgdaBound{t}\AgdaSpace{}%
\AgdaOperator{\AgdaField{⟧ᵀ}}\AgdaSpace{}%
\AgdaSymbol{→}\AgdaSpace{}%
\AgdaDatatype{Free}\AgdaSpace{}%
\AgdaSymbol{(}\AgdaFunction{Output}\AgdaSpace{}%
\AgdaOperator{\AgdaFunction{⊕}}\AgdaSpace{}%
\AgdaFunction{Nil}\AgdaSymbol{)}\AgdaSpace{}%
\AgdaDatatype{ℕ}\AgdaSymbol{)}\AgdaSpace{}%
\AgdaOperator{\AgdaFunction{⊕}}\AgdaSpace{}%
\AgdaFunction{Output}\AgdaSpace{}%
\AgdaOperator{\AgdaFunction{⊕}}\AgdaSpace{}%
\AgdaFunction{Nil}\AgdaSymbol{)}\<%
\\
%
\>[10]\AgdaFunction{y₀}\AgdaSpace{}%
\AgdaSymbol{=}\AgdaSpace{}%
\AgdaFunction{≲-left}\<%
\\
%
\\[\AgdaEmptyExtraSkip]%
%
\>[10]\AgdaFunction{y₁}\AgdaSpace{}%
\AgdaSymbol{:}\AgdaSpace{}%
\AgdaFunction{Output}\AgdaSpace{}%
\AgdaOperator{\AgdaFunction{≲}}\AgdaSpace{}%
\AgdaSymbol{(}\AgdaFunction{CC}\AgdaSpace{}%
\AgdaSymbol{(λ}\AgdaSpace{}%
\AgdaBound{t}\AgdaSpace{}%
\AgdaSymbol{→}\AgdaSpace{}%
\AgdaOperator{\AgdaField{⟦}}\AgdaSpace{}%
\AgdaBound{t}\AgdaSpace{}%
\AgdaOperator{\AgdaField{⟧ᵀ}}\AgdaSpace{}%
\AgdaSymbol{→}\AgdaSpace{}%
\AgdaDatatype{Free}\AgdaSpace{}%
\AgdaSymbol{(}\AgdaFunction{Output}\AgdaSpace{}%
\AgdaOperator{\AgdaFunction{⊕}}\AgdaSpace{}%
\AgdaFunction{Nil}\AgdaSymbol{)}\AgdaSpace{}%
\AgdaDatatype{ℕ}\AgdaSymbol{)}\AgdaSpace{}%
\AgdaOperator{\AgdaFunction{⊕}}\AgdaSpace{}%
\AgdaFunction{Output}\AgdaSpace{}%
\AgdaOperator{\AgdaFunction{⊕}}\AgdaSpace{}%
\AgdaFunction{Nil}\AgdaSymbol{)}\<%
\\
%
\>[10]\AgdaFunction{y₁}\AgdaSpace{}%
\AgdaSymbol{=}\AgdaSpace{}%
\AgdaFunction{≲-right}\AgdaSpace{}%
\AgdaSymbol{⦃}\AgdaSpace{}%
\AgdaFunction{≲-left}\AgdaSpace{}%
\AgdaSymbol{⦄}\<%
\\
%
\\[\AgdaEmptyExtraSkip]%
%
\>[10]\AgdaFunction{y₂}\AgdaSpace{}%
\AgdaSymbol{:}\AgdaSpace{}%
\AgdaFunction{Output}\AgdaSpace{}%
\AgdaOperator{\AgdaFunction{≲}}\AgdaSpace{}%
\AgdaField{proj₁}\AgdaSpace{}%
\AgdaFunction{y₀}\<%
\\
%
\>[10]\AgdaFunction{y₂}\AgdaSpace{}%
\AgdaSymbol{=}\AgdaSpace{}%
\AgdaSymbol{\AgdaUnderscore{}}\AgdaSpace{}%
\AgdaOperator{\AgdaInductiveConstructor{,}}\AgdaSpace{}%
\AgdaFunction{∙-refl}\<%
\end{code}
\begin{code}%
%
\>[8]\AgdaFunction{ex-01234567}\AgdaSpace{}%
\AgdaSymbol{:}\AgdaSpace{}%
\AgdaDatatype{Hefty}\AgdaSpace{}%
\AgdaSymbol{(}\AgdaFunction{Lift}\AgdaSpace{}%
\AgdaFunction{Output}\AgdaSpace{}%
\AgdaOperator{\AgdaFunction{∔}}\AgdaSpace{}%
\AgdaFunction{Concur}\AgdaSpace{}%
\AgdaOperator{\AgdaFunction{∔}}\AgdaSpace{}%
\AgdaFunction{Lift}\AgdaSpace{}%
\AgdaFunction{Nil}\AgdaSymbol{)}\AgdaSpace{}%
\AgdaDatatype{ℕ}\<%
\\
%
\>[8]\AgdaFunction{ex-01234567}\AgdaSpace{}%
\AgdaSymbol{=}\AgdaSpace{}%
\AgdaFunction{‵spawn}%
\>[30]\AgdaFunction{ex-01234}\<%
\\
%
\>[30]\AgdaSymbol{(}\AgdaFunction{‵atomic}\AgdaSpace{}%
\AgdaSymbol{(}\AgdaKeyword{do}\AgdaSpace{}%
\AgdaOperator{\AgdaFunction{↑}}\AgdaSpace{}%
\AgdaInductiveConstructor{out}\AgdaSpace{}%
\AgdaString{"5"}\AgdaSymbol{;}\AgdaSpace{}%
\AgdaOperator{\AgdaFunction{↑}}\AgdaSpace{}%
\AgdaInductiveConstructor{out}\AgdaSpace{}%
\AgdaString{"6"}\AgdaSymbol{;}\AgdaSpace{}%
\AgdaOperator{\AgdaFunction{↑}}\AgdaSpace{}%
\AgdaInductiveConstructor{out}\AgdaSpace{}%
\AgdaString{"7"}\AgdaSymbol{;}\AgdaSpace{}%
\AgdaInductiveConstructor{pure}\AgdaSpace{}%
\AgdaNumber{0}\AgdaSymbol{))}\<%
\end{code}
%
Inspecting the output, we see that the additional thread indeed computes atomically:
%
\begin{code}%
%
\>[8]\AgdaFunction{test-ex-01234567}\AgdaSpace{}%
\AgdaSymbol{:}%
\>[28]\AgdaFunction{un}\AgdaSpace{}%
\AgdaSymbol{(}%
\>[34]\AgdaSymbol{(}%
\>[37]\AgdaOperator{\AgdaFunction{given}}\AgdaSpace{}%
\AgdaFunction{hOut}\<%
\\
%
\>[37]\AgdaOperator{\AgdaFunction{handle}}\AgdaSpace{}%
\AgdaSymbol{(}%
\>[47]\AgdaSymbol{(}%
\>[50]\AgdaOperator{\AgdaFunction{given}}\AgdaSpace{}%
\AgdaFunction{hCC}\<%
\\
%
\>[50]\AgdaOperator{\AgdaFunction{handle}}\AgdaSpace{}%
\AgdaSymbol{(}\AgdaFunction{elaborate}\AgdaSpace{}%
\AgdaFunction{concur-elab}\AgdaSpace{}%
\AgdaFunction{ex-01234567}\AgdaSymbol{)}\<%
\\
%
\>[47]\AgdaSymbol{)}\AgdaSpace{}%
\AgdaInductiveConstructor{tt}\AgdaSpace{}%
\AgdaSymbol{)}\AgdaSpace{}%
\AgdaSymbol{)}\AgdaSpace{}%
\AgdaInductiveConstructor{tt}\AgdaSpace{}%
\AgdaSymbol{)}\AgdaSpace{}%
\AgdaOperator{\AgdaDatatype{≡}}\AgdaSpace{}%
\AgdaSymbol{(}\AgdaNumber{0}\AgdaSpace{}%
\AgdaOperator{\AgdaInductiveConstructor{,}}\AgdaSpace{}%
\AgdaString{"05673142"}\AgdaSymbol{)}\<%
\\
%
\>[8]\AgdaFunction{test-ex-01234567}\AgdaSpace{}%
\AgdaSymbol{=}\AgdaSpace{}%
\AgdaInductiveConstructor{refl}\<%
\end{code}
%
\begin{code}[hide]%
\>[0]\AgdaComment{--\ \ \ \ \ \ \ concur-elab′\ :\ Elaboration}\<%
\\
\>[0]\AgdaComment{--\ \ \ \ \ \ \ \ \ \ \ \ \ \ \ \ \ \ \ \ \ \ \ \ \ \ (Lift\ Output\ ∔\ Concur\ ∔\ Lift\ Nil)}\<%
\\
\>[0]\AgdaComment{--\ \ \ \ \ \ \ \ \ \ \ \ \ \ \ \ \ \ \ \ \ \ \ \ \ \ (\ \ Output}\<%
\\
\>[0]\AgdaComment{--\ \ \ \ \ \ \ \ \ \ \ \ \ \ \ \ \ \ \ \ \ \ \ \ \ \ ⊕\ CC\ (λ\ t\ →\ ⟦\ t\ ⟧ᵀ\ →\ Free\ Nil\ (ℕ\ ×\ String))}\<%
\\
\>[0]\AgdaComment{--\ \ \ \ \ \ \ \ \ \ \ \ \ \ \ \ \ \ \ \ \ \ \ \ \ \ ⊕\ Nil\ )}\<%
\\
\>[0]\AgdaComment{--\ \ \ \ \ \ \ concur-elab′\ =\ eLift\ ⋎\ eConcur\ ⋎\ eNil}\<%
\\
\>[0]\AgdaComment{--\ }\<%
\\
\>[0]\AgdaComment{--\ \ \ \ \ \ \ test-ex′\ :\ un\ (\ \ (\ \ given\ hCC}\<%
\\
\>[0]\AgdaComment{--\ \ \ \ \ \ \ \ \ \ \ \ \ \ \ \ \ \ \ \ \ \ \ \ \ \ \ handle\ (\ \ (\ \ given\ hOut}\<%
\\
\>[0]\AgdaComment{--\ \ \ \ \ \ \ \ \ \ \ \ \ \ \ \ \ \ \ \ \ \ \ \ \ \ \ \ \ \ \ \ \ \ \ \ \ \ \ \ handle\ (elaborate\ concur-elab′\ ex-01234)\ )}\<%
\\
\>[0]\AgdaComment{--\ \ \ \ \ \ \ \ \ \ \ \ \ \ \ \ \ \ \ \ \ \ \ \ \ \ \ \ \ \ \ \ \ \ \ \ \ tt\ )\ )\ tt\ )\ ≡\ (0\ ,\ "03142")}\<%
\\
\>[0]\AgdaComment{--\ \ \ \ \ \ \ test-ex′\ =\ refl}\<%
\\
\>[0]\AgdaComment{--\ }\<%
\\
\>[0]\AgdaComment{--\ \ \ \ \ \ \ ex-atomic-01234\ :\ Hefty\ (Lift\ Output\ ∔\ Concur\ ∔\ Lift\ Nil)\ ℕ}\<%
\\
\>[0]\AgdaComment{--\ \ \ \ \ \ \ ex-atomic-01234\ =\ ‵spawn\ (‵atomic\ (do\ ↑\ out\ "0";\ ↑\ out\ "1";\ ↑\ out\ "2";\ pure\ 0))\ (do\ ↑\ out\ "3";\ ↑\ out\ "4";\ pure\ 0)}\<%
\\
\>[0]\AgdaComment{--\ }\<%
\\
\>[0]\AgdaComment{--\ \ \ \ \ \ \ --\ ordering\ of\ handlers\ matters!}\<%
\\
\>[0]\AgdaComment{--\ \ \ \ \ \ \ test-atomic-ex\ :\ un\ ((given\ hCC\ handle\ ((given\ hOut\ handle\ (elaborate\ concur-elab′\ ex-atomic-01234))\ tt))\ tt)\ ≡\ (0\ ,\ "34")}\<%
\\
\>[0]\AgdaComment{--\ \ \ \ \ \ \ test-atomic-ex\ =\ refl}\<%
\end{code}

The example above is inspired by the resumption monad, and in particular by the scoped effects definition of concurrency due to \citet{YangPWBS22}.
\citeauthor{YangPWBS22} do not (explicitly) consider how to define the concurrency operations in a modular style.
Instead, they give a direct semantics that translates to the resumption monad which we can encode as follows in Agda (assuming resumptions are given by the free monad):
%
\begin{code}%
\>[0][@{}l@{\AgdaIndent{1}}]%
\>[2]\AgdaKeyword{data}\AgdaSpace{}%
\AgdaDatatype{Resumption}\AgdaSpace{}%
\AgdaBound{Δ}\AgdaSpace{}%
\AgdaBound{A}\AgdaSpace{}%
\AgdaSymbol{:}\AgdaSpace{}%
\AgdaPrimitive{Set}\AgdaSpace{}%
\AgdaKeyword{where}\<%
\\
\>[2][@{}l@{\AgdaIndent{0}}]%
\>[4]\AgdaInductiveConstructor{done}%
\>[10]\AgdaSymbol{:}\AgdaSpace{}%
\AgdaBound{A}%
\>[37]\AgdaSymbol{→}\AgdaSpace{}%
\AgdaDatatype{Resumption}\AgdaSpace{}%
\AgdaBound{Δ}\AgdaSpace{}%
\AgdaBound{A}\<%
\\
%
\>[4]\AgdaInductiveConstructor{more}%
\>[10]\AgdaSymbol{:}\AgdaSpace{}%
\AgdaDatatype{Free}\AgdaSpace{}%
\AgdaBound{Δ}\AgdaSpace{}%
\AgdaSymbol{(}\AgdaDatatype{Resumption}\AgdaSpace{}%
\AgdaBound{Δ}\AgdaSpace{}%
\AgdaBound{A}\AgdaSymbol{)}%
\>[37]\AgdaSymbol{→}\AgdaSpace{}%
\AgdaDatatype{Resumption}\AgdaSpace{}%
\AgdaBound{Δ}\AgdaSpace{}%
\AgdaBound{A}\<%
\end{code}
%
We could elaborate into this type using a hefty algebra \ad{Algᴴ}~\ad{Concur}~\as{(}\ad{Resumption}~\ab{Δ}\as{)} but that would be incompatible with our other elaborations which use the free monad.
For that reason, we emulate the resumption monad using the free monad instead of using the \ad{Resumption} type directly.


%%% Local Variables:
%%% reftex-default-bibliography: ("../references.bib")
%%% End:


\input{tex/sections/4-laws.tex}
\input{tex/sections/6-related.tex}
\input{tex/sections/7-conclusion.tex}
% 
\paragraph*{Acknowledgements}
We thank the anonymous reviewers for their comments which helped improve the exposition of the paper.
  Furthermore, we thank Nicolas Wu, Andrew Tolmach, Peter Mosses, and Jaro Reinders for feedback on earlier drafts.
  This research was partially funded by the NWO VENI Composable and Safe-by-Construction Programming Language Definitions
project (VI.Veni.192.259).

\label{lastpage01}

%% Bibliography
\bibliographystyle{JFPlike}
\bibliography{tex/references}

\end{document}
